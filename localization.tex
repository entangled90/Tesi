%!TEX root = tesi.tex


\chapter{Superconformal index and partition functions}

In this chapter we will focus on the description of the superconformal index in four dimensions and its relation to partition functions in three dimensions.\\
The superconformal index is a quantity that can be calculated in superconformal field theories that counts the number of BPS states with a scale dimension that saturates a unitarity bound.\\
For $\mN=4$ field theories it was first introduced in \cite{Kinney:2005ej}
and it was later modified by Romelsberger for $\mN=1$ field theories in \cite{Romelsberger:2005eg}





\section{Superconformal index}
The superconformal index can be introduced as a generalization of the Witten index, which is defined as 
\begin{equation}
I = \Tr{ \left[   (-1)^F e^{-\beta E}\right]}
\end{equation}
where $F$ is the fermion number and $E$ the energy of the state.\\
The exponential term in the energy can be seen as a regulator, otherwise the counting would not be well defined.
However, the energy $E$ can be substituted with any operator with positive eigenvalues that commutes with the supercharges. 
There can be added other operators that commute with the supercharges in order to resolve the degeneracies.
\\
We are interested in four dimensional superconformal field theories which are naturally quantized in radial quantization on $\mathcal{S}^3 \times \mathcal{R}$. 
As a result, the isometry group is now $SU(2)_L \times SU(2)_R \times \mathcal{R}$ and the eight supercharges split into doublets of $SU(2)_L$ for $Q_{\alpha}$ and of $SU(2)_R$ for $S_{\alpha}$.\\
We can pick one of the supercharges, e.g. $Q_1$, and from the superconformal algebra we have the relation
\begin{equation}
 \{Q_1, Q_1^{\dagger} \} = H - \frac{3}{2} R - 2 J_3 = \Delta
 \end{equation} 
 where $H$ is the hamiltonian in radial quantization, $R$ is the R-charge generator and $J_3$ is one of the generators of $SU(2)_L$. 
 However, in a free theory $\Delta$ has an infinite number of ground stats. 
 For this reason the operator used as a regulator is \cite{Romelsberger:2005eg}
 \begin{equation}
 	\epsilon = H - \frac{1}{2} \geq \frac{2}{3} H 
 \end{equation}
As we said before, we can add other terms in order to differentiate the various states.
Such terms are given by a generator $x $ of the $SU(2)_R$ group and generators $h$ of the internal symmetry group which commutes with the supercharges.\\
The superconformal index is then defined as 
\begin{equation}
 i( \mu, \gamma,h) =
  \Tr  \left[  (-1)^F e^{ - \mu \epsilon } x h   \right] = 
 \Tr \left[  (-1)^F
  e^{- \mu (\Delta + R + 2 J^3)}  x h   \right]
 \end{equation} 
The index is invariant under continuos deformation of the theory that preserve the operators appearing in the formula.
It may seems a trivial fact, but the index is invariant under some non-conformal deformation.
As a result, it may be computed in a weakly coupled UV theory that flows to a non trivial conformal fixed point, such as the theories appearing in electric magnetic dualities. \\
The superconformal index can be calculated in two steps.
First, we need to obtain the index on "single particle states".\\
After defining $t = e^{-\mu}$, the index for chiral multiples $\Phi_i$ with R-charge $r_i$, with flavour symmetry group $F_i$ and gauge group $G_i$ reads
\begin{equation}
i_{\Phi}(t,x,h,g) = \sum_{i} \frac{ t^{r_i} \chi_{F_i}(h) \, \chi_{G_i}(g) - t^{2-r_i} \chi_{\bar{F}_i}(h) \chi_{\bar{G}_i}(g)}{ (1-tx) (1- t x^{-1})}
\end{equation}
where $h$ and $g$ are the chemical potential for the global and gauge symmetry group respectively. $\chi_{F_i}(h)$ and $\chi_{G_i}(g)$ are the character of the representation of $\Phi_i$.

The index for a vector superfield in the adjoint representation of a gauge group $G$ is given by
\begin{equation}
i_{V} (t,x,g) = \frac{2 t^2 - t(x + x^{-1})}{(1-tx) (1 - tx^{-1})} \chi_{adj}(g)
\end{equation}
The index for the theory is given by the sum of indices on single particle states. 
Then, the superconformal index can be calculate by taking the Plethystic exponential \cite{Feng:2007ur} of the single particle index
\begin{equation}
  I(t,x,h) = \int_{G} \hbox{d} \mu(g) \exp{ \left(  \sum_{n=1}^{\infty}i(t^n,x^n,h^n,g^n) \right)}
  \end{equation}  
  where $\hbox{d} \mu(g)$ is a $G$ invariant measure.\\


