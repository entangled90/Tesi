%!TEX root = tesi.tex


\chapter{Superconformal index and partition functions}

In this chapter we will introduce the superconformal index in four dimensions and we will describe how it can be used in order to obtain the expression of partition functions in three dimensions.\\
{\color{red}{ \LARGE Devo Aggiunger qualcosa}}


\section{Superconformal index}
The superconformal index is a quantity that can be calculated in superconformal field theories that counts the number of BPS states with a scale dimension that saturates a unitarity bound.\\
For $\mN=4$ field theories it was first introduced in \cite{Kinney:2005ej}
and it was later modified by Romelsberger for $\mN=1$ field theories in \cite{Romelsberger:2005eg}.\\
The superconformal index can be introduced as a generalization of the Witten index, which is defined as 
\begin{equation}
I = \Tr{ \left[   (-1)^F e^{-\beta E}\right]}
\end{equation}
where $F$ is the fermion number and $E$ the energy of the state.\\
The exponential term in the energy can be seen as a regulator, otherwise the counting would not be well defined.
However, the energy $E$ can be substituted with any operator with positive eigenvalues that commutes with the supercharges. 
There can be added other operators that commute with the supercharges in order to resolve the degeneracies.
\\
We are interested in four dimensional superconformal field theories which are naturally quantized in radial quantization on $\mathcal{S}^3 \times \mathcal{R}$. 
As a result, the isometry group is now $SU(2)_L \times SU(2)_R \times \mathcal{R}$ and the eight supercharges split into doublets of $SU(2)_L$ for $Q_{\alpha}$ and of $SU(2)_R$ for $S_{\alpha}$.\\
We can pick one of the supercharges, e.g. $Q_1$, and from the superconformal algebra we have the relation
\begin{equation}
 \{Q_1, Q_1^{\dagger} \} = H + \frac{3}{2} R - 2 J_1 = \Delta
 \end{equation} 
 where $H$ is the hamiltonian in radial quantization, $R$ is the R-charge generator and $J_1$ is one of the generators of $SU(2)_L$. 
 However, in a free theory $\Delta$ has an infinite number of ground stats. 
 For this reason the operator used as a regulator is \cite{Romelsberger:2005eg}
 \begin{equation}
 	\epsilon = H - \frac{1}{2} R \geq \frac{2}{3} H 
 \end{equation}
As we said before, we can add other terms in order to differentiate the various states.
Such terms are given by the Cartan generators $J_1, J_2$ of the $SU(2)_L \times SU(2)_R$ isometry of the three sphere and by the generators $e_a$ of the internal symmetry group.\\
The superconformal index on the squashed sphere $\mathbb{S}_b^3 \times \mathbb{S}^1$ is then defined as \cite{Aharony:2013dha}
\begin{equation}
 i( \mu, \gamma,h) =
  \Tr  \left[  (-1)^F e^{ - \mu \epsilon } p^{J_1 + J_2 - \frac{R}{2} q^{J_1 - J_2 - \frac{R}{2}} \prod_a u_a^{e_a} }   \right]
 \end{equation} 
When we have $p=q$ the index reduces to the index on the round sphere $\mathbb{S}^3 \times \mathbb{S}^1$ \cite{Dolan:2008qi}.\\
In order for the index to be real and for convergence, the values $p,q$ are taken to satisfy the following conditions
\begin{equation}
\hbox{Im} (pq) =0 \qquad |p/q| = 1 \qquad | pq| < 1
\end{equation}
The index is invariant under continuous deformation of the theory that preserve the operators appearing in the formula.
It may seems a trivial fact, but the index is invariant under some non-conformal deformation.
As a result, it may be computed in a weakly coupled UV theory that flows to a non trivial conformal fixed point, such as the theories appearing in electric magnetic dualities \cite{Romelsberger:2007ec}. \\
The superconformal index can be calculated in two steps.
First, we need to obtain the index on \emph{single particle states}.\\
After defining $t = e^{-\mu}$, the index for chiral multiples $\Phi_i$ with R-charge $r_i$, with flavour symmetry group $F_i$ and gauge group $G_i$ reads
\begin{equation}
i_{\Phi}(t,x,h,g) = \sum_{i} \frac{ t^{r_i} \chi_{F_i}(h) \, \chi_{G_i}(g) - t^{2-r_i} \chi_{\bar{F}_i}(h) \chi_{\bar{G}_i}(g)}{ (1-tx) (1- t x^{-1})}
\end{equation}
where $h$ and $g$ are the chemical potential for the global and gauge symmetry group respectively. $\chi_{F_i}(h)$ and $\chi_{G_i}(g)$ are the characters of the representation of $\Phi_i$.\\
The index for a vector superfield in the adjoint representation of a gauge group $G$ is given by \cite{Dolan:2008qi}
\begin{equation}
i_{V} (t,x,g) = \frac{2 t^2 - t(x + x^{-1})}{(1-tx) (1 - tx^{-1})} \chi_{adj}(g)
\end{equation}
The index on the single particle states for the complete theory is given by the sum of indices for every field. 
Then, the superconformal index can be calculated by taking the Plethystic exponential \cite{Feng:2007ur} of the full single particle index
\begin{equation}
  I(t,x,h) = \int_{G} \hbox{d} \mu(g)\,  \exp{ \left(  \sum_{n=1}^{\infty}i(t^n,x^n,h^n,g^n) \right)}
  \end{equation}  
where $\hbox{d} \mu(g)$ is a $G$ invariant measure.\\
The calculation of the index can be performed using the following mathematical identities involving the elliptic Gamma function
\begin{align}
\Gamma_e ( y; p, q) \overset{def}{=} & \prod_{j,k \geq 0} \frac{1 - y^{-1}p^{j+1}q^{k+1}}{1 - y p^j q^k} = \exp \left(  
\sum_{n=1}^{\infty} \frac{1}{n} \, i_S(p^n,q^n,y^n)
\right)\\
& \; \hbox{where} \quad i_S(p,q,y) = \frac{y - pq/y}{(1-p)(1-q)}
\end{align}
The \emph{single particle index} for a generic chiral field can be written as $i_S$ with the generic change of variables, including a generic chemical potential $z$ for the global symmetries. 
\begin{equation}
p = tx \qquad q = t x^{-1} \qquad y = t^r z
\end{equation}
With the same substitution we can cast the \emph{single particle index} for a vector multiplet into
\begin{equation}
i_V(p,q) = - \frac{p}{1-p} - \frac{q}{1-q}
\end{equation}
and use the following identities, containing the chemical potential associated to the gauge group $z$
\begin{align}
\exp \left(  
\sum_{n=1}^{\infty} \frac{1}{n}\, i_V(p^n,q^n) ( z^n + z^{-n}) \right) & = \frac{1}{(1-z)(1-z^{-1}) \Gamma_e(z;p,q) \Gamma_e(z^{-1};p,q)}
\\
\exp \left(
\sum_{n=1}^{\infty} \frac{1}{n} \, i_V(p^n,q^n) \right) & = (p;p) (q;q)
\end{align}
and the \emph{Q-Pochhammer symbol} is defined as
\begin{equation}
(x;p) = \prod_{j \geq 0} ( 1- x p^j)
\end{equation}
Using these identities we can reduce the calculation of the superconformal index to a integral over the gauge group of products of elliptic Gamma functions.\\
The superconformal index can be calculated for pairs of dual theories and many mathematical integral identities are known that demonstrate the equality of their indices \cite{rains309252transformations} \cite{Dolan:2008qi}. 
This is a rather important point, since it will be the starting point of our analysis.\\
Our interest in the superconformal index follows from the fact that we can reduce the index of a four dimensional theory compactified on $\mathbb{S}^3 \times \mathbb{S}^1$ to the partition function of the three dimensional theory obtained in the limit $r_{\mathbb{S}^1} \rightarrow 0$.
% and with $\eta$ superpotential.  

\subsection{Localization}



