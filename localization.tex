%!TEX root = tesi.tex


\chapter{Superconformal index and partition functions}

In this chapter we will introduce the superconformal index in four dimensions and we will describe how it can be used in order to obtain the expression of partition functions in three dimensions.\\
{\color{red}{ \LARGE Devo Aggiunger qualcosa}}


\section{Superconformal index}
The superconformal index is a quantity that can be calculated in superconformal field theories that counts the number of BPS states with a scale dimension that saturates a unitarity bound.\\
For $\mN=4$ field theories it was first introduced in \cite{Kinney:2005ej}
and it was later modified by Romelsberger for $\mN=1$ field theories in \cite{Romelsberger:2005eg}.\\
The superconformal index can be introduced as a generalization of the Witten index, which is defined as 
\begin{equation}
I = \Tr{ \left[   (-1)^F e^{-\beta E}\right]}
\end{equation}
where $F$ is the fermion number and $E$ the energy of the state.\\
The exponential term in the energy can be seen as a regulator, otherwise the counting would not be well defined.
However, the energy $E$ can be substituted with any operator with positive eigenvalues that commutes with the supercharges. 
There can be added other operators that commute with the supercharges in order to resolve the degeneracies.
\\
We are interested in four dimensional superconformal field theories which are naturally quantized in radial quantization on $S^3 \times \mathbb{R}$. 
As a result, the isometry group is now $SU(2)_L \times SU(2)_R \times \mathbb{R}$ and the eight supercharges split into doublets of $SU(2)_L$ for $Q_{\alpha}$ and of $SU(2)_R$ for $S_{\alpha}$.\\
We can pick one of the supercharges, e.g. $Q_1$, and from the superconformal algebra we have the relation
\begin{equation}
 \{Q_1, Q_1^{\dagger} \} = H - \frac{3}{2} R - 2 J_1 = \Delta
 \end{equation} 
 where $H$ is the Hamiltonian in radial quantization, $R$ is the R-charge generator and $J_1$ is one of the generators of $SU(2)_L$. 
 However, in a free theory $\Delta$ has an infinite number of ground states. 
 For this reason the operator used as a regulator is \cite{Romelsberger:2005eg}
 \begin{equation}
 	\Xi = H - \frac{1}{2} R \geq \frac{2}{3} H 
 \end{equation}
As we said before, we can add other terms in order to differentiate the various states.
Such terms are given by the Cartan generators $J_1, J_2$ of the $SU(2)_L \times SU(2)_R$ isometry of the three-sphere and by the generators $e_a$ of the internal symmetry group.\\
The superconformal index on the sphere $S^3 \times S^1$ is then defined as \cite{Romelsberger:2007ec}
\begin{align}
 i( x, h_a) =&
%  \Tr  \left[  (-1)^F e^{ - \mu \Xi } p^{J_1 + J_2 - \frac{R}{2}} q^{J_1 - J_2 - \frac{R}{2} } \prod_a u_a^{e_a}    \right]
\Tr \left[ (-1)^F  e^{ - \mu \Xi} x^{2 J_2} \prod_a h_a^{e_a} \right] \\= &
\Tr \left[ (-1)^F  t^{\Delta} t^{ 2 J_1 - R} x^{2 J_2} \prod_i h_a^{e_a} \right] 
 \end{align} 
 where we defined $t = e^{- \mu}$.
WIth a change of variables, the superconformal index can be written as 
\begin{equation}
 i( p, q, h_a) =
 \Tr  \left[  (-1)^F 
 (pq)^{ \frac{\Xi}{2}} %e^{ - \mu \Xi }
  p^{J_1 + J_2 - \frac{R}{2}} q^{J_1 - J_2 - \frac{R}{2} } \prod_a u_a^{e_a}    \right]
\end{equation}
where $p,q$ are defined by
\begin{equation}
p = t x \quad q = t x^{-1}
\end{equation}
The values of $p,q$ must satisfy the following conditions for convergence and reality of the index
\begin{equation}
\hbox{Im} (pq) =0 \qquad |p/q| =  1 \qquad | pq| < 1
\label{eqn:reality_convergence_condition_index}
\end{equation}
The superconformal index receives contribution only from states with $\Delta=0$, since $J_1 \pm J_2 - \frac{R}{2}$ and the internal generators commute with $\Delta$.
%For $p=q$ the index on the squashed three-sphere reduces to the index on a round three-sphere.
%The index is invariant under continuous deformation of the theory that preserve the operators appearing in the formula.
%It may seems a trivial fact, but the index is invariant under some non-conformal deformation.
%As a result, it may be computed in a weakly coupled UV theory that flows to a non trivial conformal fixed point, such as the theories appearing in electric magnetic dualities \cite{Romelsberger:2007ec}. \\
The superconformal index can be calculated in two steps.
First, we need to obtain the index on \emph{single particle states}.\\
The index for chiral multiples $\Phi_i$ with R-charge $r_i$, with flavour symmetry group $F_i$ and gauge group $G_i$ reads \cite{Dolan:2008qi}
\begin{align}
i_{\Phi}(t,x,h,g) =&  \sum_{i} \frac{ t^{r_i} \chi_{F_i}(h) \, \chi_{G_i}(z) - t^{2-r_i} \chi_{\bar{F}_i}(h) \chi_{\bar{G}_i}(z)}{ (1-tx) (1- t x^{-1})} = \\
=& \sum_{i} \frac{ (pq)^{\frac{r_i}{2}} \chi_{F_i}(h) \, \chi_{G_i}(z) - (pq)^{1-\frac{r_i}{2}} \chi_{\bar{F}_i}(h) \chi_{\bar{G}_i}(z)}{ (1-p) (1- q)} 
\end{align}
where $h$ and $z$ are the chemical potential for the global and gauge symmetry group respectively. $\chi_{F_i}(h)$ and $\chi_{G_i}(z)$ are the characters of the representation of $\Phi_i$.\\
The index for a vector superfield in the adjoint representation of a gauge group $G$ is given by \cite{Dolan:2008qi}
\begin{equation}
i_{V} (t,x,g) = \frac{2 t^2 - t(x + x^{-1})}{(1-tx) (1 - tx^{-1})} \chi_{adj}(z) =  \frac{2 (pq) - (p+q )}{(1-p) (1 - q)} \chi_{adj}(z)
\end{equation}
The index on the single particle states for the complete theory is given by the sum of indices for every field. 
Then, the superconformal index can be calculated by taking the Plethystic exponential \cite{Feng:2007ur} of the full single particle index
\begin{equation}
  I(t,x,h) = \int_{G} \hbox{d} \mu(g)\,  \exp{ \left(  \sum_{n=1}^{\infty}i(t^n,x^n,h^n,z^n) \right)}
  \end{equation}  
where $\hbox{d} \mu(g)$ is a $G$ invariant measure.\\
The definition of the superconformal index can be extended to the manifold $S_b^3 \times S^1$, where $S_b^3$ is the squashed three-sphere whose isometry group is only $U(1)_L \times U(1)_R$ \cite{Hama:2011ea}.
The squashed sphere is defined by considering the metric 
\begin{align}
 d s^2 =& l^2 ( d x_0^2 + d x_1^2 ) + \bar{l}^2 ( d x_2^2 + d x_3^2) \\
 & \; \hbox{with} \qquad  x_0^2 + x_1^2 + x_2^2 + x_3^2 = 1
\end{align}
With the change of coordinates 
\begin{align}
x_0  & =  \cos \theta \cos \phi \\
x_1  & = \cos \theta \sin \phi \\
x_2  & = \sin \theta \cos \chi \\
x_3  & = \sin \theta \sin \chi 
\end{align}
we obtain the metric of the squashed sphere with squashing parameter 
$b = \frac{l}{\bar{l}}$
\begin{align}
 d s^2 = f(\theta)^2 d \,theta^2 + l^2 \cos^2 \theta d \phi ^2 + \bar{l}^2 \sin^2 \theta d \chi^2 \\
\hbox{with} \quad f( \theta) = \sqrt{ l^2 \sin^2 \theta + \bar{l}^2 \cos^2 \theta}
\end{align}
The index on the squashed sphere $S_b^3 \times S^1$ should coincide with the index on the round sphere $S^3 \times S^1$ since they have the same topology and the indices receive contributions from the same BPS states.
An explicit calculation \cite{Agarwal:2012hs} shows that this is indeed true. \\
The indices match up to a rescaling of the fugacities $(p,q) \, \rightarrow \, (p^{\frac{1}{a}}, q^{\frac{1}{b}})$ which does not modify the physical content of the index since they are arbitrary parameters and they still satisfy the conditions \eqref{eqn:reality_convergence_condition_index}.

An exact calculation of the index can be performed using the following mathematical identities involving the elliptic gamma function $\Gamma_e$.
The index of a chiral field in general can be written as
\begin{equation}
i_{\Phi} (p,q,y) = \frac{y - pq/y}{(1-p)(1-q)} \qquad \hbox{with} \quad p=tx \; q=t x^{-1} \; y = t^R
\end{equation}
where $z$ is the fugacity associated to gauge and/or global symmetries.
It can be written as an elliptic gamma function through
\begin{align}
 \exp \left(  
\sum_{n=1}^{\infty} \frac{1}{n} \, i_S(p^n,q^n,y^n) \right)=
& \prod_{j,k \geq 0} \frac{1 - y^{-1}p^{j+1}q^{k+1}}{1 - y p^j q^k}
\overset{def}{=} 
\Gamma_e ( y; p, q) 
\\
& \; \hbox{where} \quad i_S(p,q,y) = \frac{y - pq/y}{(1-p)(1-q)} \quad
\end{align}
The contribution to the index of chiral field in the (anti)fundamental representation of global and gauge symmetries can be generally
 written as
\begin{equation}
\Gamma_e \left(   (pq)^{\frac{R_i}{2}} \prod_a u_a^{e_a} ; p ,q \right)
\end{equation}
Using the same variables we can write the \emph{single particle index} of a vector multiplet into
\begin{equation}
i_V(p,q) = - \left( \frac{p}{1-p} + \frac{q}{1-q} \right)
\end{equation}
and use the following identities, where $z$ is the chemical potential associated to the gauge group
\begin{align}
\exp \left(  
\sum_{n=1}^{\infty} \frac{1}{n}\, i_V(p^n,q^n) ( z^n + z^{-n}) \right) & = \frac{1}{(1-z)(1-z^{-1}) \Gamma_e(z;p,q) \Gamma_e(z^{-1};p,q)}
\\
\exp \left(
\sum_{n=1}^{\infty} \frac{1}{n} \, i_V(p^n,q^n) \right) & = (p;p) (q;q)
\end{align}
and the \emph{Q-Pochhammer symbol} is defined as
\begin{equation}
(x;p) = \prod_{j \geq 0} ( 1- x p^j)
\end{equation}
The contribution of a vector field in the adjoint representation reads
\begin{equation}
(p;p)^{r_G} (q;q)^{r_G} \prod_{1 \leq i < j \leq N_c} \frac{1}{(1-\frac{z_i}{z_j})(1-\frac{z_j}{z_i}) \Gamma_e(\frac{z_i}{z_j};p,q) \Gamma_e(\frac{z_j}{z_i};p,q)}
\end{equation}
The calculation of the superconformal index is now reduced to a integral over the gauge group of products of elliptic gamma functions.\\
We will provide a more detailed calculation for the electric and magnetic theories of Kutasov-Schwimmer duality in the next chapter.
\\
The equality of the superconformal index for pairs of dual theories can be demonstrated using integral identities of elliptic Gamma functions that were first demonstrated by Rains in \cite{rains309252transformations}.
We will use this fact as a starting point in our discussion on the dimensional reduction of dualities to three dimensions.
%In fact, if we shrink the radius of $S^1 $ to zero the superconformal index reduces to the partition function of the three dimensional theory on $S_b^3$ obtained by dimensional reduction, i.e. with an $\eta$ superpotential.\\








\section{Localization}
Localization was first introduced on $S^4$ for $\mN=2,4$ theories in \cite{Pestun:2007rz} and then applied to $S^3$ and $S_b^3$ for $\mN=2$ theories in \cite{Kapustin:2009kz}, \cite{Hama:2010av} and \cite{Hama:2011ea}.
It can be used to calculate partition function or Wilson loops. \\
The basic principle about localization is the following.
Suppose that the theory in question admits a \emph{fermionic} symmetry $\delta$ which leaves the measure of the path integral invariant and that it squares to a \emph{bosonic} symmetry $\Delta_B$, which can be a Lorentz or gauge transformation.\\
Consider the following modified partition function, which correspond to the partition function we want to calculate for $t=0$
\begin{equation}
Z(t) = \int \mathcal{D} \phi \; e^{- S(\phi) - t (\delta V)}
\end{equation}
where $V$ is a fermionic operator invariant under $\Delta_B$, i.e. $ \Delta_B V = 0$.\\
This partition function is independent of $t$ since
\begin{equation}
 \frac{d Z}{d t} = - \int \mathcal{D} \phi \; \delta V \, e^{- S(\phi) - t (\delta V)} =  - \int  \mathcal{D} \phi \; \delta \left( e^{- S(\phi) - t (\delta V)  }\right) = 0
\end{equation}
Where the last equality holds if the fields decay sufficiently fast at infinity.\\
Since $Z(t)$ is independent of $t$ we can calculate it at $t=0$, which is the original partition function or at $t=\infty$, where the integral localizes on the saddle points of the integral if the bosonic part of $\delta V$ is positive definite.\\
In three dimensions, these saddle points are parametrized by the values of the scalar field in the vector multiplet and through gauge symmetry the integral can be reduced to the Cartan subalgebra. 
The saddle point conditions for the other fields set their value on the saddle points to zero.\\
The partition function receives contribution from the path integral only at one-loop and it can be computed exactly.
The one-loop term is given by a determinant of Laplace and Dirac operators for every field appearing in the lagrangian.
\begin{equation}
Z \sim \int [ d \sigma] \; e^{S_{saddle(\sigma)}} \prod_i \frac{ \Delta_{\psi_i}}{\Delta_{\phi_i}}
\label{eqn:localization_determinant_expression}
\end{equation}
The partition function can be written 
\begin{equation}
 Z (\mu_a , \nu_b) \sim \int \, [d  \sigma] \; e^{S_{saddle}(\sigma)} \; Z_{gauge}^{1-loop}(\sigma ) \; Z_{matter}^{1-loop}(\sigma,\mu_a,\nu_b)
\end{equation}
where $\mu_a,\nu_b$ are the real masses associated to global symmetries for left and right quarks. 
The one loop contribution for matter fields is given by a hyperbolic gamma function, sometimes called \emph{double-sine} function \cite{Hama:2011ea} which depends on the charges of the field under the global and gauge symmetries.
\begin{equation}
Z_{\Phi_i}^{1-loop} = \prod_{i}\Gamma_h ( \omega \Delta_i + \sum_{a} \mu_a e_a ; \omega_1 ,\omega_2)
\label{eqn:1-loop_localization_phi}
\end{equation}
where $\omega_1 = i b  $, $\omega_2 =  i b^{-1}$ and $\omega =\frac{\omega_1 + \omega_2}{2}$.
An overview on the properties of the hyperbolic gamma function can be found in appendix \ref{appendix:gamma_functions}.\\
For the vector multiplet the result is similar and since it is in the adjoint representation of the gauge group it is given by
\begin{equation}
Z_{V}^{1-loop} = \prod_{1 \leq i < j \leq dim(G)} \frac{1}{\Gamma_h( \pm (\sigma_i - \sigma_j) )}
\label{eqn:1-loop_localization_V}
\end{equation}
\\
As a result, the partition function can be calculated by an integral on the Cartan subalgebra of the gauge group of products of hyperbolic gamma functions and exponential factors that are associate to Fayet-Iliopulos and Chern-Simons terms. 


\section{Reduction of the superconformal index to the partition function}

The superconformal index calculated on $S_b^3 \times S^1$ reduces to the partition function of the dimensionally reduced theory on $S_b^3$ with $\eta$ superpotential in the limit $S^1 \rightarrow 0$  \cite{Gadde:2011ia} \cite{Dolan:2011rp}.\\
A heuristic argument to support this claim is that the superconformal index receives contributions only from BPS states in four dimensions which are in one-to-one correspondence with three dimensional states with non zero eigenvalue of the operators appearing in \eqref{eqn:localization_determinant_expression}.\\
Moreover, the BPS condition in four dimension once it is dimensionally reduced to three dimensions is equivalent to the saddle point conditions of the partition function \cite{Agarwal:2012hs}.\\
In order to obtain the partition function, it is necessary that all the fugacities appearing in the index approach flow to one.
They must be parametrized as function of the radius $r$ of $S^1$
\begin{equation}
p= e^{2 \pi i r \omega_1} \quad q=e^{2 \pi i r \omega_2} \quad u_a = e^{2 \pi i r \mu_a } \quad z_i = e^{2 \pi i r \sigma_i}
\label{eqn:fugacities_r_limit_real_masses}
\end{equation}
where $\mu_a$ are the real masses and $\sigma_i$ is the scalar in the vector multiplet.
\\
Since the superconformal index is written in terms of elliptic gamma functions, we can use the following identity \cite{vanDeBult:2007} to perform the limit $r \rightarrow 0$ 
\begin{equation}
\lim_{r \rightarrow 0^+} \Gamma_e (e^{ 2 \pi i r z}; e^{ 2 \pi i  r \omega_1} , e^{ 2 \pi i r  \omega_2}) \,  \sim \,
 e^{\frac{ - i \pi^2 }{6  r \omega_1 \omega_2 } ( z - \omega)} 
 \Gamma_h ( z ; \omega_1 , \omega_2 )
 \label{eqn:elliptic_to_hyperbolic_vdbult}
\end{equation}
Using this formula, we can reduce every elliptic gamma function in the index to a hyperbolic gamma function, which is the building block of the partition function in three dimension.
The parametrization of the fugacities in \eqref{eqn:fugacities_r_limit_real_masses} ensures that the arguments of the hyperbolic gamma functions obtained using this identity match the expressions present in the 1-loop corrections \cite{Dolan:2011rp} that we introduced in \eqref{eqn:1-loop_localization_phi} and \eqref{eqn:1-loop_localization_V}.\\
Applying the formula for every factor in the index we obtain the expression for the partition function in three dimension multiplied by a divergent prefactor, which is given by the product of the exponential factors in \eqref{eqn:elliptic_to_hyperbolic_vdbult} for every field.\\
It can be shown \cite{Aharony:2013dha} that this prefactor is proportional to the $U(1)_R$-gravity-gravity and flavour-gravity-gravity anomalies which coincide for dual theories.
As a result, this factor can be removed without problems since we are considering the reduction of the index  for dual theories. 









