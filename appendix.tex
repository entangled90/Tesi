%!TEX root = tesi.tex
\begin{appendices}

\chapter{Supersymmetry and superfields}
\label{appendice_susy}






\section{Supersymmetry algebra}
The supersymmetry algebra is an extension of the Poincarè group involving anticommutators together with commutators.
Since it is not a ordinary Lie algebra, Coleman-Mandula theorem does not apply for theories that are invariant under it.


The supersymmetry algebra is divided into two subalgebras, the bosonic and fermionic part.
The bosonic part contains Poincarè Lie algebra $(M_{\mu \nu}, P_{\mu})$ while fermionic subalgebra is generated by the \emph{supercharges} $(Q_{\alpha}^I, \bar{Q}_{\dot{\alpha}}^I)$ with $I = 1, \ldots , \mathcal{N} $. 
When more than one pair of supercharges is present we refer to extended supersymmetry.

The supercharges sit in spinorial representations of the Lorentz group, respectively $(\frac{1}{2},0) $ and $(0,\frac{1}{2}) $.

We will not repeat the bosonic subalgebra, since is given by the Poincarè Lie algebra.
The fermionic generators satisfy anticommutation rules between themselves and commutation rules with bosonic generators.
For this reason, the supersymmetry algebra is defined in mathematical literature as a graded Lie algebra with grade one. 
\\
The (anti)commutation rules in four dimensions are
\begin{align}
	[P_{\mu}, Q_{\alpha}^I ] & = \, 0 \\
	[P_{\mu}, \bar{Q}_{\dot{\alpha}}^I ] & = \, 0 \\
	[M_{\mu \nu}, Q_{\alpha}^I] & = i (\sigma_{\mu \nu})_{\alpha}^{\beta} Q_{\beta}^I \\
	[M_{\mu \nu}, \bar{Q}_{\dot{\alpha}}^I ] & = \, i (\bar{\sigma}_{\mu \nu})^{\dot{\alpha}}_{\dot{\beta}} \bar{Q}_{\dot{\dot{\beta}}}^I \\
	\{ Q_{\alpha}^I ,\bar{Q}_{\dot{\beta}}^J \}  & = \, 2 \sigma^\mu_{ \alpha \dot{\beta}} P_{\mu} \delta^{IJ} \\ 
	\{Q_{\alpha}^I, Q_{\beta}^J \} & = \, \epsilon_{\alpha \beta} Z^{I J } \qquad Z^{IJ} = - Z^{JI} \\
	\{Q_{\dot{\alpha}}^I, Q_{\dot{\beta}}^J \} & = \, \epsilon_{\dot{\alpha} \dot{\beta}} \left( Z^{I J } \right)^* 
\end{align}
This set of commutation rules can be found using symmetry arguments and enforcing the consistency of the algebra using the graded Jacobi identity.\\
It is important to stress the fact that $Z^{IJ}$ are operators that span an invariant subalgebra: they are \emph{central charges}.
They play an important role especially in massive representations.

There is an additional symmetry that is not present in the previous commutation rules: R-symmetry.
It is an automorphism of the algebra that act on the supercharges.
For generic $\mN$ the R-Symmetry group is  $U(\mN)$.









\section{Representations}
Since the supercharges do not commute with Lorentz generators, their action on a state will result in a state with different spin: they generate a symmetry between bosons and fermions.\\
Representations of supersymmetry contain particle with different spin but same mass and they are organized in supermultiplets.
The mass of particles in the same multiplet must be the same because $P^2$ is still a Casimir operator of the supersymmetry algebra, while the Pauli-Lubanski operator $W^2$ isn't anymore.  

Moreover, the supersymmetry algebra imposes that every state must have positive energy and that every supermultiplet must contain the same number of bosonic and fermionic degrees of freedom \emph{on-shell} .

Various supermultiplets exist and their properties depend on the number of supercharges of the theory and on what they represent e.g matter, glue or gravity. 

Massless supermultiplet are typically shorter than massive multiplet because in the massless case half of the supercharges are represented trivially.
Massive representation of extended supersymmetry can be shortened in case some of the central charges of the algebra are equal to twice the mass of the multiplet. These states are usually called (ultra)short multiplet or BPS states.


We will introduce the multiplets that can be defined for $4d \; \mathcal{N} = 1$ theories and only later we will explain the differences with $3d \; \mathcal{N} = 2$ theories.
Representations are similar because in both cases we have the same number of supercharges.

For four dimensional theories, we can define two different multiplet that are invariant under supersymmetry transformations.
The matter or chiral multiplet contains a complex scalar (\emph{squark}) and a Weyl fermion (\emph{quark}). It identifies the matter content of the theory.
The vector or gauge multiplet contains a Weyl fermion (\emph{gaugino}) and a vector (\emph{gluon or photon}).
Particles in the same multiplet transform in the same representation of global or gauge symmetries. For this reason the gaugino cannot represent matter.


A representation of these multiplets on fields can be easily found using the \emph{superspace} formalism that we will introduce in the next section.
In this formalism it is possible to represent fields  that are \emph{off-shell}, in contrast with multiplets that we introduced previously that are \emph{on-shell} since they represent states in Hilbert space.\\











\subsection{Superfields and superspace in four dimensions}

Supersymmetry representations on fields can be found more sistematically using the formulation of \emph{superspace} instead of acting directly with supercharges and verifying that the algebra closes.

A simple formulation of superspace exist for theories with 4 supercharges while for theories with a bigger number of supercharges its definition is much more complex. 
We will be interested only in theories with 4 supercharges such as theories in 4D with $\mN=1$ or 3D with $\mN=2$.

Superspace can be seen as the extension of Minkowsky space with \emph{fermionic coordinates} i.e. \emph{Grassman numbers} $\theta^{\alpha} \, , \, \bar{\theta}^{\dot{\alpha}}$.
They anticommute between themselves and commute with everything else.
\begin{equation}
 \{ \theta^{\alpha} , \theta^{\beta} \} = 0 \quad  \{ \bar{\theta}^{\dot{\alpha}} , \bar{\theta}^{\dot{\beta}} \} = 0 \quad \{   \theta^{\alpha}  , \bar{\theta}^{\dot{\beta}}\} = 0 \qquad \alpha,\dot{\alpha} = 1,2
\end{equation}
Derivation and integration in Grassmann variables are summarized by these rules 

%%%% Align messo a occhio!!
\begin{align}
 \partial_{\alpha} = \frac{\partial}{\partial \theta_{\alpha}} \quad 
 \partial^{\alpha} = -\epsilon^{\alpha \beta} \partial_{\beta} \quad
 \bar{\partial}_{\dot{\alpha}} =  \frac{\partial}{\partial \bar{\theta}^{\dot{\alpha}}} \quad 
 \bar{\partial}^{\dot{\alpha}} &= -\epsilon^{\dot{\alpha}  \dot{\beta}} \partial_{\dot{\beta}} 
\quad
 \partial_{\alpha} \theta^{\beta} = \delta^{\beta}_{\alpha} \quad
 \partial_{\alpha} \bar{\theta}^{\dot{\alpha}} = 0
 \\
 \int d \theta  = 0 \quad \int d \theta \, \theta = 1 \quad d^2\theta = & \frac{1}{2} d \theta^1 
  d \theta^2 \quad \int d^2 \theta = \frac{1}{4} \epsilon^{\alpha \beta } \partial_{\alpha} \partial{\beta}
\end{align}
For a more detailed introduction on Grassmann numbers and their properties see \cite{Bilal:2001nv}.

Using Grassmann numbers we can write the anticommutators in the supersymmetry algebra as commutators defining $\theta Q = \theta^{\alpha} Q_{\alpha}$ and $\bar{\theta} \bar{Q} = \bar{\theta}^{\dot{\alpha}} \bar{Q}_{\dot{\alpha}}$ 
\begin{equation}
[ \theta Q , \bar{\theta} \bar{Q}] = 2 \theta \sigma^{\mu} \bar{\theta} P_{\mu}  \quad , \quad [ \theta Q ,\theta Q ] = [ \bar{\theta} \bar{Q}, \bar{\theta} \bar{Q}] = 0
\end{equation}
Using this trick we are able to represent the supersymmetry algebra as a Lie algebra.
An element of the superPoincarè group can be found exponentiating the generators
\begin{equation}
G( x, \theta, \bar{\theta}, \omega) =  \exp \left( i x P + i \theta Q + i \bar{\theta} \bar{Q} + \frac{1}{2} i \omega M \right)
\end{equation}
The superspace is defined as the 4+4 dimensions group coset
\begin{equation}
  M_{4 | 1 } = \frac{\hbox{SuperPoincarè}}{\hbox{Lorentz}}
 \end{equation} 
in analogy to Minkowsky space that can be defined as the coset between Poincarè and Lorentz groups.

A generic point in superspace is parametrized by $(x^{\mu}, \theta^{\alpha}, \bar{\theta}^{\dot{\alpha}})$.
A superfields is a field in superspace i.e. function of the superspace coordinates.
Since $\theta$ coordinates anticommute, the expansion of a superfield in fermionic coordinates stops at some point.
The most general superfield $Y = Y(x, \theta, \bar{\theta})$ is given by
\begin{multline}
  Y(x, \theta, \bar{\theta}) =  f(x) +
   \theta \psi_1(x) + \bar{\theta} \bar{\psi_2} (x) + \theta \theta g_1(x) + \bar{\theta} \bar{\theta} g_2(x) \\ 
    + \theta \sigma^{\mu} \theta v_{\mu} (x) + \theta \theta\bar{\theta} \bar{\lambda}(x) + \bar{\theta} \bar{\theta} \theta \rho(x) + \theta \theta \bar{\theta} \bar{\theta} s(x) 
\end{multline} 
fields with uncontracted $\theta$ such as $\psi_1,\psi_2, \lambda, \rho$ are spinors while $v_{\mu}$ is a vector.

Supercharges can be represented as differential operators that act on superfield. Their expression is
\begin{align}
	\begin{cases}
		Q_{\alpha} & = - i \partial_{\alpha} - \sigma^{\mu}_{\alpha \dot{\beta}} \bar{\theta}^{\dot{\beta}} \partial_{\mu} \\
		\bar{Q}_{\dot{\alpha}} & = + i \dot{\partial}_{\dot{\alpha}} + \theta^{\beta} \sigma^{\mu}_{\beta \dot{\alpha} } \bar{\theta}^{\dot{\beta}} \partial_{\mu} \\
	\end{cases}
\end{align}
An infinitesimal supersymmetry transformation on a superfield is defined by
\begin{equation}
\delta_{\epsilon, \bar{\epsilon}} Y = \left( i \epsilon Q + i \bar{\epsilon} \bar{Q} \right) Y 
\end{equation}
The powerfulness of the superfield formalism is due to the fact that an integral in full superspace coordinates of a superfield is supersymmetric invariant.
\begin{equation}
\delta_{\epsilon, \bar{\epsilon}} \int d^4x \, d^2 \theta \, d^2 \bar{\theta} \; Y 
= \int d^4x \,  d^2 \theta \, d^2 \bar{\theta} \; \delta_{\epsilon, \bar{\epsilon}}  Y = 0  
\label{eqn:inv_superfields}
\end{equation}
The first equality holds because the Grassmann measure is inviariant under translation while the second is true because we can see that the variation of the superfield is either killed by the integration in the $\theta$ variables or is proportional to a spacetime derivative that does not contribute after integration in space. 

Using this fact we can construct supersymmetric invariant lagrangians by integrating superfields in superspace. 
Clearly, in order to find a physically significant lagrangian we should choose the superfield we wish to integrate wisely.
More importantly, we want to use irreducible representation of supersymmetry i.e. the supermultiplets we introduced before.
We need to find conditions that can be imposed on a general superfield that are invariant under a supersymmetry transformation.

\subsubsection{Chiral superfield}
One way to achieve this goal is to find an operator that commute with the supercharges and annihilate the superfield
An example of such operator is the \emph{covariant derivative} 
\begin{align}
	\begin{cases}
		D_{\alpha} =&  \partial_{\alpha} + i \sigma^{\mu}_{\alpha \dot{\beta}} \bar{\theta}^{\dot{\beta}} \partial_{\mu} \\
		\bar{D}_{\dot{\alpha}}  =&   \dot{\partial}_{\dot{\alpha}} + i  \theta^{\beta} \sigma^{\mu}_{\beta \dot{\alpha} } \bar{\theta}^{\dot{\beta}} \partial_{\mu} \\
	\end{cases}
\end{align}
We can define a (anti)chiral superfield $\Phi$ 
\begin{equation}
\bar{D}_{\dot{\alpha}}\Phi = 0 \quad \hbox{chiral} \qquad  \qquad D_{\alpha}  \Psi = 0 \quad \hbox{anti-chiral}
\end{equation}
This condition reduces the number of components of the superfield.
It can be easily demonstrated that if $\Psi$ is chiral, then $\bar{\Psi}$ is anti-chiral.
As a result a chiral field cannot be real. 

The expansion of a chiral fields in components is give by
\begin{equation}
\Phi ( x, \theta, \bar{\theta} ) = 
 \phi(x) + \sqrt{2}\theta \psi(x) + i \theta \sigma^{\mu} \bar{\theta} \partial{\mu} \phi(x) - \theta \theta F(x) - \frac{i}{\sqrt{2}} \theta \theta \partial{\mu} \psi(x) \sigma^{\mu} \bar{\theta} - \frac{1}{4} \theta \theta \bar{\theta} \bar{\theta} \box \phi(x)
 \end{equation}
We can see that a chiral field is composed by three fields: two complex scalars ($\phi$ and $F$) and a spinor ($\psi$). 

The chiral superfield identifies the matter multiplet we introduced previously. 
It contains an additional bosonic field ($F(x)$) that is present because superfields provide an \emph{off-shell} representation of supersymmetry and it is needed in order to close the algebra.
It is called \emph{auxiliary field} because it will not have kinetic terms in every Lagrangian that can be constructed. 

\subsubsection{Real or Vector Field}

We can impose that the superfield is real. In this way we find the 
\emph{real} or \emph{vector} multiplet.
Its general expression in component is messy and a simplification can be made noting that $\Phi + \bar{\Phi}$ is a vector superfield if $\Phi$ is chiral.
Choosing an appropriate chiral field, the real superfield can be put in what is called Wess-Zumino gauge. 
We stress the fact that the Wess-Zumino gauge is not supersymmetric invariant: after a supersymmetry transformation the vector superfield acquire its general expression involving many other field components.
In this gauge the vector superfield can be written as 
\begin{equation}
 V_{WZ} (x, \theta,\bar{\theta}) = \theta \sigma^{\mu} \bar{\theta} v_{\mu} (x) + i \theta\theta \bar{\theta} \bar{\lambda} (x) - i \bar{\theta} \bar{\theta} \theta \lambda(x) + \frac{1}{2} \theta \theta \bar{\theta} \bar{\theta} D(x)
\end{equation}
The vector superfields represents the vector multiplet (which contains radiation) and similarly to the chiral superfields contains an auxiliary field ($D(x)$).
  
\subsection{R-symmetry}
R-symmetry was first introduced with the supersymmetry algebra.
For the theories we will consider in superspace it is given by a global $U(1)_R$.
It is defined by as a transformation of the Grassmann coordinates
\begin{equation}
\theta \; \rightarrow \; e^{i \alpha} \theta \qquad \bar{\theta} \; \rightarrow \; e^{-i \alpha} \bar{\theta}
\end{equation}
$\alpha$ parametrized the transformation.
As a result supercharges transform under the transformation 
\begin{equation}
 Q \rightarrow \; e^{-i \alpha} Q  \qquad \bar{Q} \; \rightarrow \; e^{+i \alpha} \bar{Q}
\end{equation}
From this we find the commutator relations between supercharges and R-symmetry generator $R$
\begin{equation}
[R,Q] = - Q \qquad [R,\bar{Q}] = \bar{Q}
\label{eqn:r_charge_q_comm}
\end{equation}
The R-charge of a superfield is defined by
\begin{equation}
 Y (x, \theta ,\bar{\theta} ) \; \rightarrow \; e^{i R_Y \alpha} Y (x, \theta ,\bar{\theta} ) 
\end{equation}
Different component field in the superfield have different R-charge and are related because of \ref{eqn:r_charge_q_comm}.
For a chiral field we have
\begin{equation}
R[\phi] = R\left [{\Phi}\right] \qquad R[\psi] = R\left [{\Phi}\right] - 1 \qquad R[F] =  R\left [{\Phi}\right] - 2 
\end{equation}
The corresponding antichiral field carry opposite charges.


\section{Supersymmetric actions}
We will use the property we introduced in \ref{eqn:inv_superfields} to generate supersymmetric invariant lagrangians.
We start our analysis with chiral superfields.
Since lagrangians are quadratic in the fields and must be real, our best shot at finding the correct superfield to integrate is given by $ \bar{\Phi} \Phi$. 
\begin{equation}
 \mathcal{L}_{kin} \, = \, \int d^2 \theta d^2 \bar{\theta} \; \bar{\Phi} \Phi = \dem \bar{\phi} \deM \phi + \frac{i}{2} \left( \dem \psi \sigma^{\mu} \bar{\psi} - \psi \sigma^{\mu} \dem \bar{\psi} \right) + \bar{F} F + \hbox{total derivative}
\end{equation}		
We found the correct kinetic terms for the scalar and spinor field.
The auxiliary field doesn't have kinetic terms as predicted.
Many action can be find using a generalization of the equation above. It is called \emph{Kahler} potential   
\begin{equation}
K ( \bar{\Phi} , \Phi) = \sum_{m,n=1}^{\infty} c_{m,n} \bar{\Phi}^m \Phi^n \qquad \hbox{where} \quad c_{m,n} = c_{n,m}*
\end{equation}
The condition on the coefficient is imposed by the requirement of a real action. 

Actually we can produce supersymmetric actions also integrating \emph{chiral} superfields in half-superspace coordinates.
We define the \emph{superpotential} to be a holomorphic function of $\Phi$ 
\begin{equation}
 \mathcal{L}_{int} = \int  d^2 \theta d^2  \; W ( \Phi)  +  \int d^2 \bar{\theta} \; \bar{W} ( \bar{\Phi} ) = \sum_{i=1}^\infty  \int  d^2 \theta \; \lambda_n \Phi^n + \int d^2 \bar{\theta}\;  \lambda_n^{\dagger} \bar{\Phi}^n
\end{equation} 
In order to have a real lagrangian we added the hermitian conjugate.

Mixed terms with product of chiral and anti-chiral superfield are not present since they would be generic superfields and would not yield a supersymmetric lagrangian.
In fact if $W(\Phi)$ is holomorphic and $\Phi$ is a chiral superfield, $W(\Phi)$ is a chiral superfield
\begin{equation}
 \bar{D}_{\dot{\alpha}} W (\Phi) = \frac{\partial W}{ \partial \Phi } \bar{D}_{\dot{\alpha}} \Phi + \frac{\partial W}{ \partial \bar{\Phi} } \bar{D}_{\dot{\alpha}} \bar{\Phi}  = 0 
\end{equation}
and yield a proper lagrangian upon integration in $ d^2 \theta$.

Since the superpotential is integrated only in half superspace coordinates it need to be charged in an opposite way in order to provide a lagrangian invariant under R-symmetry.
Remembering  that
\begin{equation}
R[\theta] = 1 \qquad [\bar{\theta}] = - 1 \qquad R[d\theta] = -1 \qquad R [d \bar{\theta}] = 1 \qquad
\end{equation}
It's easy to see that 
\begin{equation}
R[W(\Phi)] = 2 \qquad R[\bar{W}(\bar{\Phi})] = -2
\end{equation}
For this reason in most situations the superpotential fix the supercharges of the fields.
\\
The lagrangian of super Yang-Mills theories is given by
\begin{align}
\mathcal{L}_{SYM} \, = \, & \frac{1}{32 \pi i  } \left( \int d^2 \theta \; \left( \frac{\theta_{YM}}{2 \pi}   + \frac{4 \pi i}{g^2} \right) W_{\alpha} W^{\alpha} \right) 
= 
\\
= \; & \hbox{Tr} \left[ -\frac{1}{4} F_{ \mu \nu} F^{\mu nu } - i \lambda \sigma^{\mu} D_{\mu} \bar{\lambda}  + \frac{1}{2} D^2 \right] + \frac{\theta_{YM}}{32 \pi^2 }g^2 \hbox{Tr} F_{ \mu \nu} \tilde{F}^{\mu nu } 
\end{align}
where we the chiral superfield as
\begin{equation}
 W_{\alpha} = - \frac{1}{4} \bar{D} \bar{D}\left(e^{-2 g V}  D_{\alpha}  e^{2g V} \right) \qquad  
\bar{W}_{\dot{\alpha}} = - \frac{1}{4} D D \left(   e^{2gV} \bar{D}_{\dot{\alpha}} V e^{-2gV} \right)  
\end{equation}
It can be demonstrated that $W_{\alpha}$ is chiral and is invariant under the supergauge transformation $V \; \rightarrow \; V + \bar{\Phi} + \Phi$ while the vector superfield $V$ was not.

From a perturbative point of view the inclusion of the term proportional to $\theta_{YM} {Tr} F_{ \mu \nu} \tilde{F}^{\mu nu }  $ has no effect since it is proportional to a total derivative. 
It is a parity violating term that differs from zero in non trivial topological configurations of the field (instantons).
\\
The matter lagrangian we introduced is not invariant under gauge transformation.
The correct invariant lagrangian is given by
\begin{equation}
 \mathcal{L}_{matter} \, = \, \int d^2 \theta d^2 \bar{\theta} \; \bar{\Phi} e^{ 2 g V} \Phi 
\end{equation}
The superpotential is not automatically invariant under gauge transformation. As a result only certain expression are allowed.
\\
There's an additional supersymmetric invariant lagrangian that can be constructed in a gauge theory when the gauge group contains abelian factors. 
There can be a term for every abelian factor.
It is called Fayet-Iliopulos term and is given by
\begin{equation}
 \mathcal{L}_{FI} \, = \, \sum_{A} \xi_A \int d^2 \theta d^2 \bar{\theta} V^{A} \, = \, \frac{1}{2} \sum_{A} \xi_A D^A
 \end{equation}






\end{appendices}