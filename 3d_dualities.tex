%!TEX root = tesi.tex


\chapter{Three dimensional dualities}
\section{Supersymmetry and field theories in three dimensions}
Spinors in three dimensions have different properties than their four dimension counterpart.\\
The dimension of the representation in an arbitrary dimension $D$  is given by $2^{\frac{D}{2}} $
for $D$ even, while $2^{\frac{D-1}{2}} $ for $D$ odd.
Hence, in three dimension we have a two dimensional representation.

In odd dimensions representations are irreducible and Weyl spinors do not exist: the chirality operator ($\gamma_{D+1}$ or $\gamma^*$ ) is proportional to the identity.

%This is related to the fact that representations in odd dimensions are constructed by taking the representation in one dimension less, which is even, and adding the chirality operator as the $D$-th matrix in the Clifford algebra.\\
Gamma matrices can be chosen real and the Majorana condition can be imposed, lowering the degress of freedom of the representation from four (two complex numbers) to two.\\
Since $3d \; \mN=2$ theories have four supercharges, we can use the $4d \; \mN=1$ superspace formalism. 
\\
The supersymmetry algebra and its representations can be found by dimensional reduction from four dimensions. 
The reduced algebra reads
\begin{equation}
 \{ Q_{\alpha},Q_{\beta} \} =  \{ \bar{Q}_{\alpha},\bar{Q}_{\beta} \}= 0 \qquad \{ Q_{\alpha} , \bar{Q}_{\beta}   \} = 2 \sigma^{\mu}_{\alpha \beta} P_{\mu} + 2 i \epsilon_{\alpha \beta} Z
 \end{equation} 
The central charge $Z$ is the component of the momentum  along the reduced direction.
Because of the presence of the central charge in the algebra, now states must satisfy a BPS bound of the form $M \geq Z $, which imply that for massless representations have null central charge.
\\
The automorphism of this algebra is $U(1)_R \simeq SO(2)_R$, as in four dimensions.
\\
Superspace formalism is similar to what we introduced previously, with proper changes due to the different spinor representation in three dimension such as gamma matrices.
\\
Chiral superfields contain \emph{on-shell} one complex scalar and a complex spinor as in four dimensions.\\
Vector superfield contains an additional real scalar field with respect to the four dimensional superfield.
The scalar $\sigma $ comes from the component of the vector field $A_{\mu}$ along the reduced direction.\\
\subsubsection{Coulomb branch and dualized photon	}
In three dimensions a free photon can be dualized to a scalar $\gamma$ since it has only one polarization. Instead of defining a potential $A$ in order to solve the Bianchi identity for the potential $d F = d^2 A = 0$, it can be defined a scalar such that
\begin{equation}
  * F = \frac{g^2}{\pi} d  \gamma  \quad \rightarrow \quad d * F  = 0  
\end{equation} 
where $*$ is the Hodge operator and $g$ is the gauge coupling.
The Bianchi identity can be seen as a conservation law for the topological current, which is not a symmetry in four dimensions
\begin{equation}
 J_{\mu}^{top} = \frac{1}{2 \pi} * F \; =  \; \frac{1}{2 \pi} d \gamma
\end{equation}
The topological current acts as shifts of the dualized photon $\gamma \rightarrow \gamma + \alpha$.\\
The quantization of the magnetic flux implies that the dualised foton is periodic and is normalized in such a way that $\gamma \sim \gamma + 2 \pi$.\\
The  vacuum expectation values of scalar fields in the vector multiplet parametrize a subset of the moduli space of the theory which is called the \emph{Coulomb branch}.
Using the dualized photon we can turn the vector multiplet into a chiral multiplet, whose lowest component is a good parametrization of the Coulomb branch 
\begin{equation}
 Y = \exp{ \frac{2 \pi \sigma}{g^2} + i \gamma}
\end{equation}
because of its periodicty, it's natural to assign $\gamma$ to a phase.\\
The dualization of the photon was possible because there wasn't matter in the theory.
However, matter fields couple with $\sigma$ with mass terms. As a result, in a generic point of the Coulomb branch, all charged matter fields are massive and in a low-energy description we can still dualize the gauge field.\\
We can generalize the dualization to non-Abelian gauge theories with a similar reasoning.
In a generic point of the Coulomb branch, the VEV of $\sigma$ breaks the gauge group to its maximal torus $U(1)^{r_G}$, where $r_G$ is the rank of the gauge group. 
We now have $r_G$ massless vector multiplets that can be dualized into chiral multiplets $Y_i$ with $i=1,\dotsc,r_G$.
\\
Note that since the definition of $Y$ depends on the gauge coupling, it will be modified by quantum corrections.

For $U(N)$ or $ SU(N)$ theories it's better to use the following coordinates on the Coulomb branch, since they are related to the simple roots of the algebra
\begin{equation}
 Y_k \sim \exp \left( \frac{\sigma_j- \sigma_{j+1}}{\hat{g}^2} + i ( \gamma_j - \gamma_{j+1} )   \right) \qquad \hat{g}^2 =  \frac{g^2}{4 \pi}
\end{equation}
with $j = 1, \dotsc , N -1$ for $SU(N)$ or with $ j = 1, \dotsc, N$ for $U(N)$.
From now on we will fix to a Weyl chamber by setting $\sigma_1 \geq \sigma_2 \geq \dots \geq \sigma_N$ with a Weyl transformation.

\subsubsection{Real masses}

In $3 D \, \mN=2$ theories there's another way of giving a mass to a chiral multiplet other than the superpotential term $W_{mass}= m \Phi^2$, which correspond to a holomorphic mass.\\
In these theories we can couple a global symmetry, which is not a R-symmetry, to a background vector multiplet.
We will give a vacuum expectation value $\hat{m} $ to the scalar in the multiplet.
As a result, every field charged under that symmetry will receive a mass $q \hat{m}$, where $q$ is the charge of the field under that symmetry.
If a chiral field is charged under different global symmetry, its mass  mass is a sum of the real masses for every global symmetry that we gauged.
Real masses are parity odd and belong to vector multiplets rather than chiral fields. For this reason they cannot appear in holomorphic objects such as the superpotential.





















\section{Aharony duality}
\section{Kutasov-Schwimmer duality}
