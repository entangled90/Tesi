%!TEX root = tesi.tex


\chapter{Three dimensional dualities}
\section{Supersymmetry and field theories in three dimensions}
Spinors in three dimensions have different properties than their four dimension counterpart.\\
The dimension of the representation in an arbitrary dimension $D$  is given by $2^{\frac{D}{2}} $
for $D$ even, while $2^{\frac{D-1}{2}} $ for $D$ odd.
Hence, in three dimension we have a two dimensional representation.

In odd dimensions representations are irreducible and Weyl spinors do not exist: the chirality operator ($\gamma_{D+1}$ or $\gamma^*$ ) is proportional to the identity.

%This is related to the fact that representations in odd dimensions are constructed by taking the representation in one dimension less, which is even, and adding the chirality operator as the $D$-th matrix in the Clifford algebra.\\
Gamma matrices can be chosen real and the Majorana condition can be imposed, lowering the degress of freedom of the representation from four (two complex numbers) to two.\\
Since $3d \; \mN=2$ theories have four supercharges, we can use the $4d \; \mN=1$ superspace formalism. 
\\
The supersymmetry algebra and its representations can be found by dimensional reduction from four dimensions. 
The reduced algebra reads
\begin{equation}
 \{ Q_{\alpha},Q_{\beta} \} =  \{ \bar{Q}_{\alpha},\bar{Q}_{\beta} \}= 0 \qquad \{ Q_{\alpha} , \bar{Q}_{\beta}   \} = 2 \sigma^{\mu}_{\alpha \beta} P_{\mu} + 2 i \epsilon_{\alpha \beta} Z
 \end{equation} 
The central charge $Z$ is the component of the momentum  along the reduced direction.
Because of the presence of the central charge in the algebra, now states must satisfy a BPS bound of the form $M \geq Z $, which imply that for massless representations have null central charge.
\\
The automorphism of this algebra is $U(1)_R \simeq SO(2)_R$, as in four dimensions.
\\
Superspace formalism is similar to what we introduced previously, with proper changes due to the different spinor representation in three dimension such as gamma matrices.
\\
Chiral superfields contain \emph{on-shell} one complex scalar and a complex spinor as in four dimensions.\\
Vector superfield contains an additional real scalar field with respect to the four dimensional superfield.
The scalar $\sigma $ comes from the component of the vector field $A_{\mu}$ along the reduced direction.\\
\subsubsection{Coulomb branch and dualized photon	}
In three dimensions a free photon can be dualized to a scalar $\gamma$ since it has only one polarization. Instead of defining a potential $A$ in order to solve the Bianchi identity for the potential $d F = d^2 A = 0$, it can be defined a scalar such that
\begin{equation}
  * F = \frac{g^2}{\pi} d  \gamma  \quad \rightarrow \quad d * F  = 0  
\end{equation} 
where $*$ is the Hodge operator and $g$ is the gauge coupling.
The Bianchi identity can be seen as a conservation law for the topological current, which is not a symmetry in four dimensions
\begin{equation}
 J_{\mu}^{top} = \frac{1}{2 \pi} * F \; =  \; \frac{1}{2 \pi} d \gamma
\end{equation}
The topological current acts as shifts of the dualized photon $\gamma \rightarrow \gamma + \alpha$.\\
The quantization of the magnetic flux implies that the dualised foton is periodic and is normalized in such a way that $\gamma \sim \gamma + 2 \pi$.\\
From a vector superfield $V$ we can define a linear multiplet $\Sigma$, defined as
\begin{equation}
 \Sigma = \epsilon^{\alpha \beta} D_{\alpha} D_{\beta} V
\end{equation}
It satisfies $D^2 \Sigma = 0$ and is gauge invariant under a transformation $V \rightarrow V + i (\Phi - \Phi^{\dagger})$.
The lowest component of $\Sigma$ is the scalar in the vector multiplet and it contains also a term $ \bar{\theta} \sigma_{\rho} \theta \, \epsilon^{\mu \nu \rho} F_{\nu \rho}$, whose bosonic part is proportional to the topological current.
The super Yang-Mills action can be written also as
\begin{equation}
S \sim \int d^4 \theta \; \Sigma^2
\end{equation}

The  vacuum expectation values of scalar fields in the vector multiplet parametrize a subset of the moduli space of the theory which is called the \emph{Coulomb branch}.
Using the dualized photon we can turn the vector multiplet into a chiral multiplet, whose lowest component is a good parametrization of the Coulomb branch 
\begin{equation}
 Y = \exp{ \left( \frac{2 \pi \sigma}{g^2} + i \gamma \right)}
\end{equation}
because of its periodicty, it's natural to assign $\gamma$ to a phase.\\
The dualization of the photon was possible because there wasn't matter in the theory.
However, matter fields couple with $\sigma$ with mass terms. As a result, in a generic point of the Coulomb branch, all charged matter fields are massive and in a low-energy description we can still dualize the gauge field.\\
We can generalize the dualization to non-Abelian gauge theories with a similar reasoning.
In a generic point of the Coulomb branch, the VEV of $\sigma$ breaks the gauge group to its maximal torus $U(1)^{r_G}$, where $r_G$ is the rank of the gauge group. 
We now have $r_G$ massless vector multiplets that can be dualized into chiral multiplets $Y_i$ with $i=1,\dotsc,r_G$.
\\
Note that since the definition of $Y$ depends on the gauge coupling, it will be modified by quantum corrections.

For $U(N)$ or $ SU(N)$ theories it's better to use the following coordinates on the Coulomb branch, since they are related to the simple roots of the algebra
\begin{equation}
 Y_k \sim \exp \left( \frac{\sigma_j- \sigma_{j+1}}{\hat{g}^2} + i ( \gamma_j - \gamma_{j+1} )   \right) \qquad \hat{g}^2 =  \frac{g^2}{4 \pi}
\label{eqn:Y_def_sun_theories}
\end{equation}
with $j = 1, \dotsc , N -1$ for $SU(N)$ or with $ j = 1, \dotsc, N$ for $U(N)$.
From now on we will fix to a Weyl chamber by setting $\sigma_1 \geq \sigma_2 \geq \dots \geq \sigma_N$ with a Weyl transformation.

\subsubsection{Real masses}

In $3 D \, \mN=2$ theories there's another way of giving a mass to a chiral multiplet other than the superpotential term $W_{mass}= m \Phi^2$, which correspond to a holomorphic mass.\\
In these theories we can couple a global symmetry, which is not a R-symmetry, to a background vector multiplet.
We will give a vacuum expectation value $\hat{m} $ to the scalar in the multiplet.
As a result, every field charged under that symmetry will receive a mass $q \hat{m}$, where $q$ is the charge of the field under that symmetry.
If a chiral field is charged under different global symmetry, its mass  mass is a sum of the real masses for every global symmetry that we gauged.
Real masses are parity odd and belong to vector multiplets rather than chiral fields. For this reason they cannot appear in holomorphic objects such as the superpotential.\\
In addition, real masses relative to global abelian symmetries contribute to the central charge of the supersymmetry algebra $Z$through
\begin{equation}
  Z = \sum q_i m_i
 \end{equation} 
 where $q_i$ is the charge of the field and $m_i$ the real masses under a global symmetry $U(1)_i$.\\
 Moreover, the central charge $Z$ can be promoted to a background linear superfield whose lowest component is $Z$.\\
 Another symmetry that can contribute to the central charge is $U(1)_J$, whose current is given by the linear multiplet we defined previously.
 The added term is given by $\int d^4 \theta \; V_b \Sigma $ , where $V_b$ is a background vector multiplet.
 Integrating by parts and defining  $\Sigma_b = \epsilon^{\alpha \beta} D_{\alpha} D_{\beta} V_b$ we obtain $\int d^4 \theta \; \Sigma_b V  $.
 Thus, the scalar component of the background vector field is a Fayet-Iliopulos term $\xi$, which contributes to the central charge $m_J = \xi $.
\subsection{Moduli spaces of gauge theories}

\subsubsection{Moduli space of $U(1)$ gauge theory with $N_f$ massless flavours }
For large values of $\sigma$ the Coulomb branch of the theory can be parametrized by the chiral superfield we defined before $X = e^{\frac{\Phi}{g^2}} $ which correspond to a cylinder.
However, the metric for $\gamma$ receives quantum correction and thus the topology of moduli space is changed perturbatively.
The Higgs branch intersects the Coulomb branch for $\sigma=0$ and since it is invariant under $U(1)_J$, they radius of the circle must shrink to zero where they meet, since the topological symmetry acts as shifts on the circle $\gamma$.
\\
Therefore, near the origin the moduli space looks like the intersection of three cones: the Higgs branch and two cones corresponding to two distinct parts of the Coulomb branch.
Half of the Coulomb branch is parametrized semi-classically by the field $X_+ \sim e^{\frac{\Phi}{g^2}}$, while the other half by $V_- \sim e^{ -\frac{\Phi}{g^2} }$. 
Two different chiral fields are needed since near the origin the moduli space shrinks to a point and $V_+ \rightarrow 0$. 
The Higgs branch is not modified by quantum corrections and the mesons can still be used to parametrize it.\\
The charges of the fields under the symmetries are given by
 \begin{equation}
 \begin{array}{ c | c c c c c }
  & U(1)_R &  U(1)_J & U(1)_A & SU(N_f)_R & SU(N_f)_R \\
 \hline
 Q & 0 & 0  & 1 & N_f & 1  \\  
 \tilde{Q} & 0 & 0  & 1 & 1 & \overline{N_f}  \\  
   M & 0 &  0 & 2 & N_f & \overline{N_f}\\  
   V_{\pm} & N_f & \pm 1  & -N_f & 1  &  1\\
 \end{array}
\end{equation}
The symmetries force the superpotential to have the form
\begin{equation}
W = - N_f \left(  V_+ V_- \Det{ M} \right)^{\frac{1}{N_f}}
\end{equation}
The superpotential is singular at the origin of the moduli space, indicating that there are massless degrees of freedom that need to be taken into account.\\
We can give real masses $\bar{m}_i$ ($-\bar{m}_i$) to the quarks $Q_i$ ($\tilde{Q}_i$).
The Higgs branch is parametrised by the diagonal elements of $M^i_{\tilde{i}}$ which intersects the Coulomb branch at $\sigma = \bar{m}_i$.
At every intersection, we have $U(1)$ theory with one flavour.
The coulomb branch is parametrised, at every intersection, by $X_{i,\pm} = e^{\pm \frac{\Phi - \bar{m}_i}{g^2}} $ with a superpotential given by
\begin{equation}
W = - \sum_{i = 1}^{N_f} M^i_i V_{i,+} V_{i,-} + \sum_{i=1}^{N_f - 1} \lambda_i \left(    V_{i,+} V_{i+1,-} - 1\right)
\end{equation}
where $\lambda_i$ are Lagrange multipliers in order the enforce the semiclassical identification  $ V_{i,+} V_{i+1,-} = 1 $.

{\huge Aggiungi foto sennò non si capisce niente}


\subsubsection{Moduli space of $SU(2)$ gauge theories with flavours}

We will consider a $SU(2) $ gauge theories with an even number of quarks, in order to avoid the addition of Chern Simons terms.
The charges are given by
\begin{equation}
 \begin{array}{ c | c c c  c }
  & U(1)_R  & U(1)_A & SU(2 N_f)_R \\
 \hline
 Q & 0  & 1 & 2 N_f \\  
   M & 0 & 2 & N_f (2 N_f-1)  & \\  
   V_{\pm} & 2 N_f  -2 &  - 2 N_f & 1 \\
 \end{array}
\end{equation}
and $Y$ is defined in \eqref{eqn:Y_def_sun_theories} and parametrizes the Coulom branch.
Mesons parametrises the Higgs branch and are subject to the condition $\mathrm{rank}(M) \leq 2 $.

A theory without quarks has a superpotential, generated by instantons, that lifts completely the Coulomb branch of the theory
\begin{equation}
W = \frac{1}{Y}
\end{equation}

With $N_f = 1$ the quantum moduli space consists in a smooth merging between the Higgs and the Coulomb branch of the theory since the fields are subject to the constraint
\begin{equation}
 M \, Y = 1
\end{equation}

For $N_f \geq 2 $, there is a unique superpotential consistent with the symmetries of the theory
\begin{equation}
 W = - (N_f -1 )( Y \, \mathrm{Pf} M)^{\frac{1}{N_f -1}}
\end{equation}
For $N_f = 2$ the Higgs and Coulomb branch are distinct and they touch at $M=0 \, , \, Y=0$ instead of the point with $\sigma = 0$.
The superpotential is relevant and drives the theory to a interacting fixed point.











\section{Aharony duality}
\section{Kutasov-Schwimmer duality}
