%!TEX root = tesi.tex
\documentclass[a4paper,titlepage,twoside,openright,12pt]{book}
\usepackage[british]{babel}
\usepackage{amsmath}
\usepackage{mathtools}
\usepackage{amstext}
\usepackage{amssymb}
\usepackage{amsthm}
%\usepackage{fullpage}
%%\usepackage{mathrsfs}
\usepackage[utf8]{inputenc}
%%\usepackage{natbib}
\usepackage{listings}
\usepackage{color}
\usepackage{verbatim}
\usepackage[T1]{fontenc}
\usepackage{blindtext, color}
\usepackage{lmodern}
\usepackage[toc,page]{appendix}
\usepackage[outercaption]{sidecap}
%\usepackage{floatrow}
\usepackage{graphicx}
\usepackage{color}
\usepackage{url}
%\usepackage{makeidx}
\usepackage{hyperref}
\usepackage[a4paper]{geometry}
\usepackage{microtype}
%\usepackage[nottoc,numbib]{tocbibind}
\usepackage[standard]{frontespizio}
%Per avere il titolo dei capitoli con un box
\usepackage[ Lenny ]{ fncychap }
%\usepackage{titlesec}
\usepackage{fancyhdr}
\usepackage{emptypage}

%\raggedbottom

%\usepackage[comma]{natbib}
\definecolor{gray75}{gray}{0.75}
\newcommand{\hsp}{\hspace{20pt}}
%\titleformat{\chapter}[hang]{\Huge\bfseries}{\thechapter\hsp\textcolor{gray75}{|}\hsp}{0pt}{\Huge\bfseries}


\author{
Carlo Sana }
%Università degli studi di Milano-Bicocca \\
%Dipartimento di Fisica G. Occhialini}
\date{}
\title{4D to 3D reduction of Seiberg duality for $SU(N)$ susy gauge theories with adjoint matter: a partition function approach}

%%% FANCYHDR STUFF
\renewcommand\headrulewidth{1pt}%\renewcommand\chaptermark[1]{\markboth{#1}{#1}}
%\renewcommand\sectionmark[1]{\markright{#1}{#1}}
\fancyhf{}
\fancyhead[CE]{\normalsize \nouppercase{\leftmark}}
\fancyhead[CO]{\normalsize \nouppercase{\rightmark}}
\fancyhead[LE,RO]{\scshape \thepage}
\pagestyle{fancy}

%\fancyhead[L]{\scshape{Chapter \thechapter - \chaptertitle} }
%\fancyhead[R]{ \slshape  \nouppercase{\rightmark}}
%\fancyfoot[C]{\scshape \thepage}
%\fancyfoot[C]{\scshape \thepage}

%%%%%%commands!!
%%% Comando per far le matrici con le barre verticali
\makeatletter
\renewcommand*\env@matrix[1][*\c@MaxMatrixCols c]{%
  \hskip -\arraycolsep
  \let\@ifnextchar\new@ifnextchar
  \array{#1}}
\makeatother

%%Comando per sottolineare testo con linea colorata in math mode
\newsavebox\MBox
\newcommand\Cline[2][red]{{\sbox\MBox{$#2$}%
  \rlap{\usebox\MBox}\color{#1}\rule[-1.2\dp\MBox]{\wd\MBox}{0.5pt}}}
  
%% comando per aligned nestati
\newlength{\myleftlen}
\newcommand{\setmyleftlen}[1]{\settowidth{\myleftlen}{\( \displaystyle
#1\)}}
\newcommand{\backup}{\hskip-\myleftlen\mkern-7mu}
  
  
  
%%comando per funzione segno
\newcommand{\sign}{\mbox{sign}}


%% Comando per N di supersymmetria

\newcommand{\mN}{\mathcal{N}}

%% Derivate parzielle
\newcommand{\dem}{\partial_{\mu}}
\newcommand{\deM}{\partial^{\mu}}
\newcommand{\den}{\partial_{\nu}}
\newcommand{\deN}{\partial^{\nsu}}

%Traccia e determinante
\newcommand{\Tr}{\mathrm{Tr } \,}
\newcommand{\Det}{\mathrm{det } \,}
\newcommand{\diag}{\hbox{diag\,}}

\newcommand{\til}[1]{\tilde{ #1}}


%ridefinisci overline più stretto
\newcommand{\overbar}[1]{\mkern 1.5mu\overline{\mkern-1.5mu#1\mkern-1.5mu}\mkern 1.5mu}



