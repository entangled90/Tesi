%!TEX root = tesi.tex

\section{General features of Seiberg-like dualities}

\begin{comment}
~ "Classically, and to all orders of perturbation theory, the electric and magnetic
   theories have different moduli spaces of vacua – it is only after taking nonperturbative
   effects into account that they are seen to be identical. For example, in
   the electric theory there is a classical constraint rankhMi ≤ Nc. In the dual theory,
   M is an independent field whose expectation value is unconstrained to all orders of
   perturbation theory – the constraint arises in the dual theory by quantum effects!"
   pag 27 di lectures

\end{comment}

