%!TEX root = tesi.tex

\section{General features of Seiberg-like dualities}

\begin{comment}
~ "Classically, and to all orders of perturbation theory, the electric and magnetic
   theories have different moduli spaces of vacua – it is only after taking nonperturbative
   effects into account that they are seen to be identical. For example, in
   the electric theory there is a classical constraint rankhMi ≤ Nc. In the dual theory,
   M is an independent field whose expectation value is unconstrained to all orders of
   perturbation theory – the constraint arises in the dual theory by quantum effects!"
   pag 27 di lectures

\end{comment}

\subsubsection{Moduli space of $4D \; \mN=1$ SQCD with $SU(N_c)$ gauge group and $N_f$ flavors}
As a concrete example of moduli space we will analyse $SU(N_c)$ SQCD with $N_f$ flavors.
Its classical lagrangian contains no superpotential and is given by two terms, generating \emph{D-terms} equations
\begin{equation}
\mathcal{L } =  \mathcal{L }_{SYM} + \mathcal{L }_{matter} 
\end{equation}
we refer to appendix \ref{appendice_susy} for their explicit form.

The global non anomalous symmetry group of the theory is $SU(N_f)_L \times SU(N_f)_R \times U(1)_B \times U(1)_R$. 
The matter content of the theory consists of $N_f$ $(Q_i, \tilde{Q}^{\tilde{j}})$ pairs of chiral fields, charged under \emph{left}  \emph{right} flavor symmetries respectively. 

