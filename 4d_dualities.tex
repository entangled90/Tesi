%!TEX root = tesi.tex

\chapter{Four dimensional dualities}

%!TEX root = tesi.tex
\begin{comment}
	---- OUTLINE-----
	***Supersimmetria:
		~ + simmetria = + strumenti per studiare teorie
		~Non renormalization theorems (perturbative)
		~ milder divergence 
		~exact results (e.g. superpotential, witten index , exact beta function )
		~Holomorphicity, couplings as background fields (important is smoothness of weak coupling limits, e.g. classic limit g in 0 well defined.
		Use of wilsonian action: no IR divergences)
		~moduli spaces
		--Superconformal group-- more relations, r charge, dimensions ecc
		~brane constructions?
		~Superconformal index
		~Localizations

\end{comment}
\section{Introduction}
Supersymmetric quantum field theories enjoy an enlarged group of  symmetries compared to other field theories. 
Since the symmetry group is a non trivial combination of internal and spacetime symmetries, they have many unexpected features and new techniques were found to study them.
On top of that \emph{superstring theory} provided many insights and explanations that were not clear from a field theory perspective only. 
Almost all of the new tools found are available only for supersymmetric field theories, making them the theater for exciting discoveries in physics.


\subsection{ Renormalization (and non-renormalization)}

A remarkable feature of supersymmetry is the constraints that imposes on the renormalization properties of the theories.

One of the first aspects that brought attention to supersymmetry was that divergences coming from loop diagrams were milder because of the cancellation between diagrams with bosons and fermions running in the loops. 
Even if this looks like a very promising feature, now it is a property that is exploited by even more powerful theorems about the renormalization properties of the theory.


\subsection{Renormalization}
Being a symmetry between bosons and fermions, supersymmetry imposes that states need be organized in multiplets containing different representations of the \emph{Lorentz group} such as scalars and spinors for example. Different multiplets exist and their properties depend on their explicit construction and on the number of supercharges of the theory.


In order to preserve supersymmetry, the renormalization process has to preserve the Hilbert Space structure i.e. for example the wave function renormalization of different \emph{ particles} inside a multiplet must be the same. 
With the same reasoning, the masses of different particles of the same multiplet must be the same (supersymmetry algebra requires this explicitly). 

\section{Seiberg duality}
\label{sec:seiberg_duality_4d}
Electric magnetic duality relates the dynamics of two different gauge theories in their infrared fixed point.
Even though the dual theories have different particle content, they describe the same IR physics. 
Moreover, whenever one of the dual theories gets more strongly coupled, the other becomes more weakly coupled.
This is particularly useful since it provides an alternative, weakly coupled description of the original theory.




\subsection{Electric theory: $SU(N)$ SQCD with $N_f$ flavours }
We will start our analysis on eletric-magnetic duality studying the first pair of theories that were discovered to be dual in \cite{Seiberg:1994pq}.  
We are going to analyse the properties of these theory in order to better understand the features of the duality.
\\
The electric theory is a $SU(N_c) $ supersymmetric gauge theory with $N_f$ massless flavours.
Its non-anomalous global symmetry group is 
\begin{equation}
SU(N_f)_L \times SU(N_f)_R \times U(1)_B \times U(1)_R 
\label{eqn:seib_dual_global_symm_group}
\end{equation}
The axial symmetry gives rise to an anomaly through the triangular graph $U(1)_A  \times SU(N_c)^2$, hence it is not a symmetry of the theory.\\ 
The classical lagrangian in superspace language is given by
\begin{equation}
 \mathcal{L} = \tau \int d^2 \theta \; Tr ( W_{\alpha} W^{\alpha} ) + \mathrm{h.c.} + 
 \int d^2 \theta \, d^2 \bar{\theta} \;  ({Q}^{i})^{\dagger} e^{ V} Q_i +
 \int d^2 \theta \, d^2 \bar{\theta} \; ({\tilde{Q}^{\tilde{i}})^{ \dagger}} e^{- V} \tilde{Q}_{\tilde{i}}
 \end{equation} 
$Q$ and $\tilde{Q}$ represent left and right quark superfield respectively.
The charges of the fields are given in the table below.
\begin{equation}
 \begin{array}{| c | c |  c c c c |}
 \hline
 & SU(N_c) & SU(N_f)_L  &SU(N_f)_R   & U(1)_B &  U(1)_R \\
\hline
Q & N_c & N_F & 1   &  1  &  \frac{N_f - N_c}{N_f}  \\
\tilde{Q} &\overbar{N_c}  &  1 & \overbar{ N_F}   & -1   &  \frac{N_f - N_c}{N_f}   \\	 
\hline
 \end{array} 
%	\centering
% \caption{Charges of the electric theory}
 \label{tab:suN_elect_charge}
\end{equation}
The value of the R-charge is fixed by the triangle anomaly $SU(N_c)^2 \, U(1)_R $, which can be calculated by considering diagrams with two exiting gluons and the R-symmetry current inserted in the other vertex.
\begin{figure}
\centering
\includegraphics[scale=0.6]{r-symm_anomlay.png}
\label{fig:r_symm_triangle_anom}
\caption{Feynman graphs contributing to the R-symmetry anomaly}
\end{figure}
Each fermion in the theory contributes to the anomaly which, as a result, is proportional to the R-charge of the fermions running in the loop and the Dynkin index of its representation.
Requiring the anomaly to vanish, we find
\begin{align}
R_{gaugino} T (\hbox{Ad}) + \sum_{ f} (& R_{f} - 1)  T(r)   = 0 
\label{eqn:r_symm_anom_condition}
\\
N_c + \frac{1}{2}\,  2 N_f (R_Q -1)  = 0 \quad & \rightarrow \quad R_Q = \frac{N_f - N_c}{N_f}
\end{align}
where we set the gaugino R-charge to $1$ in order to have gluons not charged under R-symmetry.
\\
Since the non-anomalous R-symmetry condition leads to a unique set of R-charges, we found the R-charges at the superconformal infrared point of the theory.


\subsubsection{Classical moduli space}
Since there is no superpotential, the classical moduli space of the theory is given by \emph{D-terms} only. 
They can be read from the on-shell lagrangian and are given by
\begin{equation}
 D^a = g \left( Q^{*i} T^a Q_i - \tilde{Q}^{* i} T^a \tilde{Q}_i \right) = 0
\end{equation}
where $T^a$ are the gauge group generators in fundamental or antifundamental representation and $i$ is a flavour index.
\\
After considering gauge and global symmetries, the squark \emph{VEVs}, represented as $N_f \times N_c$ matrices, which satisfy the D-term equation are given by
\begin{itemize}
\item for $N_f \le N_c$ and $a_i$ generic
\begin{equation}
Q = \tilde{Q} = 
\begin{pmatrix}
 a_1 & 		&	 &	 & \vdots & \vdots \\
	 & a_2  & 	 & 	 & \vdots & \vdots \\  	
 	 & 		&	\ddots &	 &\vdots & \vdots  \\
	 &  & 	 & 	a_{N_f}  & \vdots & \vdots
\end{pmatrix} 
 \\
\end{equation}
\item for $N_f \geq N_c$ and $ | a_i|^2 - | \tilde{a}_i |^2 = a $ independent of $i$.
\begin{equation}
Q  = 
\begin{pmatrix} 
	 a_1 & 		&	 &	  \\
	 & a_2  & 	 & 	 \\  	
 	 & 		&	\ddots &	   \\
	 &  & 	 & 	a_{N_c}  \\
	 \dots & \dots & \dots & \dots\\ 
	 \dots & \dots & \dots & \dots\\ 
\end{pmatrix} 
\quad
\tilde{Q} = 
\begin{pmatrix}
 \tilde{a}_1 & 		&	 &	  \\
	 & \tilde{a}_2  & 	 & 	 \\  	
 	 & 		&	\ddots &	   \\
	 &  & 	 & 	\tilde{a}_{N_c}  \\
	 \dots & \dots & \dots & \dots\\ 
	 \dots & \dots & \dots & \dots\\ 
\end{pmatrix} 
\end{equation}
\end{itemize}
For $N_f \le N_c$ in a generic point of the moduli space the gauge group is broken to $SU(N_c - N_f)$ while for $N_f \geq N_c$ is broken completely.
The gauge group breaks through the super Higgs mechanism: every broken generator is absorbed by the associated massless vector superfield.
This process gives mass to the vector superfield.  
\footnote{Remember that massive representation of supersymmetry have twice the degrees of freedom of massless ones, because in the latter half of the supercharges are represented trivially.}
The mass of the gauge superfield is given by the VEVs of the squarks.
\\
As we said in section \ref{sec:subsection_moduli_space}, we can  study the classical moduli space by finding holomorphic gauge invariant polynomials in the operators and modding out classical relations between them.
For $N_f < N_c$ we can only construct \emph{mesons} out of squarks
\begin{equation}
  M^i_j = Q^i \tilde{Q}_j
 \end{equation} 
where color indices are summed. 
Mesons have maximal rank since $N_f < N_c$ and there are no  classical constraints to impose on them. \\
When $N_f \geq N_c$ the mesons cannot have maximal rank anymore, it can be at most $N_c$.
There are additional gauge invariant operators that can be constructed: \emph{baryons}, that are defined as
\begin{align}
 B_{ i_1, \dotsc, i_{N_f - N_c}} = \epsilon_{i_1, \dotsc, i_{N_f - N_c}, j_1 ,\dotsc, j_{N_c}} \epsilon^{a_1 , \dotsc, a_{N_c}} Q^{j_1}_{a_1} \dots Q^{j_{N_c}}_{a_{N_c}}
 \\
 \tilde{B}^{ i_1, \dotsc, i_{N_f - N_c}} = \epsilon^{i_1, \dotsc, i_{N_f - N_c}, j_1 , \dotsc, j_{N_c}} \epsilon_{a_1 , \dotsc, a_{N_c}} \tilde{Q}_{j_1}^{a_1} \dots \tilde{Q}_{j_{N_c}}^{a_{N_c}}
\end{align}
Mesons and baryons can be written down using the \emph{VEVs} we found solving the \emph{D-term} equations (ignoring null components for baryons)
\begin{align}
M =& \begin{pmatrix}
a_1 \tilde{a_1} & & & & & \\
				& a_2 \tilde{a_2}	& 		&		 & & \\
				&				 	& \ddots&		& 	& \\
				&					&		& a_{N_c} \tilde{a_{N_c}} & \\
				& & & & & \\	
\end{pmatrix}
\\
B  \simeq & \;  a_1 a_2 \dots a_{N_c} 
\\
\tilde{B} \simeq & \;  \tilde{a_1} \tilde{a_2}\dots \tilde{a_{N_c}}
\end{align}
We can see that if the mesons have rank less than $N_c$, then $B$ or $\tilde{B}$ (or both) has to vanish and the other has rank one. If the mesons' rank is $N_c$ both $B$ and $\tilde{B}$ have rank one.
\\
There are classical constraints that should be imposed between mesons and baryons, but depend on the number of flavours .
For example for $N_f = N_c$ the classical constraint  is $ det \,( M) - B \tilde{B} = 0$.
\\
Singularities of the moduli space can be investigated using the gauge invariant description we have just introduced.
The part of the lagrangian that describes flat directions can be written in terms of mesons and baryons. 
The lagrangian involving mesons features a non-trivial Kahler potential that reads
\begin{equation}
  K = 2 \Tr \sqrt{ M^{\dagger} M}
 \end{equation} 
that generates a singular metric whenever the meson matrix is not invertible. 
This happens when some of the VEVs are zero, i.e. in points of the moduli space of enhanced gauge symmetry.
The appearance of this singularities is related to the fact that
some (or all) gluons are now massless and should be included in the low-energy description.




\subsubsection{Quantum moduli space}

Quantum dynamics modifies the structure of the moduli space of the theory in a different way depending on the number of colours and flavours.
\paragraph{Pure SYM}
 For pure \emph{Super Yang Mills}, i.e. no quarks, the theory exihibit a discrete set of $N_c$ vacua. 
Without quarks a non-anomalous R-symmetry cannot be found,
and the R-symmetry is broken down to the discreet symmetry $\mathbb{Z}_{2 N_c}$.
Using holomorphy and symmetry arguments, the form of the non-perturbative potential can be found and it can be shown that it induces the gaugino to condensate \cite{Veneziano:1982ah}, meaning that 
\begin{equation}
\langle \lambda \lambda \rangle = - \frac{32 \pi^2}{N_c} a \, \Lambda^3
\end{equation}
where $\Lambda$ is the dynamically generated scale of the theory defined as
\begin{equation}
\Lambda = \mu e^{-\frac{2 \pi i \tau}{b_0}} \qquad 
\tau  = \frac{4 \pi i}{g^2 (\mu) } + \frac{\theta_{YM}}{2 \pi}  \qquad b_0 = 3 N_c - N_f
\end{equation}
where $\tau$ is the complexified gauge coupling.
$|\Lambda|$ is defined as the scale at which the coupling constant blows up.\\
The gaugino condensation breaks R-symmetry to $\mathbb{Z}_2$ and in fact there are $N_c$ physically different vacua labelled by different phases of the gaugino condensate.
\paragraph{$\mathbf{ N_f < N_c}$}
The quantum corrections for \emph{SQCD} with $N_f < N_c$ flavours completely lift the moduli space through the Affleck-Dine-Seiberg (ADS) superpotential (\cite{Davis:1983mz}\cite{Affleck:1983mk}) which reads
\begin{equation}
W_{eff} = \left( N_c - N_f  \right)\left( 
\frac{\Lambda^{3N_c -  N_f }}
{\mathrm{det} {M}}
\right)^{ \frac{1}{N_c - N_f}}
\end{equation}
It is the only superpotential that is compatible with the symmetries of the theory and with the other properties of the superpotential we introduced in section \ref{sec:superpotential_hol_renorm}.
We can see that the ADS superpotential does not exist for $N_f \geq N_c$ since the exponent diverges for $N_f = N_c$ or the determinant vanishes for $N_f \ge N_c$ since the mesons do not have maximal rank.
Note that this superpotential is non-perturbative and thus it is not in contrast with the renormalization theorem of section \ref{sec:superpotential_hol_renorm}.
The effect of this superpotential is that the theory does not have ground state.
The slope of the potential goes to zero only for $\mathrm{det} \, {M}  \rightarrow \infty $.
\begin{center}
\includegraphics[scale=0.5]{ads_super.png}
\end{center}
This situation is the perfect example where, unlike the classical moduli space, quantum corrections lift completely the moduli space and the theory does not posseses a vacuum anymore.

\paragraph{$\mathbf{ N_f = N_c}$}
When the number of flavours is equal the number of colours of the theory, the classical moduli space was subject to the constraint
\begin{equation}
 \Det{M} - B \tilde{B} = 0
\end{equation}
In the quantum corrected moduli space mesons and baryons satisfy \cite{Seiberg:1994bz}
\begin{equation}
 \Det{M} - B \tilde{B} = \Lambda^{2 N_c}
\end{equation}
which flows to the classical constraint in the classical limit ($\Lambda \; \rightarrow \;0$).
\\
\begin{SCfigure}[1.3][h]
\centering
{\includegraphics[width=5 cm]{quantum_moduli_space_sqcd.png} }
{\caption{ Quantum and classical representation of the moduli space near the origin}}
\end{SCfigure}
The effect of this relation is that the origin  does not belong to the moduli space anymore and the moduli space is now smooth.
For large expectation values of $M$, $B$ and $\tilde{B}$ the classical and the quantum moduli space look similar, while in the origin of the moduli space quantum corrections modify drastically the structure of the moduli space.
Moreover, the subspace with $B$ or $\tilde{B}$ equal to zero, is not singular anymore while classically the meson matrix was constrained to have zero determinant. 
\\
Since the origin is not in the quantum moduli space the global symmetry group is necessarily broken in some way, depending on the position of the moduli space 
\begin{align}
M^i_j = \Lambda^2 \delta^i_j \quad B=\tilde{B}= 0 \quad & \rightarrow  \quad SU(N_f)_V \times U(1)_B \times U(1)_R\\
\quad M^i_j = 0 \quad B=-\tilde{B}= \Lambda^{N_c} \quad & \rightarrow  \quad SU(N_f)_L \times SU(N_f)_R \times U(1)_R
\end{align}
where $SU(N_f)_V$ is the diagonal vector subgroup.
\\
\paragraph{$\mathbf{N_f = N_c +1}$}
In the case $N_f = N_c +1 $ the classical moduli space is constrained by
\begin{equation}
 \Det{M} \left( \frac{1}{M}\right)^j_i - B_i \tilde{B}^j = 0 \qquad
 M^i_j B_i = M^i_j \tilde{B}^j = 0 
 \label{eqn:N_f__N_c_1_classical_constraints}
\end{equation}
and quantum corrections do not modify it.
In the previous section we noted that the singularities in the classical moduli space are associated to the appearance of massless gluons.
In the quantum picture, the interpretation of the singularities is different: they are associated with additional massless mesons and baryons.
At the origin of the moduli space the theory is strongly coupled and the global symmetry \eqref{eqn:seib_dual_global_symm_group} is unbroken and it can be checked using 't Hooft anomalies \cite{Seiberg:1994bz}.
Far from the origin, the mesons and baryons interact with the potential 
\begin{equation}
W = \frac{1}{\Lambda^{2 N_c -1}} \left( M^i_j B_i \tilde{B}^j - \Det{M} \right)
\end{equation}
that enforce the classical constraints \eqref{eqn:N_f__N_c_1_classical_constraints} through the equations of motion.
For large VEVs of the fields, mesons and baryons acquire large mass through the superpotential.
\paragraph{$\mathbf{ N_f > N_c +1 }$}
Starting from $N_f = N_c +2$ it is not possible to construct a physical superpotential out of gauge invariant operators, in analogy to the previous cases.
The only $SU(N_f)_L \times SU(N_f)_R$ invariant superpotential that can be written is given by
\begin{equation}
 W_{eff} \sim \Det{M} - B_{ij} M^i_k M^j_l \tilde{B}^{kl} 
 \end{equation} 
since baryons have two flavour indices.
However this superpotential does not have R-charge equal to two and if we add more flavours we should add other mesons to the superpotential. Therefore the classical moduli space is not modified by quantum corrections.
As a result, near the origin the quantum corrected moduli space looks identical to the classical one.
Unlike the case with $N_f = N_c +1$ the singularities in the moduli space cannot be interpreted as massless mesons and baryons and an effective description of these operator is singular \cite{Seiberg:1994bz}.
Since 't Hooft anomaly matching conditions are not satisfied in the singular points it is clear that a description using mesons and baryons is not correct.
\\
To find a description of the low-energy degrees of freedom of the theory we will use Seiberg duality, which provides an alternative description of the theory.

\paragraph{\textbf{Conformal window}}
%$\mathbf{  \frac{3}{2} N_c \geq N_f \geq  3 N_c}$ \textbf{: the conformal window}\\
Thetheory is not asympotically free in the range $\frac{3}{2} N_c < N_f < 3 N_c$. This can be seen by using the \emph{NSVZ} $\beta$ function\eqref{eqn:suN_Dynkin_indices}, which reads
\begin{align}
 \beta (g) & \, = \,- \frac{g^3}{16 \pi^2} \frac{3 N_c - N_f + N_f \gamma(g^2)}{1 - N_c \frac{g^2}{8 \pi^2}} \\
\gamma(g^2) &  \,= \, - \frac{g^2}{8 \pi^2} \frac{N_c^2 - 1}{N_c} + \mathcal{O} (g^4)
\end{align} 
The $\beta$ function is known to have a Banks-Zaks fixed point \cite{Banks:1981nn} in the 't Hooft limit with $\frac{N_f}{N_c} = 3 - \epsilon$ held fixed and $\epsilon \ll 1$.
However, the fixed point exists in the range of values $\frac{3}{2} N_c \geq N_f \geq  3 N_c$ with $N_f$ and $N_c$ finite.
This is possible because one loop and two loop contributions to the beta function have opposite signs.
As a result, the infrared theory is a non-trivial four dimensional superconformal theory. 
The infrared degrees of freedom are quarks and gluons that are not confining but are interacting as massless particles.
The theory is in a free non-Abelian Coulomb phase.
\\
Since the theory is superconformal, we have further restrictions on the algebra of operators \footnote{
	R-symmetry is part of the superconformal algebra instead of being an automorphism of the algebra, as in superPoincarè algebra.  
	}.
Superconformal algebra imposes that the dimension of every operator satisfy this inequality involving the R-charge
\begin{equation}
 D \geq \frac{3}{2} \; | R |
 \label{eqn:superconformal_dimension_rcharge}
\end{equation}
where the bound is satured for chiral fields.\\
The product of two chiral operator is constrained by this fact.
%Because $R( O_1 O_2) = R(O_1) + R(O_2)$, for chiral operators, which saturate the bound \eqref{eqn:superconformal_dimension_rcharge}, $D(O_1 O_2) = D(O_1) + D(O_2)$. 
Consider that for two operators $O_1,O_2$ we have $R( O_1 O_2) = R(O_1) + R(O_2)$.
Because of the bound \eqref{eqn:superconformal_dimension_rcharge}, the scaling dimension of chiral operators is additive $D(O_1 O_2) = D(O_1) + D(O_2) = \frac{3}{2} R(O_1) + \frac{3}{2} R(O_2)$.
Note that the dimension of the operator is quantum corrected, i.e. contains the anomalous dimension of the operator.
%Therefore, the OPE is not singular and the product of operators is well-defined. Because of this fact, chiral operators form the \emph{chiral ring}. 
\\
Since the superconformal R-symmetry is not anomalous and commutes with the global symmetry group of the theory it must be the R-symmetry that appears in table \ref{tab:suN_elect_charge}.
Because of \eqref{eqn:superconformal_dimension_rcharge}, the gauge invariant operators we defined previously must have the following dimensions
\begin{align}
 D(Q \tilde{Q}) &= \frac{3}{2} R(Q \tilde{Q}) = 3 \frac{N_f - N_c}{N_f}\\
 D(B) & = D(\tilde{B})  = \frac{3}{2} N_c \frac{N_f - N_c}{N_c}
\end{align}
\\
Gauge invariant operators should be in unitary representation of the superconformal algebra.
Unitarity imposes that in general $D\geq 1$ and the equality holds for singlet fields.
From the previous equation we can verify that $D(M) \geq 1$ for $ N_f \geq \frac{3}{2}N_c$ and it becomes a free field for $N_f = \frac{3}{2}N_c$. 
\\
For fewer number of flavours, the meson field is inconsistent with the unitarity bound.
The theory is conjectured to flow to a different phase.
\\
\paragraph{$\mathbf{ N_f > 3 N_c}$}
In this range, quarks prevails on gluons and change the sign of the $\beta$ function. 
This is caused by the \emph{charge screening} effect of quarks, that make the coupling constant smaller at larger distances.\\
The theory is in a free non-Abelian electric phase.
Its behaviour is not well defined at high energies because of the presence of a Landau pole at $R \sim \Lambda^{-1}$, although the theory can be a good description of a low-energy limit of another theory. 

\subsection{Magnetic theory}
The magnetic theory is a \emph{SQCD} theory with the same global symmetries as the electric theory, but with gauge group $SU(N_f - N_c)$. 
In addition there are $N_f^2$ color singlets, that we will call mesons, since they have the same properties as the mesons we can construct in the electric theory.
In the magnetic theory they are fundamental fields i.e. they are not written as gauge invariant operators from quarks. 
Since they are gauge invariant, they interact only through the superpotential
\begin{equation}
 W  = M^i_{\tilde{j}} \; q_i \, \tilde{q}^{\tilde{j}}
 \label{eqn:seiberg_duality_mag_super}
\end{equation}
where we represented dual quarks as $q,\tilde{q}$ and mesons as $M^i_{\tilde{j}} $.\\
The charges of fields of the magnetic theory are
\begin{equation}
 \begin{array}{| c | c |  c c c c |  }
 \hline
 & SU(N_c) & SU(N_f)_L  &SU(N_f)_R   & U(1)_B &  U(1)_R \\
\hline
q & N_c & \overbar{N_F} & 1   &   \frac{N_c}{N_f-N_c}   &  \frac{N_c}{N_f}  \\
\tilde{q} &\overbar{N_c}  &  1 & { N_F}   & - \frac{N_c}{N_f-N_c}   &   \frac{N_c}{N_f}   \\	 
M^i_{\tilde{j}}  &  1  & N_f & \overbar{N_f}  & 0 &  2 \frac{N_f - N_c}{N_f}\\ 
\hline
 \end{array}
 \label{tab:sun_magnetich_charges}
	%\centering
 %\caption{Charges of the magnetic theory}
\end{equation}\\
Dual quarks sit in opposite representation of flavour symmetries. 
\\
Mesons in the magnetic theory have the same charges of the mesons constructed from electric quarks.
Baryons constructed from dual quarks have the same baryonic charge as the electric baryons.
Moreover, it can be demonstrated that they are proportional to each other.\\
Similarly to the electric theory, the R-charges can be chosen in order to be non-anomalous.\\
However, the R-charges can be found using the duality.
If we impose that the meson is built from electric quarks, its R-charge is twice the R-charge of electric quarks.
Since the superpotential \eqref{eqn:seiberg_duality_mag_super} must have R-charge two, we find the R-charges of the magnetic quarks.
In this way, we found the R-charges at the superconformal infrared fixed point.

\subsubsection{Duality}
In the \emph{conformal window} the electric and magnetic theories we introduced previously give an equivalent description of the same physics in the infrared.
In this range, the magnetic theory has a non-trivial infrared fixed point too. 
At this fixed point, the superpotential \eqref{eqn:seiberg_duality_mag_super} is a relevant perturbation since it has dimension $D = 1 + 3 \frac{N_c}{N_f}  <3 $ and it drives the theory to a new fixed point.\\
Electric mesons have different dimension in the UV from the magnetic ones since they are constructed from a pair of quark.
For this reason it is necessary to introduce an energy scale $\mu$ in order to match their dimension in the UV: $ M = \mu M_m$, where $M_m$ are the magnetic mesons.\\
The strong coupling scales of the electric $\Lambda$ and magnetic $\tilde{\Lambda}$ theories are related by the relation
\begin{equation}
 \Lambda^{3 N_c - N_f} \tilde{\Lambda}^{3 (N_f - N_c) - N_f} = (-1)^{N_f - N_c} \mu^{N_f}
 \label{eqn:seib_dual_matching_scales}
\end{equation}
This relation shows that when one theory is strongly coupled, the other is weakly coupled.
Moreover, it ensures that the dual of the dual theory is the electric theory itself.
The dual of the dual magnetic theory is a $SU(N_c) $ \emph{SQCD} theory with scale $\Lambda$, $d^i$ and $\tilde{d}_{\tilde{j}}$ quarks and additional singlets $M^i_{\tilde{j}}$ and $N^{\tilde{j}}_i= q_i q^{\tilde{j}} $ with superpotential
\begin{equation}
 W = \frac{1}{\tilde{\mu}} N^{\tilde{j}}_i d^i \tilde{d}_{\tilde{j}} + \frac{1}{\mu} M^i_{\tilde{j}} N^{\tilde{j}}_i = \frac{1}{\mu} N^{\tilde{j}}_i \left(  - d^i d_{\tilde{j}}  + M^i_{\tilde{j}}  \right) 
\end{equation}
since from the previous relation we have $\tilde{\mu} = - \mu$.\\
Meson fields are massive and can be integrated out by using their equation of motion,which result in
\begin{equation}
 N^{\tilde{j}}_i  = 0 \qquad  M^i_{\tilde{j}} = d^i d_{\tilde{j}}
\end{equation}
Since the dual theories describe the same physics, there should be a mapping of gauge invariant operators between them.
We already saw that electric and magnetic mesons match in the infrared. 
A mapping should exists also for baryonic operators. Indeed it does and it is given by
\begin{equation}
\begin{aligned}
 B^{i_1 \dots i_{N_c}} & = C \; \epsilon^{i_1 \dots i_{N_c} j_1 \dots i_{N_f - N_c}} b_{j_1 \dots j_{N_f - N_C}}\\
 \tilde{B}^{i_1 \dots i_{N_c}} &= C \; \epsilon_{\tilde{i}_1 \dots \tilde{i}_{N_c} \tilde{j}_1 \dots \tilde{i}_{N_f - N_c}} b_{ \tilde{j}_1 \dots \tilde{j}_{N_f - N_C}}\\
 \text{where} \qquad C & = \sqrt{ - (-\mu)^{N_c - N_f} \Lambda^{3 N_c - N_f}}
\end{aligned}
\end{equation}
using \eqref{eqn:seib_dual_matching_scales} it can be shown that these mappings preserve the involutive action of the duality. 

As an additional check of the duality, 't Hooft anomaly matching conditions have been calculated  in \cite{Seiberg:1994pq} for the electric and magnetic theories for the various global symmetries and in both theories they are given by
\begin{equation}
\begin{aligned}
SU(N_f)^3 \quad \longrightarrow \quad   & N_c \\
U(1)_R\, SU(N_f)^2 \quad \longrightarrow \quad  & -\frac{N_c^2}{2 N_f} \\
U(1)_B\, SU(N_f)^2 \quad \longrightarrow \quad  & \frac{N_c}{2} \\
U(1)_R \quad \longrightarrow \quad  & -N_c^2 - 1 \\
U(1)_R^3 \quad \longrightarrow \quad  & -N_c^2 - 1 - 2 \frac{N_c^4}{N_f^2} \\
U(1)_B^2 U(1)_R \quad \longrightarrow \quad  & - 2 N_c^2 \\
\end{aligned}
\end{equation}
Another important check of the duality is that it remains valid under mass perturbations of the electric theory.
Suppose to add a superpotential term that gives mass to the quark in the last flavour and is equal to
\begin{equation}
	W_{mass}^{el} = m \; Q_{N_f} \tilde{Q}^{N_f}
\end{equation}
Flowing to the IR the number of quarks is decreased by one, driving the theory to a more strongly coupled fixed point\footnote{Since matter in the fundamental representation contributes with a positve term it is easy to see that this is true.}.
The new scale of the theory is given in terms of the old one by
\begin{equation}
 \Lambda_{L}^{3 N_c - (N_f - 1)} = m \; \Lambda^{3 N_c - N_f}
\end{equation}
In the magnetic theory the mass perturbation is mapped in the term
\begin{equation}
W_{mass}^{mag} = m \; M_{N_f}^{N_f}
\end{equation}
Because of this term, the gauge group gets higgsed to $SU(N_f-1 - N_c)$ and only $N_f -1$ light quarks remain in the theory. \\
The scale of the magnetic theory $\Lambda_L$ is modified and reads
\begin{equation}
\tilde{\Lambda}^{3(N_f - N_c -1) - (N_f -1)} = 
- \frac{
	\tilde{\Lambda}^{3 (N_f - N_c) - N_f}	
	}
	{
	< q_{N_f} \tilde{q}^{N_f}		>
	}
\end{equation}
where $< q_{N_f} \tilde{q}^{N_f}> = - \mu m $ because of the equation of motion for the massive flavour. \\
As expected the magnetic theory becomes more weakly coupled.\\
We conclude that the duality is preserved under massive deformations.  