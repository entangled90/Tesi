%!TEX root = tesi.tex

\begin{comment}
~ "Classically, and to all orders of perturbation theory, the 
\section{Electric-magnetic duality in three and four dimensions}


electric and magnetic
   theories have different moduli spaces of vacua – it is only after taking nonperturbative
   effects into account that they are seen to be identical. For example, in
   the electric theory there is a classical constraint rankhMi ≤ Nc. In the dual theory,
   M is an independent field whose expectation value is unconstrained to all orders of
   perturbation theory – the constraint arises in the dual theory by quantum effects!"

   pag 27 di lectures

\end{comment}



\begin{comment}
\subsubsection{Moduli space of $4D \; \mN=1$ SQCD with $SU(N_c)$ gauge group and $N_f$ flavourssis }
As a concrete example of moduli space we will analyse $SU(N_c)$ SQCD with $N_f$ flavourssis .
Its classical lagrangian contains no superpotential and is given by two terms, generating \emph{D-terms} equations
\begin{equation}
\mathcal{L } =  \mathcal{L }_{SYM} + \mathcal{L }_{matter} 
\end{equation}
we refer to appendix \ref{appendice_susy} for their explicit form.

The global non anomalous symmetry group of the theory is $SU(N_f)_L \times SU(N_f)_R \times U(1)_B \times U(1)_R$. 
The matter content of the theory consists of $N_f$ $(Q_i, \tilde{Q}^{\tilde{j}})$ pairs of chiral fields, charged under \emph{left}  \emph{right} flavours is symmetries respectively. 

\begin{lstlisting}
-------- Roba da dire nel capitolo ---------
 ~Phases of gauge theories?(asympt. free, non abelian coulomb/magnetic..)
 ~Region where duality is correct because of unitary?
 ~Use of superconformal algebra
 ~Anomaly free global symmetries
 ~Different gauge group: scale invariant theory
  ~Construct the dual from gauge invariant operators: they make sense and are consistent with other symmetries & anomalies
 ~Superpotential allowed by charges
 ~'t'Hooft anomalies
 ~Different group con be used
	\end{lstlisting}
\end{comment}


\chapter{Four dimensional dualities}

%!TEX root = tesi.tex
\begin{comment}
	---- OUTLINE-----
	***Supersimmetria:
		~ + simmetria = + strumenti per studiare teorie
		~Non renormalization theorems (perturbative)
		~ milder divergence 
		~exact results (e.g. superpotential, witten index , exact beta function )
		~Holomorphicity, couplings as background fields (important is smoothness of weak coupling limits, e.g. classic limit g in 0 well defined.
		Use of wilsonian action: no IR divergences)
		~moduli spaces
		--Superconformal group-- more relations, r charge, dimensions ecc
		~brane constructions?
		~Superconformal index
		~Localizations

\end{comment}
\section{Introduction}
Supersymmetric quantum field theories enjoy an enlarged group of  symmetries compared to other field theories. 
Since the symmetry group is a non trivial combination of internal and spacetime symmetries, they have many unexpected features and new techniques were found to study them.
On top of that \emph{superstring theory} provided many insights and explanations that were not clear from a field theory perspective only. 
Almost all of the new tools found are available only for supersymmetric field theories, making them the theater for exciting discoveries in physics.


\subsection{ Renormalization (and non-renormalization)}

A remarkable feature of supersymmetry is the constraints that imposes on the renormalization properties of the theories.

One of the first aspects that brought attention to supersymmetry was that divergences coming from loop diagrams were milder because of the cancellation between diagrams with bosons and fermions running in the loops. 
Even if this looks like a very promising feature, now it is a property that is exploited by even more powerful theorems about the renormalization properties of the theory.


\subsection{Renormalization}
Being a symmetry between bosons and fermions, supersymmetry imposes that states need be organized in multiplets containing different representations of the \emph{Lorentz group} such as scalars and spinors for example. Different multiplets exist and their properties depend on their explicit construction and on the number of supercharges of the theory.


In order to preserve supersymmetry, the renormalization process has to preserve the Hilbert Space structure i.e. for example the wave function renormalization of different \emph{ particles} inside a multiplet must be the same. 
With the same reasoning, the masses of different particles of the same multiplet must be the same (supersymmetry algebra requires this explicitly). 


\section{$SU(N)$ SQCD with $N_f$ flavours }

We will start our analysis on eletric-magnetic duality studying the first pair of theories that were discovered to be dual in \cite{Seiberg:1994pq}.  
We are gonna analyse the properties of these theory in order to better understand the features of the duality.



The electric theory is a $SU(N_c) $ supersymmetric gauge theory with $N_f$ flavours.
Its non anomalous global symmetry group is 
$$
SU(N_f)_L \times SU(N_f)_R \times U(1)_B \times U(1)_R 
\label{eqn:seib_dual_global_symm_group}
$$. 
The classical lagrangian doesn't have a superpotential and in terms of superfield is written as
\begin{equation}
 \mathcal{L} = \tau \int d^2 \theta \; Tr ( W_{\alpha} W^{\alpha} ) + 
 \int d^2 \theta \, d^2 \bar{\theta} \;  {Q}^{\dagger} e^{ V} Q +
 \int d^2 \theta \, d^2 \bar{\theta} \; {\tilde{Q}^{\dagger}} e^{- V} \tilde{Q}
 \end{equation} 
$Q$ and $\tilde{Q}$ represent left and right quark superfield respectively.\\
The charges of the fields are summarized in the table below.
\begin{table}[h!]
 \begin{tabular}{c | c |  c c c c }
 & $SU(N_c) $& $SU(N_f)_L$  &$SU(N_f)_R $  & $U(1)_B$ &  $U(1)_R$ \\
\hline
$Q$ & $N_c$ & $N_F$ & $1$   &  $1$  & $ \frac{N_f - N_c}{N_f}$  \\
$\tilde{Q}$ &$\overline{N_c} $ &  $1$ & $\overline{ N_F}$   & $-1$   &  $\frac{N_f - N_c}{N_f}$   \\	 
 \end{tabular}
	\centering
 \caption{Charges of the electric theory}
\end{table}

Mixing the R-symmetry with the baryon symmetry we can set the quark R-charges to be equal.
Since there's no superpotential their R-charges are not fixed relative to each other.
\\
The axial symmetry $U(1)_A$ is anomalous and cannot be present.
However it can mix with the \emph{natural} $U(1)_{R_0}$ R-symmetry of the theory, which is anomalous by itself, in order to give a non anomalous $U(1)_R$.\\
In fact, the value of the R-charge is fixed by the triangle anomaly $SU(N_c)^2 \, U(1)_R $, given by diagrams with two exiting gluons and R-symmetry current inserted in the cross 
\begin{center}
\includegraphics[scale=0.6]{r-symm_anomlay.png}
\end{center}
Every fermion in the theory contributes to the anomaly which, as a result, is proportional to the R-charge of the fermion running in the loop and the Dynkin index of its representation
\begin{align*}
R_{gaugino} T (\hbox{Ad}) + \sum_{ f} (& R_{f} - 1)  T(r)   = 0 \\
N_c + \frac{1}{2}\,  2 N_f (R_Q -1)  = 0 \quad & \rightarrow \quad R_Q = \frac{N_f - N_c}{N_f}
\end{align*}
where we set  the gaugino R-charge to $1$ in order to have gluons without charge.\\


\subsubsection{Classical moduli space}
Since there is no superpotential, the classical moduli space of the theory is given by \emph{D-terms} only. 
They can be read from the on-shell lagrangian and are given by
\begin{equation}
 D^a = g \left( Q^{*i} T^a Q_i - \tilde{Q}^{* i} T^a \tilde{Q}_i \right) = 0
\end{equation}
where $T^a$ are the gauge group generators in fundamental or antifundamental representation and $i$ is a flavour index.

After considering gauge and global symmetries, the squark \emph{VEVs}, represented as $N_f \times N_c$ matrices, that satisfy the D-term equation are for  $N_f \le N_c$ and $a_i$ generic
\begin{equation}
Q = \tilde{Q} = 
\begin{pmatrix}
 a_1 & 		&	 &	 & \vdots \\
	 & a_2  & 	 & 	 & \vdots \\  	
 	 & 		&	\ddots &	 &\vdots  \\
	 &  & 	 & 	a_{N_f}  & \vdots
\end{pmatrix} 
 \\
\end{equation}

for $N_f \geq N_c$ and $ | a_i|^2 - | \tilde{a}_i |^2 = a $ independent of $i$.

\begin{equation}
Q  = 
\begin{pmatrix} 
	 a_1 & 		&	 &	  \\
	 & a_2  & 	 & 	 \\  	
 	 & 		&	\ddots &	   \\
	 &  & 	 & 	a_{N_c}  \\
	 \dots & \dots & \dots & \dots\\ 
\end{pmatrix} 
\quad
\tilde{Q} = 
\begin{pmatrix}
 \tilde{a}_1 & 		&	 &	  \\
	 & \tilde{a}_2  & 	 & 	 \\  	
 	 & 		&	\ddots &	   \\
	 &  & 	 & 	\tilde{a}_{N_c}  \\
	 \dots & \dots & \dots & \dots\\ 
\end{pmatrix} 
\end{equation}


For $N_f \le N_c$ in a generic point of the moduli space the gauge group is broken to $SU(N_c - N_f)$ while for $N_f \geq N_c$ is broken completely.
The gauge group breaks through the super Higgs mechanism, where every broken generator gets absorbed by the (originally) massless vector superfield in order to make a massive vector superfield \footnote{Remember that massive representation of supersymmetry have twice the degrees of freedom of massless ones, because in the latter half of the supercharges are represented trivially.}
The mass of the gauge superfield is given by the VEVs of the squarks.

As we said in the previous section, we can  study the classical moduli space by finding holomorphic gauge invariant polynomial in the operators and modding out classical relations between them.
For $N_f \le N_c$ we can only construct \emph{mesons} out of squarks
\begin{equation}
  M^i_j = Q^i \tilde{Q}_j
 \end{equation} 
where color indices are summed. 
Mesons have maximal rank since $N_f \le N_c$ and there are no  classical to impose on them. 
They can be diagonalized in the same way we put the squarks \emph{VEVs} in diagonal form.\\ 
When $N_f \geq N_c$ the mesons cannot have maximal rank anymore, it can be at most $N_c$.
There are additional gauge invariant operators that can be constructed: \emph{baryons}, that are defined as
\begin{align}
 B_{ i_1, \dotsc, i_{N_f - N_c}} = \epsilon_{i_1, \dotsc, i_{N_f - N_c}, j_1 ,\dotsc, j_{N_c}} \epsilon^{a_1 , \dotsc, a_{N_c}} Q^{j_1}_{a_1} \dots Q^{j_{N_c}}_{a_{N_c}}
 \\
 \tilde{B}_{ i_1, \dotsc, i_{N_f - N_c}} = \epsilon^{i_1, \dotsc, i_{N_f - N_c}, j_1 , \dotsc, j_{N_c}} \epsilon_{a_1 , \dotsc, a_{N_c}} \tilde{Q}_{j_1}^{a_1} \dots \tilde{Q}_{j_{N_c}}^{a_{N_c}}
\end{align}
Mesons and baryons can be written down using the \emph{VEVs} we found solving the \emph{D-term} equations (ignoring null components for baryons)
\begin{align}
M =& \begin{pmatrix}
a_1 \tilde{a_1} & & & & & \\
				& a_2 \tilde{a_2}	& 		&		 & & \\
				&				 	& \ddots&		& 	& \\
				&					&		& a_{N_c} \tilde{a_{N_c}} & \\
				& & & & & \\	
\end{pmatrix}
\\
B  \simeq & \;  a_1 a_2 \dots a_{N_c} 
\\
\tilde{B} \simeq & \;  \tilde{a_1} \tilde{a_2}\dots \tilde{a_{N_c}}
\end{align}
We can see that if the mesons have rank less than $N_c$, then $B$ or $\tilde{B}$ (or both) has to vanish and the other has rank one. If the mesons' rank is $N_c$ both $B$ and $\tilde{B}$ have rank one.

That are classical constraints that should be imposed between mesons and baryons, but depend on the number of flavourssis .
For example for $N_f = N_c$ the classical constraint  is $ det \,( M) - B \tilde{B} = 0$.

Singularities of the moduli space can be investigated using the gauge invariant description we just introduced.
The part of the lagrangian that describes flat directions can be written in terms of mesons and baryons. 
The lagrangian involving mesons features a non trivial Kahler potential that reads
\begin{equation}
  K = 2 \Tr \sqrt{ M^{\dagger} M}
 \end{equation} 
that generate a singular metric whenever the meson matrix is not invertible. 
This happens when some of the VEVs are zero, i.e. in points of the moduli space of enhanced gauge symmetry.
The appearance of this singularities is related to the fact that
some (or all) gluons are now massless and should be included in the low-energy description.




\subsubsection{Quantum moduli space}

Quantum dynamics modifies the structure of the moduli space of the theory in a different way depending on the number of flavours.
\\
\\
{$\mathbf{N_f= 0}$}\\
 For pure \emph{Super Yang Mills}, i.e. no quarks, the theory exihibit a discrete set of $N_c$ vacua. 
Without quarks a non anomalous R-symmetry cannot be found,
and the R-symmetry is broken down to the discreet symmetry $\mathbb{Z}_{2 N_c}$.
Using holomorphy and symmetry arguments, the form of the non perturbative potential can be found and it can be shown that induces the gaugino to condensate, meaning that 
\begin{equation}
\langle \lambda \lambda \rangle = - \frac{32 \pi^2}{N_c} a \, \Lambda^3
\end{equation}
where $\Lambda$ is the dynamically generated scale of the theory defined as
\begin{equation}
\Lambda = \mu e^{-\frac{2 \pi i \tau}{b_0}} \qquad 
\tau  = \frac{4 \pi i}{g^2 (\mu) } + \frac{\theta_{YM}}{2 \pi}  \qquad b_0 = 3 N_c - N_f
\end{equation}
where $\tau$ is the complexified gauge coupling.
$|\Lambda|$ is defined as the scale at which the coupling constant blows up.\\
The gaugino condensation breaks R-symmetry to $\mathbb{Z}_2$ and in fact there are $N_c$ physically different vacua labelled by different phases of the gaugino condensate.
\\
\\
$\mathbf{ N_f < N_c}$\\
The quantum corrections for \emph{SQCD} with $N_f < N_c$ flavours completely lift the moduli space through the Affleck-Dine-Seiberg (ADS) superpotential (\cite{Davis:1983mz}\cite{Affleck:1983mk}) which reads
\begin{equation}
W_{eff} = \left( N_c - N_f  \right)\left( 
\frac{\Lambda^{3N_c -  N_f }}
{\mathrm{det} {M}}
\right)^{ \frac{1}{N_c - N_f}}
\end{equation}
It is the only superpotential that is compatible with the symmetries of the theory and with the other properties of the superpotential we introduced in section \ref{sec:superpotential_hol_renorm}.
We can see that the ADS superpotential do not exist for $N_F \geq N_c$ since the exponent diverges for $N_f = N_c$ or the determinant vanishes for $N_f \ge N_c$ since the mesons do not have maximal rank.
Note that this superpotential is non perturbative and thus it is not in contrast with the renormalization theorem of section \ref{sec:superpotential_hol_renorm}.
The superpotential is generated either by instantons for $N_f = N_C -1$ or by gaugino condensation for other number or flavours.
The effect of this superpotential is that the theory does not have ground state.
The slope of the potential goes to zero only for $\mathrm{det} \, {M}  \rightarrow \infty $.
\begin{center}
\includegraphics[scale=0.5]{ads_super.png}
\end{center}
This situation is the perfect example when, unlike the classical moduli space, quantum corrections lift completely the moduli space and the theory does not posses a vacuum anymore.
\\
\\
$\mathbf{ N_f = N_c}$\\
When the number of flavours is equal the number of colours of the theory, the classical moduli space was subject to the constraint
\begin{equation}
 \Det{M} - B \tilde{B} = 0
\end{equation}
in the quantum corrected moduli space mesons and baryons satisfy \cite{Seiberg:1994bz}
\begin{equation}
 \Det{M} - B \tilde{B} = \Lambda^{2 N_c}
\end{equation}
which flows to the classical constraint in the classical limit ($\Lambda \; \rightarrow \;0$).

\begin{SCfigure}[1.3][h]
\centering
{\includegraphics[width=5 cm]{quantum_moduli_space_sqcd.png} }
{\caption{Schematical representation of the quantum and classical moduli space near the origin}}
\end{SCfigure}

The effect of this relation is that the origin  does not belong to the moduli space anymore and the moduli space is now smooth.
For large expectation values of $M$, $B$ and $\tilde{B}$ the classical and the quantum moduli space look similar, while in the origin quantum correction modify drastically the structure of the moduli space.
Moreover, the subspace with $B$ or $\tilde{B}$ is zero, is not singular anymore while classical the meson matrix was constrained to have zero determinant. 

Since the origin is not in the quantum moduli space the global symmetry group \ref{eqn:seib_dual_global_symm_group} is necessarily broken in some way, depending on the position of the moduli space 
\begin{align}
M^i_j = \Lambda^2 \delta^i_j \quad B=\tilde{B}= 0 \quad & \rightarrow  \quad SU(N_f)_V \times U(1)_B \times U(1)_R\\
M^i_j = 0 \quad B=-\tilde{B}= \Lambda^{N_c} \quad & \rightarrow  \quad SU(N_f)_L \times SU(N_f)_R \times U(1)_R
\end{align}
$\mathbf{N_f = N_c +1}$
\\
In the case $N_f = N_c +1 $ the classical moduli space is constrained by
\begin{equation}
 \Det{M} \left( \frac{1}{M}\right)^j_i - B_i \tilde{B}^j = 0 \qquad
 M^i_j B_i = M^i_j \tilde{B}^j = 0 
 \label{eqn:N_f__N_c_1_classical_constraints}
\end{equation}
and quantum corrections do not modify it.
In the previous section we noted that the singularities in the classical moduli space are associated to the appearance of massless gluons.
In the quantum picture, the interpretation of the singularities is different: they are associated with additional massless mesons and baryons.
At the origin of the moduli space the theory is strongly coupled and the global symmetry \ref{eqn:seib_dual_global_symm_group} is unbroken and it can be checked with 't Hooft anomalies that mesons and baryons physical and contribute to the anomalies.\\
Far from the origin, mesons and baryons interact with an effective potential 
\begin{equation}
W_{eff} = \frac{1}{\Lambda^{2 N_c -1}} \left( M^i_j B_i \tilde{B}^j - \Det{M} \right)
\end{equation}
that enforce the classical constraints \ref{eqn:N_f__N_c_1_classical_constraints} through the equations of motion. Moreover, the superpotential give large mass to mesons and baryons.
\\
\\
$\mathbf{ N_f > N_c +1 }$\\
Starting from $N_f = N_c +2$ it is not possible to construct a sensible physical superpotential out of gauge invariant operators, in analogy to the previous cases.
The only $SU(N_f)_L \times SU(N_f)_R$ invariant superpotential that can be written is given by
\begin{equation}
 W_{eff} \sim \Det{M} - B_{ij} M^i_k M^j_l \tilde{B}^{kl} 
 \end{equation} 
since baryons have two flavour indices.
However this superpotential does not have R-charge equal to two and if we add more flavours we should add other mesons to the superpotential.\\
The classical constraints on mesons and baryons are satisfied quantum mechanically.
Unlike the case with $N_f = N_c +1$ the singularities in the moduli space cannot be interpreted as massless mesons and baryons and an effective description of these operator is singular \cite{Seiberg:1994bz}.
Since 't Hoof anomaly matching conditions are not satisfied in the singular points it is clear that a description using mesons and baryons is not correct.

To find a description of the low-energy degrees of freedom of the theory we will use Seiberg duality, which provides an alternative description of the theory,






\newpage
\begin{comment}
The magnetic theory is a theory with the same global symmetries as the electric theory, but the gauge group is now $SU(N_f - N_c)$ and in addition there are $N_f^2$ fields, that we will call mesons. 
Dual quarks will be represented as $q,\tilde{q}$ and mesons as $M^i_{\tilde{j}} $
The charges for the magnetic theory are given by


\begin{table}[h!]
 \begin{tabular}{c | c |  c c c c }
 & $SU(N_c) $& $SU(N_f)_L$  &$SU(N_f)_R $  & $U(1)_B$ &  $U(1)_R$ \\
\hline
$q$ & $N_c$ & $N_F$ & $1$   &  $ \frac{N_c}{N-f-N_c} $  & $ \frac{N_c}{N_f}$  \\
$\tilde{q}$ &$\overline{N_c} $ &  $1$ & $\overline{ N_F}$   & $- \frac{N_c}{N-f-N_c}$   &  $ \frac{N_c}{N_f}$   \\	 
$M^i_{\tilde{j}} $ & $1$ & $N_f$ & $\overline{N_f} $ & $0$ & $ 2 \frac{N_f - N_c}{N_f} $
 \end{tabular}
	\centering
 \caption{Charge of matter content of the magnetic theory}
\end{table}



\end{comment}









%%\subsection*{Kutasov-Schwimmer duality}


%\bibliography{bibliografia}