%!TEX root = tesi.tex

\section{Four dimensional dualities}
\begin{comment}
 Roba da dire nel capitolo:
 ~ Phases of gauge theories?(asympt. free, non abeliam coulomb/magnetic ecc)
 ~ Region where duality is correct because of unitary?
 ~Use of superconformal algebra
 ~Anomaly free global symmetryes
 ~Different gauge group : no problem, scale invariant theory
 ~ Construct the dual from gauge invariant operators: they make sense and are consistent with other symmetries & anomalies
 ~Superpotential allowed by charges
 ~'t'Hooft anomalies
 ~Different group con be used
 
\end{comment}





\subsection*{Seiberg Duality}
%%%%% DEVO aver già parlato di elettrico/magnetico come termini.
This is the duality originally found by Seiberg \cite{Seiberg:1994pq}.\\
\textbf{POSSO DIRE ALTRO!!}\\
The electric theory is a $SU(N_c) $ supersymmetric non-chiral gauge theory with $N_f$ flavours global symmetry group $SU(N_f) \times SU(N_f) \times U(1)_B \times U(1)_R $. The charges of the matter content of the theory are summarized in the table below.

\begin{table}[h!]
 \begin{tabular}{c | c |  c c c c }
 & $SU(N_c) $& $SU(N_f)_L$  &$SU(N_f)_R $  & $U(1)_B$ &  $U(1)_R$ \\
\hline
$Q$ & $N_c$ & $N_F$ & $1$   &  $1$  & $ \frac{N_f - N_c}{N_f}$  \\
$\tilde{Q}$ &$\overline{N_c} $ &  $1$ & $\overline{ N_F}$   & $-1$   &  $\frac{N_f - N_c}{N_f}$   \\	 
 \end{tabular}
	\centering
 \caption{Charge of matter content of the electric theory}
\end{table}
The \emph{R-Charge} is fixed by requiring that the \emph{R-Symmetry} is non anomalous. \\
\textbf{Da spiegare meglio.}\\
What really happens is that $U(1)_A$ symmetry (which is anomalous) mixes with the classical (anomalous) $U(1)_{R'}$ R-symmetry. Their mixing result in a non anomalous $U(1)_R$ R-symmetry and the disappearance of $U(1)_A$.\\ The triangular graph corresponding to this anomaly constrains the R-charge of the quarks imposing
\begin{align*}
R_{gaugino} T (\hbox{Ad}) + \sum_{fermions \; f} (& R_{f} - 1)  T(r)   = 0 \\
N_c + \frac{1}{2}\,  2 N_f (R_Q -1)  = 0 \quad & \rightarrow \quad R_Q = \frac{N_f - N_c}{N_f}
\end{align*}

The magnetic theory is a theory with the same global symmetries as the electric theory, but the gauge group is now $SU(N_f - N_c)$ and in addition there are $N_f^2$ fields, that we will call mesons. 
Dual quarks will be represented as $q,\tilde{q}$ and mesons as $M^i_{\tilde{j}} $
The charges for the magnetic theory are given by


\begin{table}[h!]
 \begin{tabular}{c | c |  c c c c }
 & $SU(N_c) $& $SU(N_f)_L$  &$SU(N_f)_R $  & $U(1)_B$ &  $U(1)_R$ \\
\hline
$q$ & $N_c$ & $N_F$ & $1$   &  $ \frac{N_c}{N-f-N_c} $  & $ \frac{N_c}{N_f}$  \\
$\tilde{q}$ &$\overline{N_c} $ &  $1$ & $\overline{ N_F}$   & $- \frac{N_c}{N-f-N_c}$   &  $ \frac{N_c}{N_f}$   \\	 
$M^i_{\tilde{j}} $ & $1$ & $N_f$ & $\overline{N_f} $ & $0$ & $ 2 \frac{N_f - N_c}{N_f} $
 \end{tabular}
	\centering
 \caption{Charge of matter content of the magnetic theory}
\end{table}










\subsection*{Kutasov-Schwimmer duality}



%\bibliography{bibliografia}