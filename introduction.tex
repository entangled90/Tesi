%!TEX root = tesi.tex

\chapter{Introduction}

%The most recent developments in quantum field theory have been associated to 
After the triumph of the Standard Model in the seventies, many theoretical physicist shifted their attention to the understanding of the dynamics of strongly coupled quantum field theories. \\
At the time, quantum field theories were properly understood only in perturbation theory. 
Perturbative methods rely on the fact that the theory in question is weakly coupled, i.e. its coupling constant is much lower than one. This was not a problem for QED since its coupling constant $\alpha \sim 1/137$ is much lower than one at low energies and grows slowly while increasing the renormalization scale.
As a result, QED calculations have a high precision and non-perturbative methods were not needed.\\
In 1973 was discovered that quantum chromodynamics with three flavours is asymptotically free, i.e. its coupling constant becomes smaller at higher energies. 
As a result, collider physics can be understood in terms of Feynman diagrams but at low-energies the theory is strongly coupled and perturbation theory cannot be used to understand the dynamics of the theory.
Computer simulations on a lattice are the most reliable tool to investigate QCD at low-energies, but they cannot provide a complete understanding of the behaviour of QCD.\\ 
In this context, dualities between theories at strong and weak coupling gained a lot of attention. 
By using strong-weak coupling dualities we are able to map observables from one theory to the other.
As a result, we can compute an observable in the weakly coupled theory and then map it to an observable (not necessarily the same) in the strongly coupled theory where we wouldn't have been able to compute it in perturbation theory. 




