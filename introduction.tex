%!TEX root = tesi.tex

\chapter{Introduction}

%The most recent developments in quantum field theory have been associated to 
After the triumph of the Standard Model in the seventies, many theoretical physicist shifted their attention to the understanding of the dynamics of strongly coupled quantum field theories. \\
At the time, quantum field theories were properly understood only in perturbation theory. 
Perturbative methods rely on the fact that the theory in question is weakly coupled, i.e. its coupling constant is much lower than one. This was not a problem for QED since its coupling constant $\alpha \sim 1/137$ is much lower than one at low energies and grows slowly while increasing the energy scale.
As a result, QED calculations have a high precision and non-perturbative methods were not needed.\\
In 1973 was discovered that quantum chromodynamics with three flavours is asymptotically free, i.e. its coupling constant becomes smaller at higher energies. 
As a result, collider physics can be understood in terms of Feynman diagrams but at low-energies the theory is strongly coupled and perturbation theory cannot be used to understand the dynamics of the theory.
Computer simulations on a lattice are the most reliable tool to investigate QCD at low-energies, but they cannot provide a complete understanding of the behaviour of QCD.\\ 
In this context, dualities between theories at strong and weak coupling gained a lot of attention. 
By using strong-weak coupling dualities we are able to map observables from one theory to the other.
As a result, we can compute an observable in the weakly coupled theory and then map it to an observable (not necessarily the same) in the strongly coupled theory where we wouldn't have been able to compute it in perturbation theory.  \\
The first example of strong-weak duality is the electric-magnetic duality of Maxwell's equations discovered by Dirac in 1931.  
He discovered that the equation of motion were invariant if electric and magnetic field were exchanged . 
In the presence of charges, the duality acts by exchanging magnetic and electric charges. 
%Thus, the existence of magnetic monopoles was necessary for the duality to exist.
By imposing that a quantum system has to be invariant under such duality he demonstrated that electric and magnetic charge are inversely proportional to each other. 
It can be demonstrated by requiring the monodromy of the wavefunction or by the quantization of the angular momentum of the electromagnetic field and results in the quantization condition $e g =  2 n \pi \hbar$, where $e$ is the electric charge and $g$ the magnetic charge.\\
A generalization of Dirac's electric-magnetic duality is the Montonen-Olive duality.
It states that a supersymmetric quantum field theory with $\mN=4$ in four dimensions with gauge group $G$ is equivalent to another theory with gauge group $\tilde{G}$ and with complexified gauge coupling $\tau' = - \frac{1}{\tau}$ and with electric and magnetic degrees of freedom exchanged.
The dual gauge group $G'$ is not necessarily the same as the group $G$.
The duality was conjectured in 1977 but a proof was only twenty years later using string theory.
%A generalization of this duality was found by Seiberg in 1994 






%%%% SEIBERG DUALITY
Another example of electric-magnetic duality is Seiberg duality and its generalizations.
It relates the behaviour in the infrared of two quantum field theories with different degrees of freedom in the infrared.\\
The electric theory in the original Seiberg duality is a $\mN=1$ gauge theory with $SU(N_c)$ gauge group in four dimensions and with $N_f$ quarks in the fundamental and antifundamental representation.
The magnetic theory is a $SU(N_f-N_c)$ gauge theory with $N_f$ fundamental and antifundamental quarks and with $N_f^2$ singlets fields. \\
They are called mesons because they have the same quantum numbers of the mesons that can be constructed in the electric theory by combining a quark with an antiquark. 
However, in the magnetic theory they are fundamental fields and they are not constructed from the quarks as in the electric theory.\\






