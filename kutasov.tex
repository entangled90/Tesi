%!TEX root = tesi.tex



\section{Kutasov-Schwimmer duality}


A possible generalization of Seiberg duality can be found by adding matter fields in different representations of the gauge group. \\
\subsection{Electric theory }
Kutasov and Schwimmer (\cite{Kutasov:1995ve}, \cite{Kutasov:1995np}) considered $SU(N)$ SQCD with the addition of a matter fields in the adjoint representation of the gauge group and found its magnetic dual.

The classical electric theory can be summarized by the following table of charges of the fields under the global symmetry group $SU(N_f)_L \times SU(N_f)_R \times U(1)_B \times U(1)_R
$

\begin{table}[h]
\begin{tabular}{c | c | c c c c }
 & $SU(N_c)$ &$SU(N_f)_L$  &$SU(N_f)_R $  & $U(1)_B$ &  $U(1)_R$ \\
\hline
$Q$ & $N_c$&$N_F$ & $1$   &  $1$  & $ 1 - \frac{2}{k+1} \frac{N_c}{N_f}$  \\
$\tilde{Q}$ & $\overline{N_c} $&  $1$ & $\overline{ N_F}$   & $-1$   &  $ 1 - \frac{2}{k+1} \frac{N_c}{N_f} $   \\
$X$ & \emph{Adjoint} & $1$   &$ 1$    &$ 0$   &  $\frac{2}{k+1}$ \\
\end{tabular}
\centering
\caption{Charges for the electric theory}
\label{table:charge_table_el_ks_4d}
\end{table}



and by the addition of the superpotential
\begin{equation}
	W_{Adj} = g_k \Tr{X^{k+1}}
	\label{eqn:kutasov_superpotential_X}
\end{equation}

The theory posses two different R-symmetries but the superpotential breaks explicitly one of them and imposes that the adjoint matter has R-charge $\frac{2}{k+1}$.
The scalar potential of the theory now includes the scalar field in the adjoint matter and therefore the moduli space is modified.
The superpotential add \emph{F-term} equations that need to be satisfied on the moduli space.

The remaining R-charges can be fixed by imposing that the R-symmetry can't be anomalous as we did previously.
Using formula \eqref{eqn:r_symm_anom_condition}, considering also the fermion in the adjoint matter multiplet we have
\begin{align}
N_c + (R_Q -1) \frac{1}{2} 2 N_f + (R_X - 1 ) N_c = 0 \\
(R_Q - 1 ) N_f = - R_X N_c \quad \longrightarrow \quad R_Q = 1 - R_X \frac{N_c}{N_f}
\label{eqn:kutas_r_charge_non_anomalous}
\end{align}

Imposing this condition the R-charges of the theory were fixed completely, as in Seiberg duality.
This has been possible because the superpotential \eqref{eqn:kutasov_superpotential_X} fixed independently the R-charge $\Delta_X$.
Otherwise the condition \eqref{eqn:kutas_r_charge_non_anomalous} fixes $R_Q$ as a function of $R_X$ with $R_X$ generic.
\\
It is interesting to note that the condition \eqref{eqn:kutas_r_charge_non_anomalous} can be found independently by requiring that the $\beta$ function has a fixed point
\begin{align}
0 =\beta{g} \sim & 3 T(Adj) - \sum_i T(Repr_i) (1- \gamma_i) = \\
 = & \,  T(Adj) + \sum_i T(Repr_i) ( R_i -1) = 0
\end{align}
where we have $\gamma_i + 2 = 2 D_i \;$ for chiral fields and at the superconformal fixed point  the dimension of an operator is related to its R-charge by $ \; R_i = \frac{3}{2} D_i$.
\\
The superpotential \eqref{eqn:kutasov_superpotential_X} drives the theory to a new IR point since it is either relevant or dangerously irrelevant, depending on the value of $k$.
\\
The gauge invariant operators that can be constructed are mesons and baryons multiplied with powers of the adjoint field.
Mesons operators are given by
\begin{equation}
 (M_j)^i_{\tilde{i}} = \tilde{Q}_{\tilde{i}} X^{j} Q^i \qquad j = 0,1,\dotsc,k-1
\end{equation}
Baryons are more easily introduced by first defining "dressed quarks"

\begin{equation}
 Q_{(l)} = X^{l} Q \qquad l =0, \dotsc, k-1
\end{equation}
Baryons are defined as
\begin{equation}
B^{i_1, i_2, \dotsc , i_k}) = Q_{(0)}^{i_1} \dots Q_{(k-1)}^{i_k} \qquad \hbox{with} \; \sum_{l=1}^k i_l = N_c 
\end{equation}
with color index contracted with an $\epsilon$ tensor.


\subsubsection{Vacuum structure}

The vacuum structure of this theory is complicated by the number of gauge invariant operators and the superpotential for the adjoint matter field.

In analogy to the condition $N_f \geq N_c$ in \emph{SQCD} we would like to find a range of values of $N_f,N_c$ and $k$ such that the theory admits stable vacua.  
We can add a weak deformation to the superpotential \eqref{eqn:kutasov_superpotential_X}, by adding terms with lower order powers in $X$
\begin{equation}
W(X) = \Tr \sum_{l=1}^k g_l X^{l+1} + \lambda \Tr X \qquad \hbox{with} \; g_l \ll 1
\end{equation}
where we introduced $\lambda$ to enforce the tracelessness of $X$.
Since it is a weak perturbation, the large field behavior of W is not modified.
Hence, if we can't find stable vacua with the weak perturbation, the original theory doesn't have any vacua too. 

The theory has a large sets of multiple vacua for $Q=\tilde{Q} = 0$ and $X \neq 0 $. 
$X$ can be diagonalized with eigenvalues $x_i$.
The F-terms are given by setting $W'(x_i) = 0$. 
Now, $W'(x_i)$ is a polynomial of degree $k$ in the eigenvalues $x_i$ admitting $k$ distinct solutions in general. 
As a result, ground states are labeled by a set of $k$ integers $(i_1, \dotsc, i_k)$, describing how many eigenvalues are sitting in the $l$-th minimum. Clearly, since $X$ has $N_c$ eigenvalues we have 
\begin{equation}
 \sum_{l=1}^k i_l = N_c
\end{equation}
In every vacuum, $X$ has a quadratic potential, which correspond to a mass term and can be integrated out.
The $X$ expectation values break the gauge group in the following way
\begin{equation}
	SU(N_c) \; \longrightarrow \; SU(i_1) \times SU(i_2) \times \dots \times SU(i_k) \times U(1)^{k-1}
\end{equation}
Each $SU({i_l})$ sector describes a decoupled \emph{SQCD} model which has stable vacua only if $N_f \geq N_c$, hence considering every sector we have 
\begin{equation}
i_l \leq N_f \qquad \forall 1 \leq l \leq k
\end{equation}
Taking the limit $g_l \rightarrow 0$ we find that we must have
\begin{equation}
 N_f \geq \frac{N_c}{k}
\end{equation}

For every choice of ${i_l}$ there is a moduli space obtained by giving expectation values of the quarks.
Hence, the moduli space of the theory consists of different disconnected component, associated to different ${i_l}$ choice.



\subsection{Magnetic theory}
The magnetic theory is constructed in a similar way as the magnetic theory in Seiberg duality.
The dual theory has gauge group $SU(k N_F - N_c)$. 
The baryonic charge of the dual quarks is found by imposing that baryons in the electric theory are proportional to the baryons constructed from dual quark.
\begin{table}[h!]
\begin{tabular}{c  c c c c }
 & $SU(N_f)_L$  &$SU(N_f)_R $  & $U(1)_B$ &  $U(1)_R$ \\
\hline
$q$ & $\overline{N_F}$  & $1$   &  $ \frac{N_c}{k N_f - N_c }$  & $ 1 - \frac{2}{k+1} \frac{ k N_f  - N_c}{N_f}$  \\
$\tilde{q}$  & $1 $ &  $N_F$  & $-\frac{N_c}{k N_f - N_c }$   &  $ 1 - \frac{2}{k+1} \frac{k N_f - N_c}{N_f}$   \\
$Y$ & $1$   &$ 1$    &$ 0$   &  $\frac{2}{k+1}$ \\
$ M_j$ & $N_f$ & $ \overline{N_f}$ & 0 & $2 - \frac{4}{k+1} \frac{N_c}{N_f} + j \frac{2}{k+1}$ \\
\end{tabular}
\centering
\caption{Charges for the magnetic theory}
\label{table:charge_table_mag_ks_4d}
\end{table}
The magnetic theory has a superpotential
\begin{equation}
W = \Tr Y^{k+1} + \sum_{j=0}^{k-1} M_j q Y^{k-j-1} \tilde{q} \qquad \hbox{dove} \; M_j = Q Y^j \tilde{Q}
\label{eqn:kutasov_mag_superpote}
\end{equation}
The charges of the fields are easily found by requiring duality for the two theories.
In this way, the charges of the mesons are given in terms of the electric quarks, which are fixed, and the superpotential fixes the charges for the remaining fields.
Using this method, the R-charges of the dual quarks given by
\begin{equation}
   R_q = R_X - R_Q 
 \label{eqn:R-charge_dual_quark}
\end{equation}
where we used that $R_X = R_Y = \frac{2}{k+1}$ because of \eqref{eqn:kutasov_mag_superpote} and \eqref{eqn:kutasov_superpotential_X}.



\subsubsection{Duality and mass deformations}
The 't Hooft anomaly matching conditions are satisfied and are given by
\begin{align}
SU(N_f)^3 \quad = \quad   & N_c d^{(3)}(N_f) \\
U(1)_R\, SU(N_f)^2 \quad = \quad  & -\frac{2}{k+1} \frac{N_c^2}{N_f} d^{(2)} (N_f) \\
SU(N_f)^2 U(1)_B\ \quad = \quad  & N_c d^{(2)}(N_f) \\
U(1)_R \quad = \quad  & - \frac{2}{k+1} (N_c^2 + 1) \\
U(1)_R^3 \quad = \quad  & \left( \left(\frac{2}{k+1} -1 \right)^3 +1 \right) (N_c^2-1) - \frac{16}{(k+1)^3} \frac{N_c^4}{N_f^2} \\
U(1)_B^2 U(1)_R \quad = \quad  & - \frac{4}{k+1} N_c^2
\end{align}

We consider now mass deformations of the electric theory in order to understand if duality is preserved under such deformations.
Let's modify the electric superpotential by adding a mass term
\begin{equation}
W_{el} =  g_k \Tr X^{k+1} + m \tilde{Q}_{N_f} Q^{N_f}
\end{equation}
The number of flavours in the IR is reduced by one unit keeping the number of colours the same. 
In order to preserve the duality, the magnetic theory should have gauge group $SU( k (N_f - 1) - N_c) = SU(k N_f - N_c - k)$.
Let's see if this happens.
The dual potential reads
\begin{equation}
W_{mag} = g_k \Tr Y^{k+1} + \sum_{j=1}^k M_j \tilde{q} Y^{k-j-1} q + m (M_0)^{N_f}_{N_f}
\end{equation}
Integrating out the massive fields we find 
\begin{equation}
 q_{N_f} Y^{l} \tilde{q}^{N_f} = - \delta_{l,k} m \quad l=0,\dotsc, k-1
\end{equation}
which fixes the expectation values to
\begin{align}
\tilde{q}_{\alpha}^{N_f} = & \delta_{\alpha,1} \\
q_{\alpha}^{N_f} = & \delta^{\alpha,k}\\
Y^{\alpha}_{\beta} = & 
	\begin{cases}
		\delta_{\beta+1}^{\alpha} \quad \beta = 1 , \dotsc, k-1 \\
		0 \qquad \hbox{otherwise}
	\end{cases}
\end{align}
These expectation values break the gauge group to $SU(k N_f - N_c) \rightarrow SU( k (N_f -1) - N_c) $ through the Higgs mechanism and reduces the number of flavours by one unit as required by duality.
As a result duality is preserved under mass deformations.
