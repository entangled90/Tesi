\documentclass[a4paper,12pt]{article}

\usepackage{amsmath}
\usepackage{mathtools}
\usepackage{amstext}
\usepackage{amssymb}
\usepackage{amsthm}
%\usepackage{fullpage}
%%\usepackage{mathrsfs}
\usepackage[italian]{babel}
\usepackage[utf8]{inputenc}
\usepackage[T1]{fontenc}
%%\usepackage{natbib}
\usepackage{lmodern}
\usepackage[toc,page]{appendix}
\usepackage[a4paper]{geometry}
%\usepackage[nottoc,numbib]{tocbibind}
\geometry{a4paper,top=1.5cm,bottom=1.5cm,left=2cm,right=2cm,%
heightrounded,bindingoffset=5mm}





%%%%%%%%%%% DA MODIFICARE!!!!!
%%%%%%%%%%%%%%%%%
%%%%%%%%%%%%%%


%%%%%%%%%%%%%%%%%%%OOOOOO
\author{
 Carlo Sana  ~~ \emph{matricola} 726409 \\
 Laurea Magistrale in Fisica Teorica -- 29 Giugno 2015\\
%  cell 3295795286 \\
\emph{Relatrice}: Silvia Penati 
~~
\emph{Correlatore}: Alberto Zaffaroni\\
}
\date{}
\title{ \textbf{4D to 3D reduction of Seiberg duality for $SU(N)$ susy gauge theories with adjoint matter: a partition function approach }}	
 


\pagenumbering{gobble}

\begin{document}
\maketitle
In ogni ambito della fisica i sistemi fortemente accoppiati sono difficilmente trattabili siccome raramente esistono approssimazioni che possono essere applicate ad essi.\\
In fisica delle alte energie negli ultimi vent'anni sono state scoperte alcune dualità che mettono in relazione teorie fortemente interagenti con teorie debolmente accoppiate, studiabili con un approccio perturbativo.
Le dualità attualmente più studiate in teoria dei campi sono la dualità di Seiberg e l'AdS/CFT.
%più generalmente chiamata \emph{gauge/gravity duality}. 
%In entrambi i casi, possiamo calcolare le osservabili della teoria fortemente accoppiata passando alla teoria duale, che è perturbativa. 

La dualità di Seiberg è anche chiamata dualità elettrica-magnetica in quanto condivide alcune caratteristiche con la dualità fra campi elettrici e magnetici delle equazioni di Maxwell trovata da Dirac nel 1931. 
In generale, la dualità di Seiberg associa ad una \emph{teoria elettrica}, che è una teoria di gauge non abeliana con supersimmetria, una teoria duale, quella \emph{magnetica}.
Quest'ultima è una teoria di gauge con un diverso numero di colori rispetto alla teoria elettrica e contiene i mesoni della teoria elettrica come particelle fondamentali invece che composte da due quark.
Per un particolare intervallo del numero di colori e di sapori, le due teorie fluiscono a basse energie in un punto fisso superconforme e descrivono lo stesso sistema fisico, nonostante ad alte energie mostrino una dinamica differente.
Sono note vari esempi di questa dualità che differiscono fra di loro per il gruppo di gauge o per il contenuto di materia della teoria.

La dualità di Seiberg fu scoperta per teorie di campo in 4D nel 1994 e pochi anni dopo fu scoperta una dualità analoga in 3D.
Tuttavia, non era chiaro se ci fosse una relazione tra le due.
Solo recentemente è stato scoperto un modo di effettuare una riduzione dimensionale delle dualità da 4D a 3D riuscendo a preservare la dualità fra le due teorie. \\
La riduzione delle dualità, oltre che in teoria di campo, può essere effettuata direttamente attraverso la funzione di partizione che per queste teorie può essere calcolata esattamente a partire dall'indice superconforme.
Esso è definito per teorie in 4D con una delle dimensioni compattificata su un cerchio e risulta essere uguale per teorie duali.
Nel limite in cui il raggio del cerchio tende a zero, otteniamo la funzione di partizione della teoria ridotta a tre dimensioni, la quale è soggetta ad alcuni vincoli dovuti al processo di riduzione.
\'E possibile eliminare questi vincoli manipolando opportunamente la funzione di partizione, ottenendo così una teoria puramente tridimensionale. 
Applicando questa procedura ad entrambe le teorie otteniamo le due funzioni di partizione, la cui uguaglianza è garantita dall'identità degli indici delle teorie in 4D.
% che deriva a sua volta dalla dualità in 4D.
Inoltre, le cariche dei campi della teoria possono essere identificate facilmente dalla funzione di partizione, siccome le simmetrie della teoria rimangono esplicite.

Lo scopo di questa tesi consiste nell'applicazione di questa procedura alla dualità Kutasov-Schwimmer-Seiberg, la quale si differenzia da quella di Seiberg per la presenza di un ulteriore campo di materia nella rappresentazione aggiunta del gruppo di gauge. \\
La riduzione dimensionale di tale teoria attraverso la funzione di partizione non era presente in letteratura e il nostro lavoro rappresenta una verifica indipendente dei risultati ottenuti in teoria di campo. 


\end{document}
