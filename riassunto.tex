\documentclass[a4paper,oneside,11pt]{article}

\usepackage{amsmath}
\usepackage{mathtools}
\usepackage{amstext}
\usepackage{amssymb}
\usepackage{amsthm}
%\usepackage{fullpage}
%%\usepackage{mathrsfs}
\usepackage[italian]{babel}
\usepackage[utf8]{inputenc}
\usepackage[T1]{fontenc}
%%\usepackage{natbib}
\usepackage{lmodern}
\usepackage[toc,page]{appendix}
\usepackage[a4paper]{geometry}
\usepackage[protrusion=true,expansion]{microtype}

%\usepackage[nottoc,numbib]{tocbibind}
\geometry{a4paper,top=2cm,bottom=2.5cm,left=2cm,right=2cm,%
heightrounded}
%\newenvironment{ppl}{\fontfamily{ppl}\selectfont}{\par}
%\usepackage{lmr}


\date{}

%%%%%%%%%%% DA MODIFICARE!!!!!
%%%%%%%%%%%%%%%%%
%%%%%%%%%%%%%%


%%%%%%%%%%%%%%%%%%%OOOOOO
\title{\boldmath \textbf{4D to 3D reduction of Seiberg duality for $SU(N)$ susy gauge theories with adjoint matter: \\  a partition function approach }
 }	
\author{}

\pagenumbering{gobble}
\begin{document}
\maketitle
\vspace*{-2.3cm}

\begin{center}
\rule{\textwidth}{0.5pt}
\scshape {29 Giugno 2015} \\
 \end{center}
\vspace{-0.3cm}
	 \textbf{\noindent%
	 \scshape%
	 Carlo Sana}  - \textsf{ 726409}  \hfill
 {\scshape LM in Fisica Teorica} 
 %-29 Giugno 2015
\\
 \noindent tel. \textsf{ - 329 5795286}\\
 \vspace{-0.5cm}
%  cell 3295795286 \\
\begin{center}
{ \scshape  Relatore}: 
\textsf{Silvia Penati} 
~~
{ \scshape  Correlatore}:
\textsf{Alberto Zaffaroni}

\end{center}

I sistemi fisici fortemente accoppiati sono in generale difficilmente trattabili da un punto di vista teorico perchè raramente esistono approssimazioni che possono essere applicate in questo regime.\\
In fisica delle alte energie negli ultimi vent'anni sono state scoperte alcune dualità che mettono in relazione teorie fortemente interagenti con teorie debolmente accoppiate.
Questo apre la strada alla possibiltà di descrivere sistemi in regime di accoppiamento forte mediante le ben note tecniche perturbative.
Le dualità attualmente più studiate in teoria dei campi sono la dualità di Seiberg e l'AdS/CFT (Anti-deSitter/Conformal Field Theory).

La dualità di Seiberg, anche chiamata dualità elettrica-magnetica, estende a teorie di gauge non abeliane alcune caratteristiche della dualità fra campi elettrici e magnetici delle equazioni di Maxwell trovata da Dirac nel 1931. 
In generale, la dualità di Seiberg associa ad una \emph{teoria elettrica}, che è una teoria di gauge non abeliana con supersimmetria, una teoria duale, quella \emph{magnetica}.
Quest'ultima è una teoria di gauge con un diverso numero di colori rispetto alla teoria elettrica e contiene i mesoni della teoria elettrica come particelle fondamentali invece che composte da un quark e un antiquark.
Per un particolare intervallo del numero di colori e di sapori, le due teorie fluiscono a basse energie in un punto fisso superconforme e descrivono lo stesso sistema fisico, nonostante ad alte energie mostrino una dinamica differente.
Sono noti vari esempi di questa dualità che differiscono fra di loro per il gruppo di gauge o per il contenuto di materia della teoria.

La dualità di Seiberg fu scoperta per teorie di campo in 4D nel 1994 e pochi anni dopo fu scoperta una dualità analoga in 3D.
Tuttavia, non era chiaro se ci fosse una relazione tra le due.
Solo recentemente è stato scoperto un modo per ottenere dualità in 3D a partire dalla riduzione dimensionale di quelle in 4D.
%\\
La riduzione delle dualità, oltre che in teoria di campo, può essere effettuata direttamente attraverso la funzione di partizione che per queste teorie può essere calcolata esattamente a partire dall'indice superconforme.
Esso è definito per teorie in 4D con una delle dimensioni compattificata su un cerchio e risulta essere uguale per teorie duali.
Nel limite in cui il raggio del cerchio tende a zero, otteniamo la funzione di partizione della teoria ridotta a tre dimensioni, la quale è soggetta ad alcuni vincoli dovuti al processo di riduzione.
\'E possibile eliminare questi vincoli manipolando opportunamente la funzione di partizione, ottenendo così una teoria puramente tridimensionale. 
Applicando questa procedura ad entrambe le teorie otteniamo le due funzioni di partizione, la cui uguaglianza è garantita dall'identità degli indici delle teorie in 4D.
% che deriva a sua volta dalla dualità in 4D.
Inoltre, poichè le simmetrie della teoria rimangono esplicite, le cariche dei campi della teoria possono essere identificate facilmente dalla funzione di partizione.

Lo scopo di questa tesi consiste nell'applicazione di questa procedura alla dualità Kutasov-Schwimmer-Seiberg che si differenzia da quella di Seiberg per la presenza di un ulteriore campo di materia nella rappresentazione aggiunta del gruppo di gauge. \\
La riduzione dimensionale di questa dualità era stata effettuata solo in teoria di campo, ma essa si basa su alcune assunzioni che non sono del tutto giustificate.
Il nostro lavoro non necessita di queste ipotesi e si propone quindi come una verifica indipendente a questi risultati.
\\
Infine, l'identità tra le funzioni di partizione in 3D per questa dualità corrisponde a una identità non banale tra funzioni iperboliche che tuttora non è stata ancora dimostrata in matematica.

\end{document}
