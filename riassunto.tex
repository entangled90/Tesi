\documentclass[a4paper,12pt]{article}

\usepackage{amsmath}
\usepackage{mathtools}
\usepackage{amstext}
\usepackage{amssymb}
\usepackage{amsthm}
%\usepackage{fullpage}
%%\usepackage{mathrsfs}
\usepackage[italian]{babel}
\usepackage[utf8]{inputenc}
\usepackage[T1]{fontenc}
%%\usepackage{natbib}
\usepackage{lmodern}
\usepackage[toc,page]{appendix}
\usepackage[a4paper]{geometry}
%\usepackage[nottoc,numbib]{tocbibind}
\geometry{a4paper,top=1.5cm,bottom=1.5cm,left=2cm,right=2cm,%
heightrounded,bindingoffset=5mm}





%%%%%%%%%%% DA MODIFICARE!!!!!
%%%%%%%%%%%%%%%%%
%%%%%%%%%%%%%%


%%%%%%%%%%%%%%%%%%%OOOOOO
\author{
 Carlo Sana  --- matricola nr 726409 \\
 Laurea Magistrale in Fisica Teorica\\
%  cell 3295795286 \\
Relatrice: Silvia Penati 
Correlatore: Alberto Zaffaroni\\
}
\date{29 Giugno 2015}
\title{ \textbf{4D to 3D reduction of Seiberg duality for $SU(N)$ susy gauge theories with adjoint matter: a partition function approach }}	
 


\pagenumbering{gobble}

\begin{document}
\maketitle
In ogni ambito della fisica i sistemi fortemente accoppiati sono difficilmente trattabili e raramente esistono approssimazioni che possono essere applicate a tali sistemi.
\\
In fisica delle alte energie negli ultimi vent'anni sono state scoperte diverse dualità che mettono in relazione teorie debolmente accoppiate, studiabili con un approccio perturbativo, con teorie fortemente interagenti.
L'utilità di queste dualità è indiscussa, in quanto forniscono strumenti per analizzare fasi non perturbative di una teoria. \\
Le dualità più studiate sono senza ombra di dubbio la dualità di Seiberg e l'AdS/CFT, più generalmente chiamata \emph{gauge/gravity duality}. 



\end{document}