\documentclass[a4paper,12pt]{article}

\usepackage{amsmath}
\usepackage{mathtools}
\usepackage{amstext}
\usepackage{amssymb}
\usepackage{amsthm}
%\usepackage{fullpage}
%%\usepackage{mathrsfs}
\usepackage[italian]{babel}
\usepackage[utf8]{inputenc}
\usepackage[T1]{fontenc}
%%\usepackage{natbib}
\usepackage{lmodern}
\usepackage[toc,page]{appendix}
\usepackage[a4paper]{geometry}
%\usepackage[nottoc,numbib]{tocbibind}
\geometry{a4paper,top=1.5cm,bottom=1.5cm,left=2cm,right=2cm,%
heightrounded,bindingoffset=5mm}





%%%%%%%%%%% DA MODIFICARE!!!!!
%%%%%%%%%%%%%%%%%
%%%%%%%%%%%%%%


%%%%%%%%%%%%%%%%%%%OOOOOO
\author{
 Carlo Sana  --- matricola nr 726409 \\
 Laurea Magistrale in Fisica Teorica -- 29 Giugno 2015\\
%  cell 3295795286 \\
Relatrice: Silvia Penati 
Correlatore: Alberto Zaffaroni\\
}
\date{}
\title{ \textbf{4D to 3D reduction of Seiberg duality for $SU(N)$ susy gauge theories with adjoint matter: a partition function approach }}	
 


\pagenumbering{gobble}

\begin{document}
\maketitle
In ogni ambito della fisica i sistemi fortemente accoppiati sono difficilmente trattabili e raramente esistono approssimazioni che possono essere applicate ad essi.\\
In fisica delle alte energie negli ultimi vent'anni sono state scoperte diverse dualità che mettono in relazione teorie debolmente accoppiate, studiabili con un approccio perturbativo, con teorie fortemente interagenti.
Le dualità più studiate sono senza ombra di dubbio la dualità di Seiberg e l'AdS/CFT.
%più generalmente chiamata \emph{gauge/gravity duality}. 
In entrambi i casi, possiamo calcolare le osservabili della teoria fortemente accoppiata passando alla teoria duale. \\
La dualità di Seiberg è anche chiamata dualità elettrica-magnetica in quanto presenta alcune somiglianze con la simmetria nella scambio di campi elettrici e magnetici trovata da Dirac nel 1931.
In generale, le dualità di Seiberg associano ad una teoria di gauge supersimmetrica, la cosiddetta teoria elettrica, una teoria duale, quella magnetica.
Essa è una teoria di gauge con un diverso numero di colori rispetto alla teoria elettrica e contiene i mesoni della teoria elettrica come particelle fondamentali e non come particelle composte, come nella teoria elettrica.
Per un particolare intervallo del numero di colori e di sapori, le due teorie a basse energie sono superconformi e descrivono la stessa fisica, anche se ad alte energie hanno una dinamica diversa.
Sono note vari esempi di questa dualità che differiscono per il gruppo di gauge o per il contenuto di materia della teoria.\\
Nonostante la dualità di Seiberg sia stata scoperta in quattro dimensioni nel 1994, pochi anni dopo fu scoperto un analogo tridimensionale.
Tuttavia, non era chiaro se ci fosse una relazione tra dualità in quattro dimensioni e dualità in tre.
Solo recentemente è stato scoperto un modo di effettuare una riduzione dimensionale delle dualità in quattro dimensioni in modo da preservare la dualità fra le teorie. \\
La riduzione delle dualità può essere effettuata direttamente attraverso la funzione di partizione che per queste teorie può essere calcolata esattamente come limite dell'indice superconforme, definito per teorie in quattro dimensioni con una delle dimensioni compattificata su un cerchio ed identico fra le due teorie.
Nel limite in cui il raggio del cerchio tende a zero, otteniamo una funzione di partizione di una teoria in tre dimensioni soggetta ad alcuni vincoli dovuti al processo di riduzione.
\'E possibile eliminare questi vincoli manipolando opportunamente la funzione di partizione, ottenendo così una teoria puramente tridimensionale. 
Applicando questa procedura ad entrambe le teorie otteniamo due funzioni di partizione diverse, la cui uguaglianza è garantita dall'uguaglianza degli indici superconformi in quattro dimensioni.
Inoltre, le cariche dei campi della teoria possono essere identificate facilmente dalla funzione di partizione, siccome le simmetrie della teoria rimangono esplicite.
\\
Il nostro lavoro consiste nell'applicazione di questa procedura alla dualità KSS che differisce da quella di Seiberg per la presenza di un campo di materia nella rappresentazione aggiunta del gruppo di gauge. 
La riduzione dimensionale di tale teoria  sulla funzione di partizione non era presente in letteratura e il nostro lavoro rappresenta una verifica independente dei risultati ottenuti in teoria di campo che al momento si basano su alcune assunzioni non del tutto giustificate.


\end{document}