\documentclass[a4paper,12pt]{article}

\usepackage{amsmath}
\usepackage{mathtools}
\usepackage{amstext}
\usepackage{amssymb}
\usepackage{amsthm}
%\usepackage{fullpage}
%%\usepackage{mathrsfs}
\usepackage[italian]{babel}
\usepackage[utf8]{inputenc}
\usepackage[T1]{fontenc}
%%\usepackage{natbib}
\usepackage{lmodern}
\usepackage[toc,page]{appendix}
\usepackage[a4paper]{geometry}
%\usepackage[nottoc,numbib]{tocbibind}
\geometry{a4paper,top=1.5cm,bottom=1.5cm,left=2cm,right=2cm,%
heightrounded,bindingoffset=5mm}





%%%%%%%%%%% DA MODIFICARE!!!!!
%%%%%%%%%%%%%%%%%
%%%%%%%%%%%%%%


%%%%%%%%%%%%%%%%%%%OOOOOO
\author{
 Carlo Sana  --- matricola nr 726409 \\
 Laurea Magistrale in Fisica Teorica\\
%  cell 3295795286 \\
Relatrice: Silvia Penati 
Correlatore: Alberto Zaffaroni\\
}
\date{29 Giugno 2015}
\title{ \textbf{4D to 3D reduction of Seiberg duality for $SU(N)$ susy gauge theories with adjoint matter: a partition function approach }}	
 


\pagenumbering{gobble}

\begin{document}
\maketitle
In ogni ambito della fisica i sistemi fortemente accoppiati sono difficilmente trattabili e raramente esistono approssimazioni che possono essere applicate a tali sistemi.
\\
In fisica delle alte energie negli ultimi vent'anni sono state scoperte diverse dualità che mettono in relazione teorie debolmente accoppiate, studiabili con un approccio perturbativo, con teorie fortemente interagenti.
Le dualità più studiate sono senza ombra di dubbio la dualità di Seiberg e l'AdS/CFT, più generalmente chiamata \emph{gauge/gravity duality}. 
In entrambi i casi, possiamo calcolare le osservabili della teoria fortemente accoppiata passando alla teoria duale. \\
La dualità di Seiberg è anche chiamata dualità elettrica-magnetica in quanto presenta alcune somiglianze con la simmetria nella scambio di campi elettrici e magnetici trovata da Dirac nel 1931.
In generale, le dualità di Seiberg associano ad una teoria di gauge supersimmetrica, la teoria elettrica, una teoria duale, quella magnetica.
Essa è una teoria di gauge con un diverso numero di colori rispetto alla teoria elettrica e contiene i mesoni della teoria elettrica come particelle fundamentali e non come particelle composte, come nella teoria elettrica.
Per un particolare intervallo del numero di colori e di sapori, le due teorie a basse energie sono superconformi e descrivono la stessa fisica, anche se ad alte energie hanno una dinamica diversa.
Sono note vari esempi di questa dualità che differiscono per il gruppo di gauge o per il contenuto di materia della teoria.\\
Nonostante la dualità di Seiberg sia stata scoperta in quattro dimensioni nel 1994, pochi anni dopo fu scoperto un analogo tridimensionale.
Tuttavia, non era chiaro se ci fosse una relazione tra dualità in quattro dimensioni e dualità in tre.
Solo recentemente è stasto scoperto un modo di effettuare una riduzione dimensionale delle dualità in quattro dimensioni.



\end{document}