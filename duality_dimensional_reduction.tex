%!TEX root = tesi.tex

\chapter{Reduction of 4D dualities to 3D}

%OUTLINE
% 


In the previous chapter we introduced few examples of electric-magnetic dualities in four and three spacetime dimensions.
It is natural to wonder what is the relation between them since they show some similiraties but at the same time the dynamics of field theories in three and four dimensions is different.\\
For example, in three dimensions there are no anomalies and the moduli space features the Coulomb branch. 
Moreover, in three dimensions the axial symmetry is not broken by anomalies and we have a topological symmetry, enforced by a Bianchi identity.\\
The presence of scalars in the vector multiplets allows real mass terms for fields charged under global symmetries, whereas in four dimensions we could only generate holomorphic mass terms.\\
Despite these differences we will be able to dimensionally reduce four dimensional dualities to three dimensional dualities and in some cases we will find dualities that weren't known before as stand-alone three dimensional dualities.\\

\section{General procedure of reducing dualities}
The n{\"a}ive dimensional reduction of the four dimensional dualities does not yield a three dimensional duality.
The dimensional reduction of four dimensional theories can be achieved compactifying a spatial dimension into $S^1$ with radius $r$ and then going in the limit $r \, \rightarrow \, 0$. \\
Let's analyse the behaviour of the the theory at finite radius.
In four dimensional theories, the strong coupling scale is given by
\begin{equation}
 \Lambda^{b} = \exp{ \left(  - \frac{8 \pi^2}{g_4^2}\right)}
 \end{equation} 
where $b$ is the one-loop $\beta$ function coefficient.
Recall the following relation for strong coupling scales between dual theories 
\begin{equation}
  \Lambda^{b}  \tilde{\Lambda}^{b} = (-1)^{N_f - N_c}
  \label{eqn:4d_strong_scale_relation}
\end{equation}
After compactifying a dimension, the strong coupling scale is modified \cite{Aharony:2013dha} since  
\begin{equation}
g_4^2 =  2 \pi r g_3^2 \qquad \rightarrow \qquad \Lambda^{b} = \exp{ \left(  - \frac{4 \pi}{ r g_3^2}\right)}
\end{equation}
The result of this relation is that if we take the limit $r \, \rightarrow \, 0$,
the strong coupling scale goes to zero very fast.
Moreover, it's clear that the relation \eqref{eqn:4d_strong_scale_relation} cannot hold anymore, since $\Lambda \, , \, \tilde{\Lambda} \, \rightarrow 0$ at the same time.\\
Thus, the $r \, \rightarrow \, 0$ limit does not commute with the infrared limit at fixed coupling needed in the duality.\\
As a result, we can invert the order of the limits.
We can take the limit in which we keep $\Lambda \, , \, \tilde{\Lambda}$ and $r$ fixed and look at energies $E \ll \Lambda \, , \, \tilde{\Lambda} \,, \, \frac{1}{r}$. 
In this limit, we are in the infrared, at fixed coupling but with a finite radius.
However, this is not a problem since the theory doesn't see the compactified dimensions since it hasn't enough energy to excite the Kaluza-Klein modes associated to it.
As a result, the dynamics of the theory is effectively three dimensional.\\
The theories we find with this procedure have few properties that differentiate them to purely three-dimensional theories.
The scalar associated to the compactified direction of $A_3$ is periodic, with period $\frac{1}{r}$ since the holonomy of the gauge field on the circle is gauge invariant
\begin{equation}
 P\left( \exp{ \left( i \oint A_3 \right)}  \right) 
\end{equation}
As a result the scalar is compact with period $\sim \frac{1}{r}$ whereas the three dimensional one is not.\\
Four dimensional theories on the circle have an additional non-perturbative superpotential generated by a Kaluza-Klein monopole,alternatively called twisted instanton. 
This particular type of instanton/monopole configuration
\footnote{Using this expression we refer to the fact that 't Hooft/Polyakov monopoles in four dimension can be interpreted as three-dimensional instantons after ignoring the time dimension.}
of the gauge field is generated through a non periodic gauge transformations on an instanton configuration of the gauge field.
Even though the transformation is not periodic around the circle, the gauge field obtained acting with it remains periodic. 
For this reason it's associated to a different topological sector with respect to standard instanton/monopole configurations \cite{Davies:1999uw}. \\
The superpotential generated reads
\begin{equation}
W = \eta Y_{low}
\end{equation}
where $\eta = \Lambda^b$ and $Y_{low}$ is the Coulomb branch coordinate we defined in  .\\
Theories obtained with this method are dual to each other and they differ from truly three-dimensional theories for the reasons we explained above.
For example, the introduction of the $\eta$ superpotential breaks the axial symmetry that is anomalous in four dimensions, but is allowed in three-dimensional theories.\\
After these consideration we would like to transform these dualities with finite radius $r$ into truly three-dimensional dualities, i.e. dualities for theories well-defined at high energies.
The compactness of the Coulomb branch is not really a problem since for $SU(N)$ gauge theories the $\eta$ superpotential lifts completely the Coulomb branch together with AHW superpotential \eqref{eqn:AHW_superpotential}.
The Coulomb branch $ U(N)$ gauge theories is not completely lifted, but we will see that we can take a limit in which we focus only in specific points of the Coulomb branch and the compactness of the Coulomb branch becomes irrelevant.\\
Regarding the $\eta$ superpotential, it's possible to find monopole operators $Y_{high}$, well-defined at high energies, that flow to the Coulomb branch coordinates $Y_{low}$ at low energies. 
Using these operators it is possible to define the theories in the ultraviolet, thus turning the dualities we find by reduction as standalone three-dimensional dualities.


\section{Dimensional reduction of Seiberg duality with $SU(N_c)$ gauge group}
We introduced Seiberg duality in four dimensions in section \ref{sec:seiberg_duality_4d} but we were its analog in three dimensions was discovered only by the procedure of dimensional reduction that we just introduced.\\
The superpotential that is generated for four dimensional theories compactified is given by the sum of the AHW superpotential and the $\eta$ superpotential and together they lift the Coulomb branch completely.\\

\subsubsection{Comparison of moduli spaces}
%For $N_f \geq N_c$ the four dimensional theory admits a stable vacuum.
\begin{comment}
For $N_f = N_c$ it can be described by the effective superpotential
\begin{equation}
 W=  \lambda \left( B \tilde{B} - \Det{(M)} +    \right)
\end{equation}
which modifies the Higgs branch and lifts the Coulomb branch.
For greater number of flavours the Higgs branch remains identical while the Coulomb branch remains lifted.\\
\end{comment}
From the previous sections we know that three-dimensional theories perturbed by $\eta Y$ should be the same at low energies as four dimensional theories on the circle. 
The four dimensional duality grants that the compactified theory is dual to a four dimensional magnetic theory. 
Such theory can be mapped into a three dimensional theories by the same arguments we used for the electric theory.  
We will compare the moduli spaces between four and three dimensional theories to see if this argument is valid in this case.\\
For $N_f = N_c -1$ the four dimensional theory does not admit a vacuum but the unperturbed three dimensional theory has a moduli space subject to the constraint \eqref{eqn:suN-3d-nf-nc-1-vacuum}.
Adding the $\eta$ superpotential result in a theory without a stable vacuum, as the four dimensional theory.\\
For $N_f = N_c$ the four dimensional theory has an effective description that can be implemented by a Lagrange multiplier that reads
\begin{equation}
  W_{4d} = \lambda \left( B \tilde{B} - \Det{(M)} + \Lambda^{2 N_c} \right) = \lambda \left( B \tilde{B} - \Det{(M)} + \eta \right) 
  \end{equation}  
since $b = 3 N_c - N_f = 2 N_c$ for these particular values.\\
The perturbed three-dimensional theory has an effective description given by
\begin{equation}
	W_{3d} = Y \left( B \tilde{B} - \Det{(M)} + \eta \right)
\end{equation}
which suggests the identification between $\lambda$ and $Y$.\\
For $N_f = N_c + 1$ it can be demonstrated that the moduli spaces match but the situation is a little more involved and we refer to \cite{Aharony:2013dha} for a more complete discussion.\\ 

\subsection{Flow to a pure three-dimensional duality}
It is possibile to flow to a duality between three-dimensional theories withouth the $\eta$-superpotential with a flow obtained by the assignment of real masses.\\
Let's consider a theory with $N_f + 1$ flavours and let's give a real mass $\hat{m}$ to the last flavour.
We will set in the limit in which $\hat{m} \rightarrow \infty$ in order to integrate out the last flavour.
This can be achieved by giving expectation values to the fields associated to the diagonal $SU(N_f +1 ) \times U(1)_B$ flavour symmetry given by
\begin{equation}
  \hbox{diag}(0, \dots , m ) = \mbox{diag} (m_B - M, \dots , m_B + N_f M) 
   \end{equation}   
   where $M$ is associated to $SU(N_f+1)$ and $m_B$ to $U(1)_B$.\\
 The real masses are easily mapped to the dual theory by considering the charges of the quarks under the global symmetries in question.
 As a result the first $N_f$ flavours get a mass $\hat{m}_1$ and the last one a mass of $\hat{m}_2$
 \begin{equation}
  \hat{m}_1 = \frac{ \hat{m}}{N_f - N_c +1} \qquad \hat{m}_2 = \frac{ \hat{m} (N_c - N_f)}{N_f -N_c +1 }
 \end{equation}
In the electric theory we will be interested in configrations that remain at a finite distance from the origin in the Coulomb branch
, in the limit $\hat{m} \rightarrow 0 $.
For these type of vacua, the last flavour is massive and can be ignored at low energies, resulting in an effective $SU(N_c)$ theory with $N_f$ flavours.\\
The monopole operator $Y_{high}$ is related to the low-energy coordinate $Y_{low}$ by the relation $Y_{low} = Y_{high}/ m $, where $m$ is the complex mass of quark.
Since the real mass is given by a component of a vector multiplet, the superpotential $W = \eta Y_{high}$ vanishes, since we didn't assign any complex mass to the quark.\\
As a result, we obtain a $SU(N_c)$ gauge theory with $N_f$ massless flavours and no $\eta$-superpotential as the electric theory. 
The absence of the $\eta$-superpotential results in a theory with axial symmetry.
The charges of the fields are given by
\begin{equation}
\begin{array}{ c | c | c c c c c}
  & SU(N_c) & U(1)_R & U(1)_A & U(1)_B  & SU(N_f)_L & SU(N_f)_R \\
 \hline
 & \\
 Q &N_c & 1 & -1 & 1 & \overline{N_f} & 1 \\  
 \tilde{Q}& \overline{N_c} & 1  & -1 & -1  & 1 & {N_f}  \\  
 Y & 1 & 2(N_c - N_f -1 ) & -2 N_f  & 0 &  1 & 1 \\
\end{array}
\end{equation}
The magnetic theory does not have vacuum at the origin of the Coulomb branch since every flavour gets a mass proportional to $\hat{m}$.
The vacuum configuration that has the bigger number of massless quarks is given by following the vacuum expectation value of $\tilde{\sigma}$
\begin{equation}
  \tilde{\sigma} = \diag(\overbrace{- \tilde{m}_1, \dotsc,-\tilde{m}_1}^{N_f - N_c \; \hbox{values}}, -\tilde{m}_2 )
 \end{equation} 