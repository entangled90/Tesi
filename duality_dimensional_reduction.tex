%!TEX root = tesi.tex

\chapter{\bfseries Reduction of 4D dualities to 3D}

%OUTLINE
% 


In the previous chapters we introduced few examples of electric-magnetic dualities in four and three spacetime dimensions.
It is natural to wonder what is the relation between them since they show some similiraties but at the same time the dynamics of quantum field theories is different in different spacetime dimensions.\\
For example, in three dimensions there are no anomalies and the moduli space exhibits a Coulomb branch. 
Moreover, in three dimensions the axial symmetry is not broken by anomalies and theories may have an additional topological symmetry, associated to abelian factors in the gauge group,  which is enforced by a Bianchi identity.\\
The presence of a scalar field in every vector multiplet allows for real mass terms for fields charged under global symmetries, whereas in four dimensions we could only generate holomorphic mass terms through superpotential terms.\\
Despite these differences, in general four dimensional dualities can be dimensionally reduced to three dimensions and in some cases the reduction process will unveil new dualities in three dimensions.
\\

\section{General procedure of reducing dualities}
The n{\"a}ive dimensional reduction of four dimensional dualities, which can be achieved by compactifying a spatial dimension onto $S^1$ with radius $r$ and then going in the limit $r \, \rightarrow \, 0$, in general doesn't result in a three dimensional duality .
 \\
Let's analyse the behaviour of the the theory at finite radius to understand the reasons why the n\"aive reduction doesn't work.\\
In four dimensional theories, the strong coupling scale is given by
\begin{equation}
 \Lambda^{b} = \exp{ \left(  - \frac{8 \pi^2}{g_4^2}\right)}
 \end{equation} 
where $b$ is the one-loop $\beta$ function coefficient and we set the renormalization scale $\mu$ to 1 for convenience.
Recall the following relation for strong coupling scales between dual theories 
\begin{equation}
  \Lambda^{b}  \tilde{\Lambda}^{b} = (-1)^{N_f - N_c}
  \label{eqn:4d_strong_scale_relation}
\end{equation}
After compactifying one dimension, the strong coupling scale is modified \cite{Aharony:2013dha} since  $g_4^2 =  2 \pi r g_3^2$ and as a result
\begin{equation}
  \Lambda^{b} = \exp{ \left(  - \frac{4 \pi}{ r g_3^2}\right)}
\end{equation}
The consequence of the compactification is that if we take the limit $r \, \rightarrow \, 0$,
the strong coupling scale goes to zero.
Moreover, it's clear that the relation between strong coupling scales of dual theories \eqref{eqn:4d_strong_scale_relation} is incompatible with this limit, since both $\Lambda $ and $ \tilde{\Lambda} $ go to zero.\\
Thus, the $r \, \rightarrow \, 0$ limit does not commute with the infrared limit at fixed coupling needed in the duality.\\
We can take the limit in which we keep $\Lambda \, , \, \tilde{\Lambda}$ and $r$ fixed and look at energies $E \ll \Lambda \, , \, \tilde{\Lambda} \,, \, \frac{1}{r}$. 
In this limit, we are in the infrared, at fixed coupling but with a finite radius.
However, this is not a problem since the effective theory at low energy doesn't see the compactified dimension.
As a result, the dynamics of the theory is effectively three dimensional.\\
The theories we find with this procedure have few properties that distinguish them from purely three-dimensional theories.
The scalar associated with the holonomy of the gauge field along the compactified direction is periodic, because
\begin{equation}
 P\left( \exp{ \left( i \oint A_3 \right)}  \right) 
\end{equation}
is gauge invariant.\\
As a result the scalar of the compactified theory is compact whereas the three dimensional one is not. 
%The period of the compactified scalar depends on the global properties of the gauge group. For $SU(N)$ and $U(N)$ gauge groups the period is $\sim \frac{1}{r}$ while for other groups such as $SP(2N)$ or $SO(N)$ its expression is more involved.\\
Four dimensional theories on the circle have an additional non-perturbative superpotential generated by a Kaluza-Klein monopole.
This particular type of instanton/monopole configurations
\footnote{Using this expression we refer to the fact that 't Hooft/Polyakov monopoles in four dimension can be interpreted as three-dimensional instantons after ignoring the time dimension.}
of the gauge field is generated through a non-periodic gauge transformations on an instanton configuration of the gauge field.
Even though the transformation is not periodic around the circle, the gauge field obtained acting with it remains periodic. 
For this reason it is associated to a different topological sector with respect to standard instanton/monopole configurations \cite{Davies:1999uw}. \\
The generated superpotential reads
\begin{equation}
W = \eta Y_{low}
\label{eqn:W_eta_superpotential_first}
\end{equation}
where $\eta = \Lambda^b$ and $Y_{low}$ is the Coulomb branch coordinate we defined in \eqref{eqn:definition_of_Y_sun}.\\
Theories obtained with this method are dual to each other and they differ from truly three-dimensional theories for the reasons we explained above.
For example, the introduction of the $\eta$ superpotential breaks the axial symmetry that is anomalous in four dimensions, but is allowed in three-dimensional theories.\\
After these considerations we would like to flow from these dualities with finite radius $r$ to truly three-dimensional dualities, i.e. dualities for theories well-defined at high energies.
The compactness of the Coulomb branch is not really a problem since for $SU(N)$ gauge theories the $\eta$ superpotential lifts completely the Coulomb branch together with the Affleck-Harvey-Witten superpotential \eqref{eqn:AHW_superpotential}.
%The Coulomb branch $ U(N)$ gauge theories is not completely lifted, but we will see that we can take a limit in which we focus only in specific points of the Coulomb branch and the compactness of the Coulomb branch becomes irrelevant.\\
Regarding the $\eta$ superpotential, in the case we will analyse it is possible to find monopole operators $Y_{high}$, well-defined at high energies, that flow to the Coulomb branch coordinates $Y_{low}$ at low energies. 
Using these operators it is possible to define the theories in the ultraviolet, thus turning the dualities we find by reduction as standalone three-dimensional dualities.


\section{ Dimensional reduction of $SU(N_c)$ Seiberg duality}
We introduced Seiberg duality in four dimensions in section \ref{sec:seiberg_duality_4d} but its analog in three dimensions was discovered only by the procedure of dimensional reduction that we will introduce more precisely in this section.\\
The superpotential that is generated from four dimensional compactified theories is given by the sum of the AHW superpotential \eqref{eqn:def_AHW_Superpotential_Sun} and the $\eta$ superpotential \eqref{eqn:W_eta_superpotential_first} and together they lift the Coulomb branch completely.

\subsubsection{Comparison of moduli spaces}
%For $N_f \geq N_c$ the four dimensional theory admits a stable vacuum.
\begin{comment}
For $N_f = N_c$ it can be described by the effective superpotential
\begin{equation}
 W=  \lambda \left( B \tilde{B} - \Det{(M)} +    \right)
\end{equation}
which modifies the Higgs branch and lifts the Coulomb branch.
For greater number of flavours the Higgs branch remains identical while the Coulomb branch remains lifted.\\
\end{comment}
From the previous sections we know that three-dimensional theories perturbed by $\eta Y$ should be the same at low energies as four dimensional theories on the circle. 
From the four dimensional duality we know that the compactified theory is dual to a four dimensional magnetic theory. 
Such theory can be mapped into a three dimensional theory by the same arguments we used for the electric theory.  
We will compare the moduli spaces between four and three dimensional theories to see if this argument is valid in this case.\\
For $N_f = N_c -1$ the four dimensional theory does not admit a vacuum but the unperturbed three dimensional theory has a moduli space subject to the constraint \eqref{eqn:suN-3d-nf-nc-1-vacuum}.
Adding the $\eta$ superpotential result in a theory without a stable vacuum, as the four dimensional theory.\\
For $N_f = N_c$ the four dimensional theory has an effective description that can be implemented by a Lagrange multiplier that reads
\begin{equation}
  W_{3d }^{S^1} = \lambda \left( B \tilde{B} - \Det{(M)} + \Lambda^{2 N_c} \right) = \lambda \left( B \tilde{B} - \Det{(M)} + \eta \right) 
  \end{equation}  
since $b = 3 N_c - N_f = 2 N_c$ for these particular values.\\
The three-dimensional theory with the additiona $\eta$ superpotential has an effective description given by
\begin{equation}
	W_{3d} = Y \left( B \tilde{B} - \Det{(M)} + \eta \right)
\end{equation}
which suggests the identification between $\lambda$ and $Y$.\\
For $N_f = N_c + 1$ it can be demonstrated that the moduli spaces match but the situation is a little more involved and we refer to \cite{Aharony:2013dha} for a more complete discussion.\\ 

\subsection{Flow to a pure three-dimensional duality}
\label{sec:seib_4dto3d_flow_massive}
Let's consider a three dimensional theory with $\eta $ superpotential and with $N_f + 1$ flavours and let's give a real mass $\hat{m}$ to the last flavour.
We will set in the limit in which $ m \rightarrow \infty$ in order to integrate out the last flavour.
This can be achieved by giving expectation values to the fields associated to the diagonal $SU(N_f +1 ) \times U(1)_B$ flavour symmetry given by
\begin{equation}
\begin{aligned}
  \hbox{diag}(0, \dots , m ) = \mbox{diag} (m_B - M, \dots , m = m_B + N_f M)
  \\
   \hbox{with } \; m_B = M \; \rightarrow m = M(N_f+1)  
\end{aligned}
   \end{equation}   
  where $M$ is associated to the diagonal $SU(N_f+1)$ and $m_B$ to $U(1)_B$.\\
 The real masses are easily mapped to the dual theory by considering the charges of the quarks under the global symmetries in question (cfr. tables \eqref{tab:suN_elect_charge} and \eqref{tab:sun_magnetich_charges}).
 \begin{equation}
  M \; \rightarrow \;  - M \qquad m_B \; \rightarrow \; \frac{N_c}{N_f - N_c +1} m_B
 \end{equation}
 As a result the first $N_f$ flavours get a mass $\tilde{m}_1$ and the last one a mass $\tilde{m}_2$
 \begin{equation}
  \tilde{m}_1 = \frac{ \tilde{m}}{N_f - N_c +1} \qquad \tilde{m}_2 = \frac{ \tilde{m} (N_c - N_f)}{N_f -N_c +1 }
 \end{equation}
In the electric theory we will be interested in configrations that remain at a finite distance from the origin in the Coulomb branch in the limit ${m} \rightarrow 0 $, i.e. with the VEV of $\sigma_i$ finite.
For these types of vacua, the last flavour is massive and can be integrated out at low energies, resulting in an effective $SU(N_c)$ theory with $N_f$ flavours.
Since we integrated out an equal number of left and right fermions, no Chern-Simons terms are generated \cite{Aharony:1997bx}.
\\
The monopole operator $Y_{high}$ is related to the low-energy coordinate $Y_{low}$ by the relation $Y_{low} = Y_{high}/ m $, where $m$ is the complex mass of quark \cite{Aharony:2013dha}.
Since the real mass is given by a component of a vector multiplet, the superpotential $W = \eta Y_{high}$ vanishes, since we didn't assign any complex mass to the quark.\\
As a result, the electric theory we obtain is a $SU(N_c)$ gauge theory with $N_f$ massless flavours and no $\eta$-superpotential. 
The absence of the $\eta$-superpotential results in a theory with axial symmetry, since it was the only term that was breaking it.
The charges of the fields are given by
\begin{equation}
\begin{array}{| c | c | c c c c c| }
 \hline 
  & SU(N_c) & U(1)_R & U(1)_A & U(1)_B  & SU(N_f)_L & SU(N_f)_R \\
 \hline

 Q &N_c & 1 & -1 & 1 & \overbar{N_f} & 1 \\  
 \tilde{Q}& \overbar{N_c} & 1  & -1 & -1  & 1 & {N_f}  \\  
 Y & 1 & 2(N_c - N_f -1 ) & -2 N_f  & 0 &  1 & 1 \\
 \hline
\end{array}
\end{equation}
The magnetic theory does not have vacuum at the origin of the Coulomb branch since each flavour acquire masses proportional to ${m}$.
The vacuum configuration with the great number of massless quarks is given by following the vacuum expectation value of the dual gluon $\tilde{\sigma}$
\begin{equation}
  \tilde{\sigma} = \diag \left( \overbrace{- \tilde{m}_1, \dotsc,-\tilde{m}_1}^{N_f - N_c \; \hbox{values}}, -\tilde{m}_2  \right)
 \end{equation}
Note that this expectation value is traceless, as it should be.\\
This vacuum breaks the gauge symmetry $SU(N_f-N_c + 1) \, \rightarrow \, SU(N_f - N_c) \times U(1)$. 
We can view it as $U(N_f - N_c)$ with quarks in the representations
\begin{equation}
\begin{array} { | c | c | c | }
 \hline
 \hbox{component}& \hbox{quark} & U(N_f - N_c)  \\
\hline
1,\dotsc,N_f & q & (N_f - N_c)_{\;1}\\
1,\dotsc,N_f & \tilde{q} & (\overbar{N_f - N_c})_{\;1}\\
N_f +1 & p & 1_{-(N_f-N_c)}\\
N_f +1 & \tilde{p} & 1_{\;(N_f-N_c)}\\ 
\hline
\end{array}
\end{equation}
The mesons' components created from quarks with the same real mass remain light, since left and right quarks have opposite charges and global symmetries. 
Thus, the massless mesons are a $N_f \times N_f$ matrix, that can be identified with the mesons $M$ of the electric theory, and a single singlet field $M^{N_f+1}_{N_f+1}$ which we will call $Y$, since we will identify later with the monopole operator $Y$.\\
The off-diagonal components, i.e $M^i_{N_f+1}$ and $M^{N_f+1}_i$ with $i=1,\dotsc,N_f$ have a mass proportional to $\hat{m}$ and can be integrated out.\\
The $\eta$ superpotential can be written in terms of $U(N_f - N_c)$ Coulomb branch variables that we defined in \eqref{eqn:un_moduli_space_coordinates}.
Since the gauge group comes from the breaking of $SU(N_f - N_c +1) $ the superpotential is written as
\begin{equation}
 W = \tilde{\eta} \tilde{V}_-
 \end{equation} 
In addition to this superpotential, there's a AHW term \cite{Affleck:1982as} in the superpotential, related to the breaking of the gauge group from $SU(N_f - N_c +1)$ to $SU(N_f - N_c) \times U(1)$ which is given by a term 
\begin{equation}
W = \tilde{V}_+
\end{equation}
The complete superpotential of the magnetic theory is given by
\begin{equation}
W_{mag} = M \, q \tilde{q} + Y \, p \tilde{p} + \tilde{\eta} \tilde{V}_- + \tilde{V}_+
\label{eqn:sun_reduced_duality_superpotential}
\end{equation}
At this point, $\tilde{\eta}$ does not play any role anymore and it can be absorbed in a redefinition of $\tilde{V_-}$.
The Coulomb branch of this theory is completely lifted by the last two terms of the superpotential.\\
The global symmetries of the theory consist of the $SU(N_f)_L \times SU(N_f)_R \times U(1)_R$ symmetries inherited from four dimensions, the topological symmetry $U(1)_J$ associated to the abelian factor in the gauge group and two abelian factors given by $U(1)_B$ and $U(1)_A$, that are not broken by the superpotential.\\
From the matching of the gauge invariant operators between the two theories we can fix most of the quantum numbers.
The baryons of the electric theory are matched in the magnetic theory to $b = q^{N_f-N_c} p$ and $\tilde{b} = \tilde{q}^{N_f - N_c} \tilde{b}$.\\
The $U(1)_B$ symmetry mixes with abelian term of the gauge group and it can be chosen arbitrarily.\\
The charges are the following
\begin{equation}
\begin{array} {| c | c | c c c c c |}
\hline
& U(N_f - N_c) & SU(N_f)_L & SU(N_f)_R & U(1)_B & U(1)_A & U(1)_R \\
\hline
q & (N_f - N_c)_{\; 1} & \overbar{N_f} & 1 & 0 & -1 & 1 \\
\tilde{q} & (\overbar{N_f - N_c})_{\;- 1} &   1& \overbar{N_f} & 0 & -1 & 1 \\
p & 1_{-(N_f - N_c)} & 1 &  1 & N_c & N_f & -(N_f - N_c) \\
\tilde{p} & 1_{(N_f - N_c)} & 1 &  1 & - N_c & N_f & -(N_f - N_c) \\
M & 1 & 1 & 1 & 0 & 2 & 0 \\
Y & 1 & 1 & 1 & 0 & - 2 N_f & 2 (N_f - N_c+1)\\
\hline
\tilde{V}_{\pm} & 1 & 1 & 1 &0 & 0 &2\\
\hline
\end{array}
\label{tab:seiberg_3d_charge_mag}
\end{equation}
The meson field $Y$ has the same quantum numbers of the $Y$ field in the electric and for this reason is natural to identify them.
The quantum numbers of $V_{\pm}$ are calculated by counting zero modes and the fact that are consistent with the superpotential \eqref{eqn:sun_reduced_duality_superpotential} is a check of the result.\\
As a result of this process we found a three-dimensional duality that wasn't known as a standalone duality before the method of dimensional reduction was discovered in \cite{Aharony:2013dha}.
The duality can be checked by the matching of the partition functions between the two theories. 
We will focus on this approach in the next chapters. \\
As an additional check of this duality we can use it to the reduce to the Aharony duality we introduced in section \ref{sec:aharony_duality}. 
Since it is a duality involving $U(N)$ gauge theories, we can achieve this by gauging the baryonic symmetry, since it has opposite charges for left and right quarks. \\
The Coulomb branch coordinates are now given by the independent operators $V_{\pm}$.
The $U(N_c)$ theory has an additional global symmetry $U(1)_J$ associated to the $U(1)$ factor in the gauge group. 
%Monopole operators are charged under this symmetry.\\
The duality maps the electric $U(N_c)$ theory with $N_f$ flavours with superpotential
\begin{equation}
W_{el} = \eta V_+ V_-
\end{equation}
to a $U(N_f - N_c)$ theory with $N_f $ flavours and mesons $M$ with
\begin{equation}
W_{mag} = \tilde{\eta} Y_+ Y_- + M q \tilde{q}
\end{equation}
Both theories have a one-dimensional Coulomb branch which is not lifted by the superpotential, since it only affects the $SU(N)$ factor of the group and a matching between the two is possible.\\
The same procedure can be applied to the duality we found without the $\eta$ superpotential.
The steps on the electric theory are similar to the $SU(N_c)$ case we described previously, while the charges of the magnetic theory we used in \eqref{tab:seiberg_3d_charge_mag} was chosen in such a way that the only fields charged under the baryonic symmetry were $p,\tilde{p}$.
As a result, the new sector of the theory is a $U(1)$ gauge theory with only one flavour.
Its low-energy description is given by a superpotential, relating its monopole operators $V_+,V_-$ with the meson $N= p \tilde{p}$ by
\begin{equation}
W = - V_+ V_- N
\end{equation}
The superpotential we obtained by the reduction \eqref{eqn:sun_reduced_duality_superpotential} contains a term given by $W = Y N $. The equation of motion of $N$ sets $Y= \tilde{V}_+ \tilde{V}_-$.
Integrating out the massive fields the monopole operators $\hat{X}_{\pm}$ of the low-energy theory $U(N_f - N_c)$ can be related to the high-energy monopoles$Y_{\pm}$ of $U(N_f - N_c) \times U(1)_B$ by the relation \cite{Aharony:2013dha}
\begin{equation}
Y_{\pm} \simeq \hat{X}_{\pm} \tilde{V}_{\mp}
\end{equation}
As a result the superpotential
\footnote{
  We rescaled the field $Y_-$ by a factor $\tilde{\eta}$ 
  since it does not play any role in the duality anymore. 
}
 is given by
\begin{equation}
W_{mag} = M q \tilde{q} + \hat{X}_- \tilde{V}_+ + \hat{X}_+ \tilde{V}_-
\end{equation}
which is the same superpotential of the magnetic theory in Aharony duality, after the identification of the singlets $\tilde{V}_{\pm}$ with the fields $V_{\pm}$ of the electric theory.



\section{Reduction of KSS duality}

\subsection{$U(N_c)$ duality}
We will first focus on the reduction of KSS duality introduced in section \ref{sec:kutasov_duality4d} but with $U(N_c)$ gauge group which is obtained by gauging the baryonic $U(1)_B$ symmetry.
The dynamics of the $U(1)$ factor in the gauge group is infrared free so it does not affect the duality.\\
The reduction of the theory to $\mathbb{S}^3 \times \mathbb{S}^1$ features a $\eta$ superpotential that is generated by Kaluza-Klein monopoles as in the previous section.\\
Because of the superpotential $\Tr{X^{k+1}} $ there are $2k $ unlifted directions in the moduli space that are parametrised by $t_{i,\pm}$ as in \eqref{eqn:3d_mag_monopoles_def}. 
The counting of the fermion zero modes for the KK-monopoles shows that they generate a superpotential \cite{Nii:2014jsa}
\begin{equation}
W_{\eta} = \eta \sum_{j=0}^{k-1} \, t_{j,+} \, t_{k-1-j,-} 
\end{equation}
In the magnetic theory a similar superpotential is generated, involving magnetic monopoles $\tilde{t}_{i,\pm}$ of the dual theory.
The addition of these superpotential guarantees that the duality is consistent with the compactification.\\
Since we are interested in the reduction to the Kim-Park duality (section \ref{sec:kim-park-duality} ) without $\eta$-superpotential, we will add real masses to some quarks as we did in section \ref{sec:seib_4dto3d_flow_massive}.\\
Since the baryonic symmetry is gauged, we need to start from $N_f +2 $ flavours and assign masses $(m,\tilde{m})$ associated to the background gauging of the $SU(N_f+2)_L \times SU(N_f +2 )_R$ global symmetry
\begin{align}
 m_{el} = & \diag \left( \underbrace{m_A,\dotsc,m_A}_{N_f}, M - \frac{m_A N_f}{2},-M +\frac{m_A N_f}{2}\right) \qquad \\
 \tilde{m}_{el} = & \diag \left( 
 \underbrace{
 m_A , \dotsc, m_A
 }_{N_f}
 ,-M +\frac{m_A N_f}{2}, M - \frac{m_A N_f}{2} \right)
\end{align}
This mass assignment breaks the symmetry group to $SU(N_f)_L \times SU(N_f)_R \times U(1)_A $, where $U(1)_A$ is the axial symmetry that the $\eta$-superpotential breaks. 
However, since we are in the limit in which $M $ is large, the $\eta$-superpotential vanish in this flow since the quarks don't have complex masses as in\cite{Aharony:2013dha}.
As a result, the theory flows to the electric theory of Kim-Park duality.\\
The magnetic theory is now a $U(k(N_f + 2) - N_c)$ gauge theory with $N_f +2$ flavours. 
The real masses of the electric theory are easily mapped to the magnetic theory by comparing the global symmetries
\begin{align}
 m_{mag} =&    \diag \left(  \underbrace{m_A , \dotsc, m_A}_{N_f} ,  M- \frac{m_A N_f}{2} , -M -\frac{m_A N_f}{2}\right)  \\
  \tilde{m}_{mag} =& \diag \left(
  \underbrace{
  m_A,\dotsc,m_A
}_{N_f}
  , -M - \frac{m_A N_f}{2}, M - \frac{m_A N_f}{2}\right)
\end{align}
%
%
The magnetic theory can be weakly perturbed by a polynomial superpotential in the adjoint matter field
\begin{equation}
W = \sum_{j=2}^{k+1} s_j \, \Tr{\tilde{X}^{j}}
\end{equation}
which in this case doesn't contain a term proportional to $\Tr{\tilde{X}}$ as in the $SU(N) $ case since it would force $\tilde{X}$ to be traceless.\\
The superpotential triggers a breaking of the gauge group 
\begin{equation}
 U(k (N_f + 2) - N_c ) \; \longrightarrow \; U(r_1) \times \dots \times U(r_k) \quad \sum_{j=1}^k r_i = k (N_f + 2) - N_c
  \end{equation}
As in section \ref{sec:kutasov_vacuum_struct_deformed}, in every vacuum the adjoint matter is massive and is integrated out, resulting in $k$ different SQCD sectors without adjoint matter.
The vacuum configuration for the adjoint scalar such that we have the largest number of massless particles is
\begin{equation}
   \tilde{\sigma}_{U(r_i)} = \diag \left(  0, \dots, - M , M \right)
  \end{equation}
that results in the breaking pattern
\begin{equation}
U(r_1) \times \dots U(r_k) \; \longrightarrow \; \left( U(r_1-2) \times U(1)^2 \right) \times  \dots \times \left( U(r_k-2) \times U(1)^2 \right) 
\end{equation}
%The $2k$ $U(1)$ subsectors resulting from the breaking can be dualized using mirror symmetry \cite{Aharony:1997bx} into $2k \; XYZ$ models, made of gauge singlets. 
%They interact with the monopole operators of the $U(r_i)$ sector they belonged to.\\
%The weak perturbation can be turned off while taking the $M \rightarrow \infty$ limit.
%In this way we find the magnetic theory of Kim-Park duality, where the electric monopoles are given by the singlets of the dualised $XYZ$ models with the correct superpotential with the monopoles of the magnetic theory \eqref{eqn:kimpark_superpotential_monopoles}.
The fields that remain massless in this vacuum are the following
\begin{itemize}
\item $\left(q_j^{i} , \tilde{q}_j^{i}\right)$ dual quarks with $i=1,\dotsc,k$ and$j=1,\dotsc,N_f$ which corresponds to the first $N_f$ flavours of quarks for every $U(r_i-2)$ sector tha remain massless
\item $\left(p_a^i , \tilde{p}_a^i\right)$ with $a=1,2$. They are the last two flavours of quarks charged under one of the two $U(1)$ sector for each value of $i$
\item $(M_j)^{i}_{\tilde{i}}$ mesons with $j=0,\dotsc,k-1$ coming from the first $N_f$ flavours of electric quarks.
\item $(M_j)^a$ mesons associated to $M^{N_f+1}_{N_f+1}$ and $M^{N_f+2}_{N_f+2}$
\end{itemize}
and the other fields obtain a large mass in this limit.\\
We can define the following composite operators out of these fields where for $\tilde{X}$ we mean its value in the vacuum since it is massive. 
\begin{align}
 (N_j)^i = & q_j^{i} \tilde{X}^j \tilde{q}_j^{i} \\
 \hat{(N_j)}_a^i =& p_a^i \tilde{X}^j \tilde{p}_a^i
\end{align}
The superpotential can be rewritten and is given by
\begin{equation}
W = \sum_{j=0}^{k-1} \sum_{i=1}^{k} M_j 
%q^i Y^{k-j-1} \tilde{q}^i 
N_{k-j-1}^i
+
\sum_{a=1}^{2}  \sum_{i=1}^{k} \sum_{j=0}^{k-1} 
M_j^a 
%p^a Y^{k-j-1} \tilde{p}^a
\hat{(N_j)}_a^i
\end{equation}
The two $U(1)$ sectors can be given an effective description using \eqref{eqn:moduli_space_u1_nfsuperpotential} which results in
\begin{equation}
W_{U(1)} = \sum_{a=1,2} \sum_{i=1}^{k} \hat{(N_0)}_a^i {v}^{i,a}_+ {v}^{i,a}_-
\end{equation}
where ${v}^{i,a}_{\pm}$ are the Coulomb branch coordinates of the $i$-th  $U(1)$ sector.\\
The breaking of each gauge group factor from $U(r_i) \rightarrow U(r_i-2) \times U(1)^2$ generates three Affleck-Harvey-Witten terms in the superpotential
\begin{equation}
W_{AHW} = \sum_{i=1}^k \left( \tilde{V}_{i,+} {v}^{i,1}_- + \tilde{V}_{i,-} {v}^{i,2}_+   \right)
\end{equation}
where $V_{i,\pm}$ are the Coulomb branch coordinates for the $U(r_i-2)$ sectors.\\
The equations of motion set the terms with $\hat{ (N_j)_a^i}$.
%and the $\eta$ superpotential to zero.
After sending the mass $M$ to infinity, we can weakly switch off the deformation in the adjoint field. 
As a result, the $U(r_i-2)$ factors recombine to $U(k N_f - N_c)$ with adjoint matter $Y$.\\
While removing the deformation, the 2k $U(1)$ factors become $U(k) \times U(k)$ each with one flavour and one adjoint matter field. 
Such a theory should not have a vacuum \eqref{eqn:kutasov_vacuum_condition} but the additional terms in the superpotential should modify this condition.
We assume that this enhancement to $U(k)$ does not modify the results and we will see in the next chapter with an independent argument on the partition function that our assumption is indeed correct.\\
The Coulomb branch coordinates of the $U(r_i-2)$ terms give the Coulomb branch coordinates of the gauge group $U(k N_f -N_c)$ while the coordinates of the $U(1)^2$ factors are identified with the chiral fields $v_{j,\pm}$ \eqref{tab:kim-park-magnetic} that are mapped to the Coulomb branch coordinates of the electric theory.
The superpotential is then given by
\begin{equation}
W = \Tr{\tilde{X}^{k+1}} + \sum_{j=0}^{k-1} M_j \, q Y^{k-j-1} \tilde{q} + \sum_{j=0}^{k-1} \left(  v_{j,+} \tilde{v_{k-1-j,-}} + v_{j,-} \tilde{v}_{k-1-j,+} \right)
\end{equation}
 which is the same as the one appearing in the magnetic theory of Kim-Park duality.














\subsection{ \boldmath $SU(N_c)$ duality }
\label{sec:reduction_kss_sun}
This duality first appeared in \cite{Park:2013wta} and it was obtained from the duality of the previous section by ungauging the $U(1)$ factor.
In order to derive the duality with $SU(N_c)$ gauge group we do not gauge the baryonic symmetry as in the previous case and we start directly from KSS duality in four dimensions.
\\
The Coulomb branch of this theory is different from a $U(N_c)$ theory and the monopole operators are now given by \cite{Nii:2014jsa}
\begin{align}
Y_i \, \sim \, & \exp \left(   \frac{ \Phi_i - \Phi_{i+1}}{g^2}  \right) \qquad 
Y = \prod_i Y_i
\\
V_{i,j} \, \sim \, & (X_{11})^i (X_{N_c N_c})^j\, Y
\end{align}
These monopole operators are related to the $U(N_c)$ ones by the relations
\begin{equation}
 V = V_+ V_- \qquad V_{i,j} = V_{i,+} V_{-,j}
 \end{equation} 
After the compactification on the circle, Kaluza-Klein monopoles generate the terms \cite{Nii:2014jsa}
\begin{equation}
W_{\eta, el} = \sum_{j=0}^{k-1} \eta \, V_{j,k-1-j} \qquad W_{\tilde{\eta}, mag} = \sum_{j=0}^{k-1 } \tilde{\eta} \, \tilde{V}_{j,k-1-j} 
\end{equation}  
which can be verified by the counting of the zero modes.\\
The electric theory is the reduction of the electric theory in four dimensions with the addition of the $\eta$ superpotential defined above.
The same is true for the magnetic theory too.\\
As in the previous cases, we can flow to a duality withouth these terms in the superpotential by a suitable assignment of real masses.\\ 
We start our reduction with $N_f+1$ flavours.
In the electric theory we assign the following values for real masses associated to the $SU(N_f+1)_L \times SU(N_f+1)_R \times U(1)_B$ global symmetry. 
The real mass of the left quarks are
\begin{align}
m_{el} &= 
%\diag \left(
\begin{pmatrix}
   (M_B - M)  + m_A \\ &  \ddots 
   \\ 
   %& \ddots  \\s
   &  & (M_B - M)  + m_A \\
   &  &  & (M_B + N_f M) - m_A N_f  \\
   %   \right) \\
  \end{pmatrix} \\
%&= \diag \left(  m_A, \dotsc, m_A, m - m_A N_f     \right) \quad m= N_F M + M_B\\
& = 
\begin{pmatrix}
   m_A \\ &  \ddots 
   \\ 
    & & m_A \\
    & &  & - m_A N_f + m  \\
   %   \right) \\
  \end{pmatrix} \qquad m = M_B + N_f M\\
  \end{align}
The real masses of the right quark $\tilde{Q}$
  \begin{align}
\tilde{m}_{el} &= 
\begin{pmatrix}
%\diag \left(   
%
M -M_B   + m_A \\ & \ddots \\ & &  M -M_B   + m_A \\
& & &  - M_B -  N_f M - m_A N_f \\
%     \right) \\
\\
\end{pmatrix}
\\
&= 
%\diag \left(
 \begin{pmatrix}
  m_A \\ & \ddots \\ & &  m_A \\ & &  &  - m - m_A N_f \\     
 \end{pmatrix}
  \qquad m = M_B + N_f M
 %\right) 
 \label{eqn:real_mass_reduction_el}
\end{align}
This mass assignment breaks the global symmetry group to $SU(N_f)_L \times SU(N_f)_R \times U(1)_A \times U(1)_B$.\\
By comparison of the charges of the fields under the global symmetries, we can map the masses to the magnetic theory
\begin{equation}
 \tilde{M} \, = \, - M \qquad \tilde{M}_B \, \rightarrow \, \frac{N_c}{k (N_f+1) - N_c} M_B \qquad \tilde{m}_A = m_A
\end{equation}
which results in
\begin{align}
m_{mag} & = 
\begin{pmatrix}
 m_1 - m_A \\
& \ddots \\
&&  m_1 - m_A\\
&&& m_2 + m_A N_f\\
\end{pmatrix}
\\
\tilde{m}_{mag} & =\
\begin{pmatrix}
     - m_1 - m_A \\ & \ddots \\ &&  - m_1 - m_A \\ &&& -m_2 + m_A N_f\\   
      \end{pmatrix}
\end{align}
where we defined
\begin{align}
m_1 & = \tilde{M_B} - \tilde{M} = \frac{k}{k (N_f +1) - N_c} m\\
m_2 & = \tilde{M_B} + N_f \tilde{M} = - \frac{ k N_f - N_c}{k (N_f +1) - N_c} m
\end{align}
The global symmetries in the magnetic theory are broken in the same way as in the electric one.\\
We can introduce a deformation to the superpotential for both theories
\begin{equation}
 W_{el}= \sum_{j=1}^{k} s_j \Tr{ X^{j} } \qquad W_{mag} = \sum_{j=1}^{k}\tilde{s}_j \Tr{ Y^{j}}
\end{equation}
Where the terms proportional to $ \Tr{X},\Tr{Y}$ are Lagrange multipliers to enforce the traceless of $X$ and $Y$.\\
By giving a large value to $m$ we flow to a theory withouth $\eta$ superpotential, as in the previous cases.
In the electric side, the deformation breaks the gauge group $SU(N_c) \rightarrow SU(i_1) \times \dots \times SU(i_k) \times U(1)^{k-1}$ with $\sum_j i_j = N_c$.\\
Taking the vacuum expectation value for $\sigma$ at the origin, the last quark flavour gets integrated out and we find a theory with $N_f$ flavours and no $\eta$ superpotential.
\\
In the magnetic side, the perturbation breaks the gauge group in a similar way, from $SU( k (N_f+1) - N_c)$ to
\begin{align} 
U(j_1) \times \dots \times U(j_k) / U(1) &  \simeq  SU(j_1) \times \dots \times SU(j_k) \times U(1)^{k}/U(1) \\  
 & \simeq  SU(j_1) \times \dots \times SU(j_k) \times U(1)^{k-1}\\
 & \hbox{with} \quad\sum_l j_l = k (N_f+1) - N_c  \;
\end{align} 
However, since each quark has a large real mass, the vacuum cannot be chosen at the origin and we need to choose the vacuum in order to have the largest number of massless fields.
As a result, the gauge group is broken further into
\begin{equation}
SU(i_1 -1 ) \times \dots \times SU(i_k -1 ) \times U(1)^{k} \times U(1)^{k-1}
\end{equation}
since the vacuum of $\tilde{\sigma}$ in every $SU(i_k)$ sector is given by  
\begin{equation}
\tilde{\sigma}_{i_1} = 
\begin{pmatrix}
 - m_1 & \\
  & \ddots  \\
  & & - m_1 & \\
  & & & - m_2 \\
\end{pmatrix}
 \; ,
 \dotsc 
,
\;
\tilde{\sigma}_{i_k} = 
\begin{pmatrix}
 - m_1 & \\
  & \ddots  \\
  & & - m_1 & \\
  & & & - m_2 \\
\end{pmatrix}
\label{eqn:reduction_sun_sigma_vev}
\end{equation}
The $U(1)^{k-1}$ factor is associated to the breaking of the group from the vacuum expectation values of $Y$ while the $U(1)^k$ factor results from the breaking caused by the vacuum choice of $\til{\sigma}$.\\
Before doing so, we are going to use $3D \, \mN=2$ mirror symmetry \cite{Aharony:1997bx} in the $U(1)^{k-1}$ sector of the deformed theory.
As a result, this sector will be described by a $ U(1)$  gauge theory with $k$ flavours $(b_i, \tilde{b}_i)$.\\
%We used mirror symmetry in order to dualize the $U(1)^{k-1}$ gauge sector in the deformed theory.
We need to stress that we used mirror symmetry applied to the $U(1)^{k-1}$ gauge sector in the deformed theory but after switching off the deformation, the gauge group becomes $U(k)$ and a mirror dual is not known.\\
{\Huge Sistemare qui, aggiungi che è caso limite di dualità note, abeliani duali a singoletti?}
\\
We assume that this enhancement to $U(k)$ in the undeformed theory does not modify the results we obtained. 
We will show an independent method to reduce the duality through the partition function which does not rely on the deformation of the superpotential.
This method will yield the same result we found using this assumption.
\\
In order to obtain Kim-Park duality, we will need to switch off the deformation on both sides of the duality.\\
In the electric side, the gauge group recovers the original $SU(N_c)$ and we find the electric theory of Kim-Park duality, without $\eta$ superpotential.
The charges of the fields are given by
\begin{equation}
\begin{array}{| c | c c c c c c | }
\hline
\hbox{Fields} & SU(N_f)_L & SU(N_f)_R & U(1)_B & U(1)_A & U(1)_J & U(1)_R  \\
\hline
Q & N_f & 0  & N_c & 1 & 0  & \Delta_Q \\
\tilde{Q}   & 0  &\overbar{N_f} &  - N_c & 1  & 0 & \Delta_Q \\
X & 0 & 0 & 0 & 0 &0 & \Delta_X \\
\hline
\end{array}
\end{equation}
In the magnetic side, we recover $SU(k N_f - N_c) \times U(1) \times U(1)_{mirror} \simeq U(k N_f - N_c) \times U(1)_{mirror}$ with the following massless fields
\begin{itemize}
\item $U(k N_f - N_c)$ quarks in the (anti)fundamental representation $(q,\tilde{q}$
\item $U(k N_f - N_c)$ adjoint matter $Y$
\item $U(k N_f - N_c)$ singlets $M_j$ with $N_f \times N_f$ flavours indices.
% Goes away with mirror symmetry
%\item $U(k N_f - N_c)$ and flavour singlet $\tilde{M}_j$, originating from the last component of $M_j$
\item $U(1) $ quarks with $k$ flavours $(b,\tilde{b})$, obtained by mirror symmetry 
\end{itemize}
The quantum numbers of these fields are obtained by the matching of the mesons and baryons.\\
%In the magnetic theory the baryons contain the Coulomb branch coordinate of the mirror $U(1)$ theory $V_-^{mirror}$ since it is identified with the quarks of the mirror sector \cite{Nii:2014jsa}.\\
The charges of the fields are the following
\begin{equation}
\centering
\begin{array}{|c |c c c c c c |}
\hline
 & SU(N_f)_L & SU(N_f)_R & U(1)_B & U(1)_A & U(1)_J & U(1)_R  \\
\hline
q & \overbar{N_f} & 0  & 
0 %\frac{N_c}{k N_f - N_c}
  & -1  & 0  & \Delta_{Y} - \Delta_Q \\
\tilde{q}   & 0  &N_f &
 0 % -\frac{N_c}{k N_f - N_c}
 & - 1  & 0 &\Delta_{Y} - \Delta_Q \\
Y & 0 & 0 & 0 & 0 &0 & \Delta_{Y} \\
M_j  & N_f & \overbar{N_f} & 0 & 2 & 0 & 2 \Delta_Q + j \Delta_{Y}\\
b_j & 0  & 0 & 0  &   -N_f &  
1 %\frac{1}{2}
   & N_f ( 1 - \Delta_Q) + \Delta_{Y} (- N_c+ j +1  )  \\
 \tilde{b} _j & 0  & 0 & 0  &   - N_f &  
 -1%- \frac{1}{2} 
   & N_f ( 1 - \Delta_Q) + \Delta_{Y} (- N_c + j +1 )  \\
 \tilde{v}_{j,\pm} & 1 & 1 & 0 & N_f & \pm 1 & -(1 - \Delta_Q) N_f + \Delta_{Y} (N_c+1 + j)\\
\hline
\end{array}
%\caption{Tabella delle cariche per il primo caso}
\end{equation}
where $\tilde{v}_{i,\pm}$ is one of the two Coulomb branch coordinates of the $U(k N_f - N_c)$ gauge group.\\
Note that the fields $(b_i,\tilde{b}_i)$ are charged under the topological symmetry and have the right quantum numbers to couple with the monopole operators $\tilde{v}_{i,\pm}$.\\
The baryonic symmetry can mix with the $U(1)$ factor in the $U(k N_f -N_c)$ gauge group and as a result its definition is ambiguous.
We chose to remove this ambiguity by leaving all the fields uncharged under it.
Another possibility is to give opposite charges to the quarks $(q,\tilde{q})$.\\
The $\eta$ superpotential can be rewritten using monopole operators at high energy through the identification \cite{Nii:2014jsa}
\begin{equation}
 \tilde{V}_{j, k-1-j} = b_{j} \tilde{v}_{k-1-j,-}
\end{equation}
As in the previous cases, there is a Affleck-Harvey-Witten superpotential
associated to the gauge symmetry breaking $SU(k (N_f +1) - N_c) \rightarrow U(k N_f - N_c) \times U(1)^{mirror}$ given by
\begin{equation}
W_{AHW} = \sum_{j=0}^{k-1} \tilde{b}_{j+1} \tilde{v}_{k-1-j,+}
\end{equation}
Adding all these terms we obtain the superpotential of the theory which reads
\begin{equation}
W_{mag} = \Tr{Y^{k+1}} + \sum_{j=0}^{k-1} M_j \, \tilde{q} Y^{k-1-j} q + \sum_{j=0}^{k-1} \left( b_{j} \tilde{v}_{k-j-1,-} + \tilde{b}_{j} \tilde{v}_{k-j-1,+}    \right) 
\end{equation}
which is consistent with what was found with ungauging technique from the $U(N_c)$ duality in \cite{Park:2013wta}.


































