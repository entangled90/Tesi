\documentclass[a4paper,12pt]{article}

\usepackage{amsmath}
\usepackage{mathtools}
\usepackage{amstext}
\usepackage{amssymb}
\usepackage{amsthm}
%\usepackage{fullpage}
%%\usepackage{mathrsfs}
\usepackage[english]{babel}
\usepackage[utf8]{inputenc}
\usepackage[T1]{fontenc}
%%\usepackage{natbib}
\usepackage{lmodern}
\usepackage[toc,page]{appendix}
\usepackage[a4paper]{geometry}
%\usepackage[nottoc,numbib]{tocbibind}
\geometry{a4paper,top=1.5cm,bottom=1.5cm,left=2cm,right=2cm,%
heightrounded,bindingoffset=5mm}





%%%%%%%%%%% DA MODIFICARE!!!!!
%%%%%%%%%%%%%%%%%
%%%%%%%%%%%%%%


%%%%%%%%%%%%%%%%%%%OOOOOO
\author{
 Carlo Sana  ~--~ \emph{matricola} 726409 \\
 Laurea Magistrale in Fisica Teorica -- 29 June 2015\\
%  cell 3295795286 \\
\emph{Supervisor}: Silvia Penati 
~
\emph{Assistant supervisor}: Alberto Zaffaroni\\
}
\date{}
\title{ \textbf{4D to 3D reduction of Seiberg duality for $SU(N)$ susy gauge theories with adjoint matter: a partition function approach }}	
 


\pagenumbering{gobble}

\begin{document}
\maketitle
In each branch of physics strongly coupled systems represents a challenge because the lack of approximation methods available makes their analysis very difficult.\\
In the last twenty years we assisted to the discovery of many dualities in high energy physics between strongly coupled theories with weakly coupled ones, which can be studied with perturbative methods.
AdS/CFT  and Seiberg duality are the strong-weak dualities that gained the most attention because of their broad range of applicability.
Seiberg duality is also called electric-magnetic duality because it shares some features with the duality between electric and magnetic fields in the Maxwell's equations discovered by Dirac.
Generally, Seiberg duality associates to an \emph{electric theory}, which is a non-abelian supersymmetric gauge theory, a dual theory, the \emph{magnetic} one. 
The magnetic theory is a another gauge theory with a different number of colours with respect to the electric theory and contains mesons of the electric theory as fundamental particles instead of being composite by two quarks.
For a particular range of flavours and colours, both theories flow to a superconformal fixed point and describe the same physics, even though their dynamics at high energies is different.
Various example of this duality are known and they differ because of the gauge group or their matter content.

Seiberg duality was discovered for field theories in 4D in 1994 and only few years later an analogue duality was discovered in 3D.
However, there was no clue if dualities in different dimensions are related to each other.



\end{document}
