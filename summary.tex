\documentclass[a4paper,12pt]{article}

\usepackage{amsmath}
\usepackage{mathtools}
\usepackage{amstext}
\usepackage{amssymb}
\usepackage{amsthm}
%\usepackage{fullpage}
%%\usepackage{mathrsfs}
\usepackage[british]{babel}
\usepackage[utf8]{inputenc}
\usepackage[T1]{fontenc}
%%\usepackage{natbib}
\usepackage{lmodern}
\usepackage[toc,page]{appendix}
\usepackage[a4paper]{geometry}
\usepackage{microtype} %% font che esce a fine riga un pelo
%\usepackage[nottoc,numbib]{tocbibind}
\geometry{a4paper,top=1cm,bottom=1.5cm,left=2cm,right=2cm,%
heightrounded} %,bindingoffset=0mm}





%%%%%%%%%%% DA MODIFICARE!!!!!
%%%%%%%%%%%%%%%%%
%%%%%%%%%%%%%%


%
\date{}
\title{ \textbf{\boldmath{4D to 3D reduction of Seiberg duality for $SU(N)$ susy gauge theories with adjoint matter: a partition function approach }}	}
 \author{}


\pagenumbering{gobble}
%\date{29 June 2015}
\begin{document}
\maketitle
\vspace*{-2.3cm}
\begin{center}
\textit {29 Giugno 2015} \\
 \end{center}
\vspace{-0.3cm}
	 \textbf{Carlo Sana}  ~-~ 726409  \hfill
 LM in Theoretical Physics 
 %-29 Giugno 2015
\\
 \hspace{2cm} \textit {tel. - 3295795286}\\
\vspace{-0.5cm}
%  cell 3295795286 \\
\begin{center}
\textit{Supervisor}: 
\textsf{Silvia Penati} 
~~
\textit{Assistant supervisor}:
\textsf{Alberto Zaffaroni}

\end{center}



Strongly coupled systems are generally difficult to analyse from a theoretical point of view because approximation methods that can be used in this regime rarely exist.\\
%%%% ASSIST TO?
In the last twenty years we witnessed to the discovery of many dualities in high energy physics between theories at strong coupling and theories at weak coupling.
Thus, we can describe physical systems at strong coupling by using well-understood perturbative techinques in the weak coupling regime. 
AdS/CFT (Anti-deSitter/Conformal Field Theory) and Seiberg duality are the most studied examples of strong-weak duality because of their broad range of applicability.

Seiberg duality, also called electric-magnetic duality, extends to non abelian gauge theories some of the features of the duality between electric and magnetic fields in Maxwell's equations, originally discovered by Dirac in 1931.
Generally speaking, Seiberg duality is a correspondence between an \emph{electric theory}, which is a non-abelian supersymmetric gauge theory, and a dual magnetic theory. 
On the magnetic side, one has a gauge theory with a different number of colours  in which the mesons of the electric theory, which are built from two quarks, appear as fundamental particles.
For a particular range of flavours and colours, both theories flow to a superconformal fixed point and describe the same physics, even though their dynamics at high energies is different.
Various example of this duality are known and they differ because of the gauge group or their matter content.

Seiberg duality was discovered for four dimensional field theories in 1994 and only few years later an analogous duality was discovered in 3D.
However, there was no clue if dualities in 4D and 3D are related to each other.
Only recently it was discovered a method to dimensionally reduce dual theories from 4D to 3D in order to obtain 3D dualities from the 4D ones.\\
The reduction of the duality can be performed also directly on the partition function which can be calculated exactly from the superconformal index. 
The index is defined for theories in 4D with one dimension compactified into a circle and it is identical between dual theories even though their expression is different.
In the limit in which the radius of the circle shrinks to zero, the index reduces to the partition function of the theory reduced to 3D, which is subject to some additional constraints because of the reduction process.
It is possible to eliminate these constraints by performing a renormalization group flow directly on the partition function.
Applying this procedure to both theories, we obtain their partition functions without these constraints. 
They are equal to each other because of the matching between the indices in 4D.
Moreover, because the symmetries of the theory are explicit in the expression of the partition function, the charges of the field can be easiliy read from it.

The purpose of this thesis is to apply this reduction procedure to the Kutasov-Schwimmer-Seiberg duality, which differs from Seiberg duality because of the addition of a matter field in the adjoint representation of the gauge group.

The dimensional reduction of this duality was performed only with field theory techniques, which made use of some assumptions that are not completely justified.
Our work does not require these ipotesi and thus it is an indipendent check for these known results.
In the end, the identities between the partition functions for this duality corresponds to a non-trivial mathematical identity between integrals of hyperbolic gamma functions which is not yet demonstrated in mathematics.



\end{document}
