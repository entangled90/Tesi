\documentclass[a4paper,12pt]{article}

\usepackage{amsmath}
\usepackage{mathtools}
\usepackage{amstext}
\usepackage{amssymb}
\usepackage{amsthm}
%\usepackage{fullpage}
%%\usepackage{mathrsfs}
\usepackage[english]{babel}
\usepackage[utf8]{inputenc}
\usepackage[T1]{fontenc}
%%\usepackage{natbib}
\usepackage{lmodern}
\usepackage[toc,page]{appendix}
\usepackage[a4paper]{geometry}
%\usepackage[nottoc,numbib]{tocbibind}
\geometry{a4paper,top=1.5cm,bottom=1.5cm,left=2cm,right=2cm,%
heightrounded,bindingoffset=5mm}





%%%%%%%%%%% DA MODIFICARE!!!!!
%%%%%%%%%%%%%%%%%
%%%%%%%%%%%%%%


%%%%%%%%%%%%%%%%%%%OOOOOO
\author{
 Carlo Sana  ~--~ \emph{matricola} 726409 \\
 Laurea Magistrale in Fisica Teorica -- 29 June 2015\\
%  cell 3295795286 \\
\emph{Supervisor}: Silvia Penati 
~
\emph{Assistant supervisor}: Alberto Zaffaroni\\
}
\date{}
\title{ \textbf{4D to 3D reduction of Seiberg duality for $SU(N)$ susy gauge theories with adjoint matter: a partition function approach }}	
 


\pagenumbering{gobble}

\begin{document}
\maketitle
Strongly coupled systems represent a challenge in each branch of physics because the lack of approximation methods available makes their analysis difficult.\\
%%%% ASSIST TO?
In the last twenty years we assisted to the discovery of many dualities in high energy physics between theories at strong coupling and theories at weak coupling.
The latter can be studied with perturbative methods that can't be applied in the strong coupling regime.
AdS/CFT and Seiberg duality are the most studied examples of strong-weak duality because of their broad range of applicability.
Seiberg duality is also called electric-magnetic duality because it shares some features with the duality between electric and magnetic fields in Maxwell's equations originally discovered by Dirac in 1931.
In general, Seiberg duality associates to an \emph{electric theory}, which is a non-abelian supersymmetric gauge theory, a dual theory, the \emph{magnetic} one. 
The magnetic theory is a another gauge theory with a different number of colours with respect to the electric theory and contains the mesons of the electric theory as fundamental particles instead of being composite by two quarks.
For a particular range of flavours and colours, both theories flow to a superconformal fixed point and describe the same physics, even though their dynamics at high energies is different.
Various example of this duality are known and they differ between themselves because of the gauge group or their matter content.

Seiberg duality was discovered for field theories in 4D in 1994 and only few years later an analogue duality was discovered in 3D.
However, there was no clue if dualities in 4D and 4D are related to each other.
Only recently it was discovered a method to dimensionally reduce dualities from 4D to 3D preserving the duality between the two theories.\\
The reduction of the duality, other than in field theory, can be performed directly on the partition function which can be calculated exactly from the superconformal index. 
It is defined for theories in 4D with one dimension compactified into a circle and it is identical between dual theories even though their expression is different.
In the limit in which the radius of the circle shrinks to zero, it reduces to the partition function of the theory reduced to 3D, which is subject to some constraints caused by the reduction process.
It is possible to eliminate these constraints by performing a renormalization group flow directly on the partition function.
Applying this procedure to both theories, we obtain their partition functions without additional constraints. 
They are equal to each other because of the matching between the indices in 4D.
Moreover, the charges of the fields can be read off from the partition function, because it preserves explicitly the symmetries of the theory.

The purpose of this thesis is to apply this reduction procedure to the Kutasov-Schwimmer-Seiberg duality, which differs from Seiberg duality because of the addition of a matter field in the adjoint representation of the gauge group.
The dimensional reduction on the partition function was not present in literature for this duality and our work represents and independent check of the dimensional reduction performed in field theory .



\end{document}
