%!TEX root = tesi.tex


\chapter{Dimensional reduction of \\ Kutasov-Schwimmer duality}

The scope of this chapter is to provide an independent check of the results present in literature regarding Kim-Park duality for $SU(N_c)$ gauge group.\\
We will reduce Kutasov-Schwimmer duality in three dimensions first by calculating the superconformal index in four dimensions and then reducing it to the three dimensional partition function.
We can see by the charge assignement of the fields that the partition functions obtained with this method feature an $\eta$ superpotential, which breaks the symmetries that are anomalous in four dimensions.\\
Since we are interested in a truly three dimensional duality we will get rid of the $\eta$ superpotential in the same way we did in field theory i.e. by a suitable real mass assignment.\\
In this way, we can provide an independent and more solid proof for the results obtained in \cite{Nii:2014jsa}.


\section{Electric theory}
The electric theory was introduced in section \ref{sec:kutasov_duality4d} and we report the charge table in order to fix our notation.
\begin{equation}
\begin{array}{| c | c | c c c l | }
\hline
 & SU(N_c) &SU(N_f)_L  &SU(N_f)_R   & U(1)_B &  U(1)_R \\
\hline
Q & N_c&N_F & 1   &  1  & R_Q =  1 - 
R_X %\frac{2}{k+1}
 \frac{N_c}{N_f}  \\
\tilde{Q} & \overbar{N_c} &  1 & \overbar{ N_F}   & -1   &   R_Q = 1 -
 R_X %\frac{2}{k+1}
  \frac{N_c}{N_f}    \\
X & N_c^2-1 & 1   & 1    & 0   &  R_X = \frac{2}{k+1} \\
\hline
\end{array}
\centering
%\caption{Charges for the electric theory}
%\label{table:charge_table_el_ks_4d}
\end{equation}
\subsection{Calculation of the index}
The expression for the superconformal index on \emph{single particle states} can be easily calculated with equations \eqref{eqn:superconforma_index_chiral} and \eqref{eqn:superconformal_index_vector}
\begin{equation}
% r = R_Q s = R_X / 2
\begin{aligned}
 & i_E (p,q,v, y,\tilde y ,z) = \\ 
 -& \left({p \over 1-p}+{q \over 1-q} -{1 \over (1-p)(1-q)}
\left((p q)^{\frac{R_X}{2}}- (p q)^{1-\frac{R_X}{2}}\right)
\right) \left( p_{N_c}( z)\, p_{N_c}(z^{-1})-1\right ) \\ 
&
\begin{aligned}
+&
{1\over (1-p)(1-q)}
\bigg(
&&(p q)^{{1\over 2}R_Q} \, v \, p_{N_f}(y)\, p_{N_c}(z)
- (p q)^{1- {1\over 2}R_Q} \, {1 \over v}\, p_{{N_f}}(y^{-1})\, p_{{N_c}}(z^{-1}) \\
%\qquad\qquad\qquad\qquad
&
 && +
(p q)^{{1\over 2}R_Q}\, {1\over v}\, p_{{N_f}}({\tilde y}\,) \, p_{{N_c}}(z^{-1})
- (p q)^{1- {1\over 2}R_Q} \, v\, p_{N_f}({\tilde y}^{-1})\, p_{N_c}(z)\bigg) \\
\end{aligned}
\end{aligned}\label{eqn:superconf_index_electric}
\end{equation}
where $ p,q$ are defined as in the previous chapters and the other fugacities are associated to the other symmetries by
\begin{equation}
v \rightarrow U(1)_B \quad y \rightarrow SU(N_F)_L \quad \tilde{y} \rightarrow SU(N_F)_R \quad z \rightarrow SU(N_c) 
\end{equation}
In \eqref{eqn:superconf_index_electric} we have written explicitly the character for the adjoint and the (anti) fundamental representations of $SU(N)$ which are given by \
\begin{gather}
%\begin{aligned}
 \chi_{adj}(z) = p_{N}( z)\, p_{N}(z^{-1})-1 \quad \chi_{N}(y) = p_{N}(y) \quad \chi_{\overbar{N}}(\tilde{y}) = p_{N}(\tilde{y}^{-1}) \\
 p_{N}(x) = \sum_{i =1}^{N} x_i \qquad p_{N}(x) = \sum_{i =1}^{N} \frac{1}{x_i}
%\end{aligned}
\end{gather}
Moreover, we have the following constraints on the fugacities, since they are in the Cartan subgroup of $SU(N_c)$ or $SU(N_f)$
\begin{equation}
\prod_i^{N_c} z_i = 1 \qquad \prod_i^{N_f} y_i = 1 \qquad \prod_i^{N_f} \tilde{y}_i = 1   
\end{equation}
If we use the definition of the polynomials $p_N(x)$ we obtain
\begin{equation}
\begin{aligned}
 & i_E (p,q,v, y,\tilde y ,z) = \\ 
 & 
  -\left({p \over 1-p}+{q \over 1-q} -{1 \over (1-p)(1-q)}
\left((p q)^{\frac{R_X}{2}}- (p q)^{1-\frac{R_X}{2}}\right)
\right) \left( \sum_{1\leq i,j \leq N_c} \frac{z_i}{z_j}-1\right ) \\ 
&
\begin{aligned}
&
+{1\over (1-p)(1-q)} \sum_{a=1}^{N_f} \sum_{j=1}^{N_c}
\bigg(
&&(p q)^{{1\over 2}R_Q} \, v \, y_a \, z_j
- (p q)^{1- {1\over 2} R_Q} \, v^{-1} \, y_a^{-1} \, z_j^{-1} \\
%\qquad\qquad\qquad\qquad
&
 && +
(p q)^{{1\over 2}R_Q}\, v^{-1}\, {\tilde y}_a \,z_j^{-1}
- (p q)^{1- {1\over 2}R_Q} \, v\, {\tilde y_a}^{-1}\, z_j\bigg) \\
\end{aligned}
\end{aligned}
\end{equation}
The index is given by the sum of contributions for every field in the theory
\begin{equation}
i_E(p,q,v,y,\tilde{y},z) = i_E^V(p,q,z) + i_E^{X} (p,q,z) + i_E^{Q,\tilde{Q}}(p,q,v,y,z)
\end{equation}
and the complete index is given by the Plethystic exponential
\begin{equation}
\begin{aligned}
I_E(p,q,v,y,\tilde{y}) = & \int_{SU(N_c)} d \mu(z) \; \exp \left( \sum_{n=1}^{\infty} \frac{1}{n} i_E(p^n,q^n,v^n,y^n,\tilde{y}^n,z^n) \right) \\
= & \int_{SU(N_c)} d \mu (z) \prod_{fields \; F } I_E^F(p,q,v,y,\tilde{y},z) 
\end{aligned}
\end{equation}
where $I_E^F$ is given by the Plethystic exponential for the field F.\\
Since the fugacity $z$ is in the maximal torus of the gauge group, we can restrict the integral to $T^{N_c -1}$.
\begin{equation}
  \int_{SU(N_c)}\, d \mu(z) \, f(z) = {1 \over N_c!} \int_{T^{N_c-1}} \prod_{i=1}^{N_c }
  { d z_i \over { 2 \pi i z_i} } \Delta (z) \Delta (z^{-1}) f(z) \bigg \rvert_{\prod_{i=1}^{N_c} z_i = 1}
 \end{equation}
 where $\Delta (z) $ is the Vandermonde determinant:
 $$
 \Delta (z ) \, = \, \prod_{ \overset{1 \leq i,j< \leq N_c} { i \neq j}}^{N_c} ( z_i - z_j) \, = \,  \prod_{ \overset{1 \leq i,j< \leq N_c} { i \neq j}}^{N_c}\bigg ( 1 - {z_i \over  z_j } \bigg) \, z_j = \,  \prod_{ \overset{1 \leq i,j< \leq N_c} { i \neq j}}^{N_c}\bigg ( 1 - {z_i \over  z_j } \bigg)
 $$
The last equality holds since $ \prod_{i=1}^{N_c} z_i = 1$.
 
We will calculate the superconformal index by considering separately the contributions from the various fields.
The complete calculations for every field can be found in appendix \ref{appendix:index_electric}.\\
\paragraph{Vector field}
The vector field contributes to the index with
\begin{equation}
\begin{aligned} 
I_E^V(p,q,z) = &\exp \bigg( \sum_{n=1}^{\infty} {1 \over n} i_E^{Vett} (p^n,q^n,z^n) \bigg) = \\
 &\exp \bigg( \sum_{n=1}^{\infty} - {1 \over n} \, \bigg( {p^n \over 1-p^n} + {q^n \over 1-q^n} \bigg)  \bigg(\bigg( \sum_{ 1 \leq i,j \leq N_c}  {z_i^n \over z_j^n}  \bigg)- 1 \bigg) \bigg) \, = \\
 = & (p;p)^{N_c-1} ( q;q)^{N_c-1}  \prod_{ 1 \leq i < j \leq N_c} \frac{ 1 }{ \big( 1 -{ z_i \over z_j} \big) \big ( 1 - { z_j \over z_i} \big) \Gamma_e( {z_i \over z_j};p,q) \Gamma_e({z_j \over z_i}; p,q)  } \\
 =& (p;p)^{N_c-1} ( q;q)^{N_c-1}  \prod_{ 1 \leq i < j \leq N_c}  \frac{1}{ \Delta(z) \Delta (z^{-1})} \,  \frac{ 1 }{ \Gamma_e( {z_i \over z_j};p,q) \Gamma_e({z_j \over z_i}; p,q)}
 \end{aligned}
 \label{eqn:sci_electric_vector}
 \end{equation}
 Note there is a term that cancels the Vandermonde determinant coming from the integration measure. This feature is not exclusive to $SU(N)$ gauge group but happens also for $SP(2N) $ and $SO(N)$ groups.\\
\paragraph{Adjoint matter}
The single particle index for this field reads
\begin{equation}
	i_E^{X}(p,q,z) = {1 \over (1-p)(1-q)}
	\left((p\,q)^{\frac{R_X}{2}}- (p\,q)^{1-\frac{R_X}{2}}\right) \left( \left(  \sum_{ 1 \leq i,j \leq N_c}  {z_i\over z_j}  \right)- 1 \right)
\end{equation}
which results in
\begin{equation}
 I_E^X (p,q,z) = \Gamma_e \left ( (pq)^\frac{R_X}{2};p,q \right)^{N_c-1} \displaystyle \prod_{1\leq i<j\leq N_c} \Gamma_e \left( (pq)^\frac{R_X}{2} \,  {z_i \over z_j} \right) \Gamma_e \left( (pq)^\frac{R_X}{2} \,  {z_j \over z_i}\right)
\end{equation}

\paragraph{Quarks}
The quarks in the (anti) fundamental representation have a single particle index of the form
\begin{multline}
i_E^{Q,\tilde{Q}} (p,q,v,y,\tilde{y})=\\ 
{1\over (1-p)(1-q)} \sum_{j=1}^{N_c}
\bigg(  \sum_{a=1}^{N_f} \left(  
(p q)^{{1\over 2}R_Q} \, v \, y_a \, z_j
- (p q)^{1- {1\over 2} R_Q} \, v^{-1} \, y_a^{-1} \, z_j^{-1} \right)\\
 +
 \sum_{b=1}^{N_f} \left(  
(p q)^{{1\over 2}R_Q}\, v^{-1}\, {\tilde y}_b \,z_j^{-1}
- (p q)^{1- {1\over 2}R_Q} \, v\, {\tilde y_b}^{-1}\, z_j \right)\bigg) 
\end{multline}
where $Q$ and $\tilde{Q} $ generate the first and second line of the index, respectively.\\
Their contribution can be written as
\begin{equation}
I_E^{Q,\tilde{Q}} ( p ,q,v,y,\tilde{y},z) \, = \, \prod_{ j}^{ N_c} \prod_{ a,b}^{N_f} \Gamma_e \left( (pq)^{R_Q \over 2} \, v  \,  y_a \,  z_j \right)
\Gamma_e \left( (pq)^{R_Q \over 2}\,  v^{-1} \,  {\tilde y_b}^{-1}\,  z_j^{-1}\right)
\end{equation}
\subsubsection{Expression of the superconformal index }
The formula for the superconformal index can be obtained by multiplying all the contributions calculated above and reads
\begin{multline}
I_{E} ( p,q,v,y,\tilde y ) = 
  \frac{ (p;p)^{N_c-1} ( q;q)^{N_c-1} }{ N_c !} \, \Gamma_e \left(   (pq)^\frac{R_X}{2};p,q\right)^{N_c-1} \\
\oint_{T^{N_c-1}} \left( \prod_{i=1}^{N_c } { d z_i \over { 2 \pi i z_i} } \right) \delta \left( \prod_{i=1}^{N_c} z_i - 1 \right)
\prod_{ 1 \leq i < j \leq N_c} \frac{ \Gamma_e \left( (pq)^\frac{R_X}{2} {z_i \over z_j} \right) \Gamma_e \left( (pq)^\frac{R_X}{2} {z_j \over z_i} \right) }{ \Gamma_e \left( {z_i \over z_j};p,q \right) \Gamma_e \left({z_j \over z_i}; p,q \right)} \\
 \prod_{ j}^{  N_c} \prod_{ a,b}^{ N_f} \Gamma_e \left( (pq)^{R_Q \over 2} \, v \,  y_a \,  z_j \right)
\Gamma_e \left( (pq)^{R_Q \over 2} v^{-1} \, \tilde {y}_b^{-1}\,  z_j^{-1}\right)
\end{multline}

{\Huge PARLA DELLE BALANCING CONDITIONS!}
\subsection{Reduction of the superconformal index to the partition function}
We can now proceed to reduce the index to the partition function on the squashed three sphere by appliying the procedure we introduced in the previous chapter. 
We need to parametrize the fugacities as
\begin{gather}
p = e^{ 2 \pi i r  \omega_1 } \, \, \,  q = e^{ 2 \pi i r \omega_2 } \,  \, \, z_i = e^{ 2 \pi i r \sigma_i } \\
\, \, y_a = e^{ 2 \pi i r m_a } \, \,  \,
y_a = e^{ 2 \pi i r  {\tilde m_a} } \, \, \,  v = e^{ 2 \pi i r m_B}
\label{eqn:fugacities_redefined_el}
\end{gather}
and we have the following conditions on the real masses
\begin{equation}
\sum_a m_a =0 \qquad \sum_b m_b = 0 \qquad \sum_j \sigma_j = 0
\end{equation}
By using the identity \eqref{eqn:elliptic_to_hyperbolic_vdbult} on every term we can obtain the partition function as an integral of products of hyperbolic gamma functions.
Explicit calculation can be found in appendix \ref{appendix:reduction_index_to_pf}.\\
We will use the following notation for the hyperbolic gamma function
\begin{equation}
\Gamma_h (z ; \omega_1 , \omega_2) = \Gamma_h(z) \qquad \Gamma_h ( u \pm z) = \Gamma_h( u + z) \Gamma_h( u - z)
\end{equation}
From now on we will denote the three dimensional R-charges of quarks as $\Delta_Q$ and as $\Delta_X$ for the adjoint matter since they are gonna be different from the four dimensional values, which are fixed by the R-symmetry anomaly.

\paragraph{Vector field}

For the vector field in the adjoint representation we can use the identity \eqref{eqn:appendix_sl2z_theta} applied to $SU(N_c)$ gauge group
\begin{multline}
\lim_{r \rightarrow 0} \frac{ (p;p)^{N_c- 1}(q;q)^{N_c- 1} }{ N_c ! }
\oint_{T^{N_c -1}} \prod_{j=1}^{N_c -1 } \frac{ d z_j}{2 \pi i z_j} \prod_{1 \leq i<j \leq N_c } \frac{1}{\Gamma_e( \frac{z_i}{z_j} )\Gamma_e( \frac{z_j}{z_i} )} \sim
\\
\sim 
\exp \left(  
- \frac{ i \pi \omega (N_c^2 -1)}{6 r \omega_1 \omega_2}
\right)
 \frac{1}{N_c !} \int \prod_{j=1}^{N_c -1 } \frac{d \sigma_j}{\sqrt{- \omega_1 \omega_2} } \prod_{1 \leq i<j \leq N_c }\frac{1}{\Gamma_h( \pm (\sigma_i - \sigma_j) }
\end{multline}

\paragraph{Adjoint matter}

\begin{multline}
 \Gamma_e \left((pq)^{\frac{\Delta_X}{2}} \right)^{N_c -1} 
 \prod_{1 \leq i <j \leq N_c} 
\Gamma_e \left( (pq)^{\frac{\Delta_X}{2}} \left( { z_i \over z_j} \right) \right) \Gamma_e \left( (pq)^{\frac{\Delta_X}{2}} \left( { z_j \over z_i} \right) \right) \sim \\
 \exp \left(- \frac{i \pi} {6 r \omega_1 \omega_2 }  \left( (N_c^2-1) \omega (\Delta_X - 1)  \right) \right)   \prod_{1 \leq i <j \leq N_c} 
  \, \Gamma_h \left( \Delta_X \omega \pm (\sigma_i - \sigma_j) \right) 
 %\Gamma_h \left( \Delta_X \omega + (\sigma_j - \sigma_i) \right)
\end{multline}


\paragraph{Quarks}

\begin{multline}
\prod_{a,b}^{N_f} \prod_{j}^{N_c}
\Gamma_e\left(
(pq)^{\Delta_Q v y_a z_j }\right) 
\Gamma_e\left(
(pq)^{\Delta_Q v^{-1} \tilde{y}_b^{-1} z_j^{-1} } \right)  = \\
=\, \exp \left( {- \frac{i \pi} {6 r \omega_1 \omega_2 }  ( 2 \omega N_c N_f  (R_Q - 1) + N_c \sum_a m_a - N_c \sum_b \tilde {m_b} )} \right) \times  \, 
\\
\times
\prod_{a,b}^{N_f} \prod_{j}^{N_c}
 \, \Gamma_h ( \mu_a + \sigma_j) \Gamma_h ( \nu_b- \sigma_j)
\end{multline}
Where we defined the real masses $\mu_a,\nu_b$
\begin{equation}
 \mu_a = \omega \Delta_Q + m_a + m_B  \quad  \nu_b = \omega \Delta_Q - \tilde m_b - m_B
\end{equation}

\subsubsection{Divergent contribution}
Summing up all the exponential terms from the previous limits we obtain
\begin{multline}
 \exp 
 \biggl[
 -\frac{i \pi\,  } {6 r \omega_1 \omega_2 }   
 \biggl(
 \omega \biggl(   (N_c^2 -1) +  (N_c^2 - 1) (\Delta_X -1) + \\ 
   + 2  N_c N_f  (R_Q - 1) \biggr) 
    + N_c \sum_a m_a - N_c \sum_b \tilde {m_b} \biggr)
\biggr]
\end{multline}
which is compatible with the following general formula \cite{Aharony:2013dha}
\begin{equation}
 \exp 
 \biggl[
 -\frac{i \pi\,  \omega} {6 r \omega_1 \omega_2 }   
 \biggl(
 \omega \biggl(
  |G| + \sum_{\alpha}(R_{ \alpha} -1 ) 
  \biggr)
   + \sum_a m_a \sum_{\alpha} e^{(\alpha)_a}
 \biggr)
\biggl]
\end{equation}
which is proportional to the gravitational anomaly \cite{Aharony:2013dha}
which is identical between the electric and the magnetic theory.

\subsubsection{Partition function}
We can write down the formula for the partition function of the electric theory by putting al the pieces together and ignoring the divergent prefactor calculated previously. 
The partition function is then
\begin{multline}
Z_{el} ( \mu_a , \nu_b ) \; = 
\; {1 \over N_c ! } \;
\Gamma_h ( \Delta_X \omega ; \omega_1 , \omega_2)^{N_c-1}
\int
\prod_{i=1}^{N_c} \frac{d \sigma_i}{\sqrt{- \omega_1 \omega_2}} \; \delta( \sum_i \sigma_i)  \\
 \prod_{ 1 \leq i<j \leq N_c} \frac{ \Gamma_h( \Delta_X \omega \pm (\sigma_i - \sigma_j)) }{ \Gamma_h ( \pm (\sigma_i - \sigma_j) )}
 \prod_{a,b}^{N_f} \prod_{j=1}^{N_c} \Gamma_h ( \mu_a + \sigma_j) \Gamma_h ( \nu_b - \sigma_j)
\end{multline}
with 
\begin{equation}
 \mu_a = \omega \Delta_Q + m_a + m_B  \quad  \nu_b = \omega \Delta_Q - \tilde m_b - m_B
\end{equation}
Since we obtained the partition function by compactifying on $S^1$ we have the following condition on the real masses which is enforced by the $\eta$ superpotential
\begin{equation}
  { 1 \over 2 }\left(  \sum_a \mu_a + \nu_a  \right) = \omega \left( - N_c + N_c (1 - \Delta_X )  + N_f  \right)
  \label{eqn:R-symm Anomaly}
\end{equation}
Note that this condition is equivalent to require that the R-symmetry in four dimension is anomaly free. 
As a result the R-charges of the fields for the theory with $\eta$ superpotential are identical to the four dimensional R-charges, as we noted in the field theory side of the duality in the previous chapters.\\




\subsection{Flow to a duality withouth $\eta$ superpotential}

In order to obtain the duality without $\eta$ superpotential we start with $N_f + 1 $ flavours and we assign real masses in the same way as \eqref{eqn:real_mass_reduction_el}. \\
The real mass assignment breaks the gauge group $ SU(N_f+1)^2 \times U(1)_B \, \rightarrow \, SU(N_f)_L \times SU(N_f)_R \times U(1)_A \times U(1)_B$. \\
We will assign a large value to the mass $m$. 
The real masses are given by
\begin{equation}
\begin{aligned}
\mu_a = 
  \begin{cases}
    \omega \, \Delta_Q + m_a + m_A + m_B \qquad  &a = 1, \dotsc, N_f\\ 
    \omega \, \Delta_M +  m - m_A N_f  + m_B \qquad  &a = N_f+1\\ 
  \end{cases}
  \\
\nu_b = 
  \begin{cases}
    \omega \, \Delta_Q + \tilde{m}_b + m_A - m_B \qquad  &b = 1, \dotsc, N_f\\ 
    \omega \, \Delta_M - m - m_A N_f  - m_B \qquad  &b = N_f+1\\ 
  \end{cases}
\end{aligned}
\end{equation}
The R-charge of the last flavour is different from the other quarks because the axial symmetry $U(1)_A$ mixes with R-symmetry.\\
With this real mass assignment we can now perform the limit $m \rightarrow \infty$ which can be done on the partition function using the following mathematical identity for the hyperbolic gamma function \cite{vanDeBult:2007}
\begin{equation}
% \lim_{ m \rightarrow \infty } \Gamma_h ( \omega \Delta + \sigma_i + M + m) = \\
% \exp \left[ \mbox{sign} (m) \; \frac{\pi i }{2 \omega_1 \omega_2} \left( [ \omega (\Delta
% -1) + \sigma_i + (m+M)]^2 - \frac{\omega_1^2 + \omega_2^2}{12} \right) \right]
 \lim_{ t \rightarrow \infty } \Gamma_h ( z + t; \omega_1, \omega_2) = 
\exp \left[ 
 \hbox{sign}(t) \; 
  \left(
    \left(
      z +t - \omega
    \right)^2 - \frac{\omega_1^2 + \omega_2^2}{12}
  \right)
\right]
\end{equation}
Applying this identity to the terms associated to the last flavour of quarks that have mass $m$ we obtain
\begin{equation}
\begin{aligned}
\Gamma_h (  \mu_{N_f+1}(m) + \sigma_i ) \, = &\, \exp \bigg( \, \mbox{sign}(m) \frac{\pi i}{2 \omega_1 \omega_2 } \big[ [ \omega (\Delta_M - 1) + \sigma_i + \\
+ & ( m + m_B - N_f m_A)]^2 - \frac{ \omega_1^2 + \omega_2^2 }{12} \big] \bigg )\\
\Gamma_h (  \nu_{N_f+1}(m) - \sigma_i  ) \, = &\, \exp \bigg( \, \mbox{sign}(-m) \frac{\pi i}{2 \omega_1 \omega_2 } \big[ [ \omega (\Delta_M - 1) - \sigma_i + \\
&( - m - m_B - N_f m_A)]^2 - \frac{ \omega_1^2 + \omega_2^2 }{12} \big] \bigg )\\
\end{aligned}
\end{equation}
Inserting these expressions in the partition function we obtain 
\begin{multline}
 \prod_{i=1}^{N_c} \exp \bigg[ \frac{\pi i}{2 \omega_1 \omega_2 } \bigg[ 4  ( m  + m_B ) ( \omega (\Delta_M - 1) -  m_A N_f ) +  4 \sigma_i  (\omega (\Delta_M - 1) -  m_A N_f)\bigg] \bigg] = \\
 \exp \bigg[ \frac{\pi i}{2 \omega_1 \omega_2 } \bigg[ 4 N_c  ( m  + m_B ) ( \omega (\Delta_M - 1) -  m_A N_f ) 
 +4\big( \sum_{i=1}^{N_c}  \sigma_i  \big) \big(\omega (\Delta_M - 1) -  m_A N_f \big)  \bigg] \bigg]
\end{multline}
We can define the $ c(x) = e^{ \frac{i \pi x }{2 \omega_1 \omega_2}}$ in order to ease the notation.
The exponential factor then becomes
\begin{equation}
c( 4 N_c  ( m  + m_B ) ( \omega (\Delta_M - 1) -  m_A N_f ) \;  c ( 4\big( \sum_{i=1}^{N_c}  \sigma_i  \big) \big (\omega (\Delta_M - 1) -  m_A N_f \big)
\end{equation}
We can now use the condition on the real masses that is generated by the $\eta$ superpotential \eqref{eqn:R-symm Anomaly} which reads
\begin{equation}
  { 1 \over 2 } \sum_a \mu_a + \nu_a  = \omega ( -N_c \Delta_X + N_f + 1 )  \, =\omega( \, N_f \Delta_Q +  \Delta_M )
\end{equation}
Using this condition we can the R-charge of last flavour $\Delta_M$ as a function of the real masses of the light quarks and the exponential factor reads
\begin{multline}
c \left( 4 N_c  ( m  + m_B ) ( \omega ( N_f (1 - \Delta_Q)  - N_c \Delta_X ) -  m_A N_f )\right)\,  \\
c\left( 4\big( \sum_{i=1}^{N_c}   \sigma_i  \big) \big ( \omega( N_f (1 - \Delta_Q)  - N_c \Delta_X)  -  m_A N_f ) \big) \right)
\end{multline}
The term on the first line can be taken outside the integral and we will found an identical term in the magnetic side.
The second term will be set to one by the delta function $\delta(\sum_i \sigma_i)$ upon integration.\\
The partition function for the electric theory after integrating out the $N_f +1$-th flavour is then given by
\begin{multline}
%\begin{aligned}
 Z_{el} ( \mu_i , \nu_i )\,  =  \\
 %\frac{1}{ (2 \ \pi i)^{N_c} }
 c \left( 4 N_c  ( m  + m_B ) (-  m_A N_f   + \omega ( N_f (1 - \Delta)  - N_c \Delta_X ) ) \right) 
 {1 \over N_c ! }
\Gamma_h ( \Delta_X \omega )^{N_c-1} 
\\ \int
\prod_{i=1}^{N_c} \frac{d \sigma_i}{\sqrt{- \omega_1 \omega_2}} \, \delta( \sum_i \sigma_i)  c \left( 4\big( \sum_{i=1}^{N_c}  \sigma_i  \big) \big( \omega( N_f (1 - \Delta)  - N_c \Delta_X)  -  m_A N_f \big) \right) 
\\
   \prod_{ 1 \leq i<j \leq N_c} \frac{ \Gamma_h( \Delta_X \omega \pm (\sigma_i - \sigma_j)) }{ \Gamma_h ( \pm (\sigma_i - \sigma_j) )}
 \prod_{a,b=1}^{N_f} \prod_{j=1}^{N_c} \Gamma_h ( m_a + m_B + m_A + \sigma_j) \Gamma_h ( -\tilde{m}_a -m_B + m_A - \sigma_j)
 %\end{aligned}
\end{multline}
Note that except for the exponential factors discuessed above, it corresponds to the partition function for a pure three-dimensional theory, withouth $\eta$ superpotential.\\
The charges can be easily read off from the partition function and are given by 
\begin{table}[h!]
 \begin{tabular}{|c |c |c |c |c |c |c |}
\hline
Fields & $SU(N_f)_L$ & $SU(N_f)_R$ & $U(1)_B$ & $U(1)_A$ & $U(1)_J$ & $U(1)_R $ \\
\hline
$Q$ & $N_f$ & 0  &$ N_c $& $1 $& 0  & $\Delta_Q$ \\
$\tilde{Q} $  & 0  &$\overline{N_f}$ & $ - N_c $& $1 $ & 0 & $\Delta_Q$ \\
$X$ & 0 & 0 & 0 & 0 &0 & $\Delta_X$ \\
\hline
\end{tabular}
\centering
\end{table}


\section{Magnetic theory}
The four dimensional theory of Kutasov-Schwimmer duality was introduced in section \ref{sec:kutasov_duality4d}. 
It is a $SU(\tilde{N}_c) = SU( k N_f - N_c)$ SQCD theory with $N_f$ flavours $q,\tilde{q} $, a matter field $Y$ in the adjoint representation and $k$ mesons $M$ that are identified with the electric mesons. There is a superpotential for the adjoint field $ \Tr Y^{k+1}$ that results in $R_X = R_Y$\\
\begin{equation}
\begin{array}{|c |  c | c c c c |}
\hline
 & SU(\tilde{N_c})  & SU(N_f)_L  &SU(N_f)_R   & U(1)_B &  U(1)_R \\
\hline
q &  \tilde{N_c} & \overbar{N_F}  & 1   &   \frac{N_c}{k N_f - N_c }  &  1 - \frac{2}{k+1} \frac{ \tilde{N_c}}{N_f}  \\
\tilde{q} & \overbar{\tilde{N_c}}  & 1  &  N_F  & -\frac{N_c}{k N_f - N_c }   &   1 - \frac{2}{k+1} \frac{\tilde{N_c}}{N_f}   \\
Y & 1  &  \tilde{N_c}^2-1  & 1    & 0   &  \frac{2}{k+1} \\
 M_j & 1 & N_f &  \overbar{N_f} & 0 & 2 - \frac{4}{k+1} \frac{N_c}{N_f} + j \frac{2}{k+1} \\
 \hline
\end{array}
\end{equation}




\subsection{Superconformal index}
We will now repeat the process that was applied to the electric theory for the magnetic side.
The single particle index is given by
\begin{equation}
\begin{aligned}
 & i_M (p,q, \tilde v, y,\tilde y ,\tilde z) =  &&\\
& -\bigg({p \over 1-p}+{q \over 1-q} -{1 \over (1-p)(1-q)} \big((p q)^{\frac{R_Y}{2}}- (p q)^{1-\frac{R_Y}{2}}\big)
\bigg) \big( p_{\tilde N_c}( \tilde z)\, p_{\tilde N_c}(\tilde z^{-1})-1\big ) + && \\
&
\begin{aligned}
&+
{1\over (1-p)(1-q)} 
&&\bigg((p q)^{{1 \over 2 } R_q} \,\tilde v \, p_{N_f}(y^{-1})\, p_{\tilde N_c}(\tilde z)
- (p q)^{1-  {1\over 2}R_q} \, {1 \over \tilde v}\, p_{{N_f}}(y)\, p_{{\tilde N_c}}(\tilde z^{-1}) + \\
& && 
 + (p q)^{{1 \over 2 } R_q}\, {1\over \tilde v}\, p_{{N_f}}({\tilde y}\,) \, p_{{\tilde N_c}}(\tilde z^{-1})
- (p q)^{1-{1\over 2}R_q} \, \tilde v \, p_{N_f}({\tilde y}^{-1})\, p_{\tilde N_c}(\tilde z)\bigg) +  \\
& && \sum_{l=0}^{k-1}   \left( (pq)^{ R_Q + l \frac{R_Y}{2} } p_{N_f}(y) p_{N_f}(\tilde y^{-1}) - (pq)^{1 - ( R_Q + \frac{R_Y}{2} l )} p_{N_f}(y^{-1} p_{N_f}{\tilde y}\right) 
\end{aligned}
\end{aligned}
\end{equation}
where $R_q$ is the R-charge of the dual quark which is fixed in term of the charge $R_Q$ of the electric quark by requiring the validity of the duality.
The last line is the contribution from the  $k$ different mesons.\\
Using the explicit form of the polynomials $p_{N}(x)$ we have
\begin{equation}
\begin{aligned}
 i_M &(p,q, \tilde v, y,\tilde y ,\tilde z) = \\
& -\bigg({p \over 1-p}+{q \over 1-q} -{1 \over (1-p)(1-q)} \big((p\,q)^{\frac{R_Y}{2}}- (p\,q)^{1-\frac{R_Y}{2}}\big)
\bigg) \big( \sum_{i,j}^{\tilde N_c} \tilde z_i \tilde z_j^{-1} -1\big ) + \\
%% CHIRALS
&
\begin{aligned}
& +{1\over (1-p)(1-q)} && \bigg[ \sum_j^{\tilde N_c} \bigg( \sum_a^{N_f}\left(  (p\,q)^{{1 \over 2 } R_q} \,\tilde v \, y_a^{-1}\, \tilde z_j\right) 
-  (p\,q)^{1- {1\over 2}R_q} \, {1 \over \tilde v}\, y_a\,
\tilde z_j^{-1} + \\
&  &&+ \sum_b^{N_f} \left( (p\,q)^{{1 \over 2 } R_q}\, {1\over \tilde v}\,
({\tilde y_b}\,) \,
(\tilde z_j^{-1})
- (p\,q)^{1-  {1\over 2}R_q} \, \tilde v \,
{\tilde y_b}^{-1} \,
\tilde z_j \bigg) \right) + \\
%% MESONS
 & &&\sum_a^{N_f} \sum_b^{N_f} \sum_{l=0}^{k-1}   \bigg(  (pq)^{ R_Q + l \frac{R_Y}{2}   } y_a \tilde y_b^{-1}   - (pq)^{1 -( R_Q + l \frac{R_Y}{2}  )}
y_a^{-1}  {\tilde y_b}  \bigg) \bigg] 
\end{aligned}
\end{aligned}
\end{equation}
The index is now formally similar to the electric one except for the contribution of the mesons which can be easily calculatedby identifying the $(pq)^{R_q} + l \frac{R_Y}{2} y_a \tilde{y}_b^{-1}$ with the argument of the elliptic gamma function.
Their contribution reads
\begin{equation}
 \exp \bigg( \sum_{n}^{\infty} \frac{1}{n} i_M^{Meson} (p^n,q^n, \tilde v^n, y^n,\tilde {y}^n) \bigg) = 
\prod_a^{N_f} \prod_b^{N_f}  \prod_{l=0}^{k-1} \Gamma_e ( (pq)^{R_q + l \frac{R_Y}{2}} y_a \tilde{y}_b^{-1} ; p ,q)
\end{equation}
The complete expression for the superconformal index for the magnetic theory is given by
\begin{multline}
%\begin{aligned}
%&
I_{Mag} ( p,q,y,\tilde v,\tilde y) = \\
 %$& 
 { 1 \over \tilde{N_c} !} (p;p)^{\tilde{N_c}-1} ( q;q)^{\tilde{N_c}-1} \, 
 \Gamma_e( (pq)^\frac{R_Y}{2};p,q)^{\tilde{N_c}-1} \bigg( \prod_a^{N_f} \prod_b^{N_f}  \prod_{l=0}^{k-1} \Gamma_e ( (pq)^{R_Q + l \frac{R_Y}{2}} y_a \tilde{y_b}^{-1}; p ,q) \bigg) \\
%&
\int \bigg( \prod_{i=1}^{ \tilde{N_c }} { d z_i \over { 2 \pi i z_i} } \bigg) \delta \bigg( \prod_{i=1}^{\tilde{N_c}} z_i - 1 \bigg)
\prod_{ 1 \leq i < j \leq \tilde{N_c}} \frac{ \Gamma_e \big( (pq)^\frac{R_Y}{2} {z_i \over z_j} \big) \Gamma_e \big( (pq)^\frac{R_Y}{2} {z_j \over z_i} \big) }{ \Gamma_e( {z_i \over z_j};p,q) \Gamma_e({z_j \over z_i}; p,q)} \\
%&
  \qquad \prod_{ j}^{\tilde{N_c}} \prod_{ a,b}^{N_f} \Gamma_e ( (pq)^{{1 \over 2 } R_q} \tilde{v}\,  y_a^{-1} \tilde{z_j} ; p,q)
\; \Gamma_e ( (pq)^{{1 \over 2 } R_q} \tilde{v}^{-1} \tilde {y}_b \tilde{z_j}^{-1} ; p,q)
%\end{aligned}
\end{multline}

\subsection{Reduction to the partition function}
We reduce the index to the partition function using the same procedure employed for the electric theory.

\paragraph{Vector field}
We can use the identity \eqref{eqn:appendix_sl2z_theta} as in the electric theory with $\tilde{N}_c$ and $\tilde{\sigma}$ instead of $N_c$ and $\sigma$.  The result is
\begin{multline}
\lim_{r \rightarrow 0} \frac{ (p;p)^{\tilde{N}_c- 1}(q;q)^{\tilde{N}_c- 1} }{ \tilde{N}_c ! }
\oint_{T^{\tilde{N}_c -1}} \prod_{j=1}^{\tilde{N}_c -1 } \frac{ d \tilde{z}_j}{2 \pi i \tilde{z}_j} \prod_{1 \leq i<j \leq \tilde{N}_c } \frac{1}{\Gamma_e( \frac{\tilde{z}_i}{\tilde{z}_j} )\Gamma_e( \frac{\tilde{z}_j}{\tilde{z}_i} )} \sim
\\
\sim 
\exp \left(  
- \frac{ i \pi \omega (\tilde{N}_c^2 -1)}{6 r \omega_1 \omega_2}
\right)
 \frac{1}{\tilde{N}_c !} \int \prod_{j=1}^{\tilde{N}_c -1 } \frac{d \tilde{\sigma}_j}{\sqrt{- \omega_1 \omega_2} } \prod_{1 \leq i<j \leq \tilde{N}_c }\frac{1}{\Gamma_h( \pm (\tilde{\sigma}_i - \tilde{\sigma}_j) }
\end{multline}

\paragraph{Adjoint matter}
The reduction of the adjoint matter is identical to the electric theory (using \eqref{eqn:elliptic_to_hyperbolic_vdbult}) which results in
\begin{multline}
 \Gamma_e \left((pq)^{\frac{\Delta_Y}{2}} \right)^{\tilde{N}_c -1} 
 \prod_{1 \leq i <j \leq \tilde{N}_c} 
\Gamma_e \left( (pq)^{\frac{\Delta_Y}{2}} \left( { z_i \over z_j} \right) \right) \Gamma_e \left( (pq)^{\frac{\Delta_Y}{2}} \left( { z_j \over z_i} \right) \right) \sim \\
 \exp \left(- \frac{i \pi} {6 r \omega_1 \omega_2 }  \left( (\tilde{N}_c^2-1) \omega (\Delta_Y - 1)  \right) \right)   \prod_{1 \leq i <j \leq \tilde{N}_c} 
  \, \Gamma_h \left( \Delta_Y \omega \pm (\sigma_i - \sigma_j) \right) 
 %\Gamma_h \left( \Delta_X \omega + (\sigma_j - \sigma_i) \right)
\end{multline}

\paragraph{Quarks}
\begin{multline}
 \prod_{ j }^{\tilde{N_c}} \prod_{ a,b }^{N_f}  \Gamma_e ( (pq)^{{1 \over 2 } R_q} \, \tilde{v} \, y_i^{-1} \tilde{z_j} ; p,q)\Gamma_e ( (pq)^{{1 \over 2 } R_q} \, \tilde{v}^{-1} \tilde y_b \tilde{z_j}^{-1} ; p,q)  = \\
  =   \exp \bigg( \frac{- i \pi }{6 r \omega_1 \omega_2} \big( 2  N_f \tilde{N_c}\omega (R_q - 1 ) +  \tilde{N_c} \big( \sum_{a=1}^{N_f} - m_a + \tilde{m}_a \big)  \big) \bigg)  \Gamma_h \big(\tilde{ \mu}_a + \tilde{\sigma_j} \big) \Gamma_h \big( \tilde{\nu}_b - \tilde{\sigma_j} \big)
\end{multline}
where we defined the real masses $\tilde{\mu}_a,\tilde{\nu}_b$
\begin{equation}
\tilde{\mu}_a = - m_a + \tilde{m}_B + \omega (R_q) \qquad 
\tilde{\nu}_b = + \tilde{m}_b - \tilde{m}_B + \omega (R_q) \qquad 
\end{equation}



\paragraph{}




