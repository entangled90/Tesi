%!TEX root = tesi.tex


\chapter{Dimensional reduction of \\ Kutasov-Schwimmer duality}

The scope of this chapter is to provide an independent check of the results present in literature regarding Kim-Park duality for $SU(N_c)$ gauge group.\\
We will reduce Kutasov-Schwimmer duality in three dimensions first by calculating the superconformal index in four dimensions and then reducing it to the three dimensional partition function.
We can see by the charge assignement of the fields that the partition functions obtained with this method feature an $\eta$ superpotential, which breaks the symmetries that are anomalous in four dimensions.\\
Since we are interested in a truly three dimensional duality we will get rid of the $\eta$ superpotential in the same way we did in field theory i.e. by a suitable real mass assignment.\\
In this way, we can provide an independent and more solid proof for the results obtained in \cite{Nii:2014jsa}.


\section{Electric theory}
The electric theory was introduced in section \ref{sec:kutasov_duality4d} and we report the charge table in order to fix our notation.
\begin{equation}
\begin{array}{| c | c | c c c l | }
\hline
 & SU(N_c) &SU(N_f)_L  &SU(N_f)_R   & U(1)_B &  U(1)_R \\
\hline
Q & N_c&N_F & 1   &  1  & R_Q =  1 - 
R_X %\frac{2}{k+1}
 \frac{N_c}{N_f}  \\
\tilde{Q} & \overbar{N_c} &  1 & \overbar{ N_F}   & -1   &   R_Q = 1 -
 R_X %\frac{2}{k+1}
  \frac{N_c}{N_f}    \\
X & N_c^2-1 & 1   & 1    & 0   &  R_X = \frac{2}{k+1} \\
\hline
\end{array}
\centering
%\caption{Charges for the electric theory}
%\label{table:charge_table_el_ks_4d}
\end{equation}
\subsection{Calculation of the index}
The expression for the superconformal index on \emph{single particle states} can be easily calculated with equations \eqref{eqn:superconforma_index_chiral} and \eqref{eqn:superconformal_index_vector}
\begin{equation}
% r = R_Q s = R_X / 2
\begin{aligned}
 & i_E (p,q,v, y,\tilde y ,z) = \\ 
 & 
  -\left({p \over 1-p}+{q \over 1-q} -{1 \over (1-p)(1-q)}
\left((p q)^{\frac{R_X}{2}}- (p q)^{1-\frac{R_X}{2}}\right)
\right) \left( p_{N_c}( z)\, p_{N_c}(z^{-1})-1\right ) \\ 
&
\begin{aligned}
&
+{1\over (1-p)(1-q)}
\bigg(
&&(p q)^{{1\over 2}R_Q} \, v \, p_{N_f}(y)\, p_{N_c}(z)
- (p q)^{1- {1\over 2}R_Q} \, {1 \over v}\, p_{{N_f}}(y^{-1})\, p_{{N_c}}(z^{-1}) \\
%\qquad\qquad\qquad\qquad
&
 && +
(p q)^{{1\over 2}R_Q}\, {1\over v}\, p_{{N_f}}({\tilde y}\,) \, p_{{N_c}}(z^{-1})
- (p q)^{1- {1\over 2}R_Q} \, v\, p_{N_f}({\tilde y}^{-1})\, p_{N_c}(z)\bigg) \\
\end{aligned}
\end{aligned}\label{eqn:superconf_index_electric}
\end{equation}
where $ p,q$ are defined as in the previous chapters and the other fugacities are associated to the other symmetries by
\begin{equation}
v \rightarrow U(1)_B \quad y \rightarrow SU(N_F)_L \quad \tilde{y} \rightarrow SU(N_F)_R \quad z \rightarrow SU(N_c) 
\end{equation}
In \eqref{eqn:superconf_index_electric} we have written explicitly the character for the adjoint and the (anti) fundamental representations of $SU(N)$ which are given by \
\begin{gather}
%\begin{aligned}
 \chi_{adj}(z) = p_{N}( z)\, p_{N}(z^{-1})-1 \quad \chi_{N}(y) = p_{N}(y) \quad \chi_{\overbar{N}}(\tilde{y}) = p_{N}(\tilde{y}^{-1}) \\
 p_{N}(x) = \sum_{i =1}^{N} x_i \qquad p_{N}(x) = \sum_{i =1}^{N} \frac{1}{x_i}
%\end{aligned}
\end{gather}
Moreover, we have the following constraints on the fugacities, since they are in the Cartan subgroup of $SU(N_c)$ or $SU(N_f)$
\begin{equation}
\prod_i^{N_c} z_i = 1 \qquad \prod_i^{N_f} y_i = 1 \qquad \prod_i^{N_f} \tilde{y}_i = 1   
\end{equation}
If we use the definition of the polynomials $p_N(x)$ we obtain
\begin{equation}
\begin{aligned}
 & i_E (p,q,v, y,\tilde y ,z) = \\ 
 & 
  -\left({p \over 1-p}+{q \over 1-q} -{1 \over (1-p)(1-q)}
\left((p q)^{\frac{R_X}{2}}- (p q)^{1-\frac{R_X}{2}}\right)
\right) \left( \sum_{1\leq i,j \leq N_c} \frac{z_i}{z_j}-1\right ) \\ 
&
\begin{aligned}
&
+{1\over (1-p)(1-q)} \sum_{i=1}^{N_f} \sum_{j=1}^{N_c}
\bigg(
&&(p q)^{{1\over 2}R_Q} \, v \, y_i \, (z_j)
- (p q)^{1- {1\over 2} R_Q} \, v^{-1} \, y_i^{-1} \, z_j^{-1} \\
%\qquad\qquad\qquad\qquad
&
 && +
(p q)^{{1\over 2}R_Q}\, v^{-1}\, {\tilde y}_i \,z_j^{-1}
- (p q)^{1- {1\over 2}R_Q} \, v\, {\tilde y_i}^{-1}\, z_j\bigg) \\
\end{aligned}
\end{aligned}
\end{equation}
The index is given by the sum of contributions for every field in the theory
\begin{equation}
i_E(p,q,v,y,\tilde{y},z) = i_E^V(p,q,z) + i_E^{X} (p,q,z) + i_E^{\Phi}(p,q,v,y,z) + i_E^{\overbar{\Phi}} (p,q,v,\tilde{y},z)
\end{equation}
and the complete index is given by the Plethystic exponential
\begin{equation}
\begin{aligned}
I_E(p,q,v,y,\tilde{y}) = & \int_{SU(N_c)} d \mu(z) \; \exp \left( \sum_{n=1}^{\infty} \frac{1}{n} i_E(p^n,q^n,v^n,y^n,\tilde{y}^n,z^n) \right) \\
= & \int_{SU(N_c)} d \mu (z) \prod_{fields \; F } I_E^F(p,q,v,y,\tilde{y},z) 
\end{aligned}
\end{equation}
where $I_E^F$ is given by the Plethystic exponential for the field F.\\
Since the fugacity $z$ is in the maximal torus of the gauge group, we can restrict the integral to $T^{N_c -1}$.
\begin{equation}
  \int_{SU(N_c)}\, d \mu(z) \, f(z) = {1 \over N_c!} \int_{T^{N_c-1}} \prod_{i=1}^{N_c }
  { d z_i \over { 2 \pi i z_i} } \Delta (z) \Delta (z^{-1}) f(z) \bigg \rvert_{\prod_{i=1}^{N_c} z_i = 1}
 \end{equation}
 where $\Delta (z) $ is the Vandermonde determinant:
 $$
 \Delta (z ) \, = \, \prod_{ \overset{1 \leq i,j< \leq N_c} { i \neq j}}^{N_c} ( z_i - z_j) \, = \,  \prod_{ \overset{1 \leq i,j< \leq N_c} { i \neq j}}^{N_c}\bigg ( 1 - {z_i \over  z_j } \bigg) \, z_j = \,  \prod_{ \overset{1 \leq i,j< \leq N_c} { i \neq j}}^{N_c}\bigg ( 1 - {z_i \over  z_j } \bigg)
 $$
The last equality holds since $ \prod_{i=1}^{N_c} z_i = 1$.\\
 
We will calculate the superconformal index by considering separately the contributions from the various fields.
The complete calculations for every field can be found in appendix \ref{appendix:index_electric}.\\
\paragraph{Vector field}
The vector field contributes to the index with
\begin{equation}
\begin{aligned} 
I_E^V(p,q,z) = &\exp \bigg( \sum_{n=1}^{\infty} {1 \over n} i_E^{Vett} (p^n,q^n,z^n) \bigg) = \\
 &\exp \bigg( \sum_{n=1}^{\infty} - {1 \over n} \, \bigg( {p^n \over 1-p^n} + {q^n \over 1-q^n} \bigg)  \bigg(\bigg( \sum_{ 1 \leq i,j \leq N_c}  {z_i^n \over z_j^n}  \bigg)- 1 \bigg) \bigg) \, = \\
 = & (p;p)^{N_c-1} ( q;q)^{N_c-1}  \prod_{ 1 \leq i < j \leq N_c} \frac{ 1 }{ \big( 1 -{ z_i \over z_j} \big) \big ( 1 - { z_j \over z_i} \big) \Gamma_e( {z_i \over z_j};p,q) \Gamma_e({z_j \over z_i}; p,q)  } \\
 =& (p;p)^{N_c-1} ( q;q)^{N_c-1}  \prod_{ 1 \leq i < j \leq N_c}  \frac{1}{ \Delta(z) \Delta (z^{-1})} \,  \frac{ 1 }{ \Gamma_e( {z_i \over z_j};p,q) \Gamma_e({z_j \over z_i}; p,q)}
 \end{aligned}
 \label{eqn:sci_electric_vector}
 \end{equation}
 Note there is a term that cancels the Vandermonde determinant coming from the integration measure. This feature is not exclusive to $SU(N)$ gauge group but happens also for $SP(2N),SO(N)$ groups.\\
\paragraph{Adjoint matter}





\section{Magnetic theory}

