\documentclass[10pt,compress]{beamer}
\usepackage[italian]{babel}
\usepackage{amsmath}
\usepackage{mathtools}
\usepackage{amstext}
\usepackage{amssymb}
\usepackage{amsthm}
%\usepackage{fullpage}
%%\usepackage{mathrsfs}
%\usepackage[utf8]{inputenc}
%%\usepackage{natbib}
\usepackage{listings}
\usepackage{color}
\usepackage{verbatim}
\usepackage[T1]{fontenc}
\usepackage{ color}
%\usepackage{lmodern}
\usepackage[toc,page]{appendix}
\usepackage[outercaption]{sidecap}
%\usepackage{floatrow}
\usepackage{graphicx}
\usepackage{color}
\usepackage{url}
%\usepackage{makeidx}
\usepackage{hyperref}
%\usepackage{titlesec}
\usepackage{transparent}
\usepackage{eso-pic}
%\usepackage{mtheme}
%\usepackage{math}
%\usepgfplotslibrary{dateplot}
\hypersetup{colorlinks,
            citecolor=blue,
            filecolor=black,
            linkcolor=black,
            urlcolor=blue}

%\usetheme{Madrid}
%usecolortheme{}
\usetheme[usetitleprogressbar]{m}

%\usepackage{booktabs}
%\usepackage[scale=2]{ccicons}
%\usepackage{minted}





%\usepgfplotslibrary{dateplot}

%\usemintedstyle{trac}

\beamertemplatenavigationsymbolsempty	% NESSUN SIMBOLO DI NAVIGAZIONE 
\setbeamercovered{transparent}

\makeatletter
	\patchcmd{\beamer@sectionintoc}{\vskip 3em}{\vskip 2em}{}{}
	\patchcmd{\beamer@subsectionintoc}{\vskip 2em}{\vskip 2em}{}{}
\makeatother
\setbeamercovered{transparent}



\date{}
\title{\boldmath \bfseries \scshape 4D to 3D reduction of Seiberg duality for {\boldmath SU(N)} susy gauge theories with adjoint matter: a partition function approach}
\author{ \scshape{Carlo Sana} }
\institute{\scshape Università degli Studi di Milano-Bicocca\\
Scuola di Scienze \\
Dipartimento di Fisica "G. Occhialini"
}
\date{\scshape 29 giugno 2015}
%\logo{\includegraphics[width=1.25cm]{logounimib.jpg}}





\begin{document}


\frame{\titlepage}

\begin{frame}{Overview della tesi}
\begin{block}{Dualità KSS}<+->
	Generalizzazione della dualità di Seiberg con un ulteriore campo di materia nella aggiunta del gruppo di gauge $SU(N_c)$
\end{block}
\begin{block}{Riduzione dimensionale}<+->
Ridurre entrambe le teorie da 4D a 3D in modo da mantenere la relazione di dualità tra di loro non può essere fatto in modo banale.\\
\end{block}

\begin{block}{Riduzione sulla funzione di partizione}<+->
Effettueremo la riduzione dimensionale non usando tecniche di teoria di campo, ma calcolando la funzione di partizione e lavorando direttamente su di essa.\\
Questo metodo ha numerosi vantaggi.
\end{block}


\end{frame}

\section{Dualità di Seiberg in 4D e 3D}
\begin{frame}{Le dualità di Seiberg in 4D}
Generalizzazione per teorie di campo supersimmetriche della dualità fra campi elettrici e magnetici di Dirac.\\
\vspace{0.5cm}

Dualità originale ('94) fra due teorie di SQCD con gruppo di gauge $SU(N_c)$.\\
\vspace{0.5cm}

{ \centering \emph{SQCD} = estensione supersimmetrica della QCD \\
}\begin{table}
\begin{tabular}{l l}
Superquark \quad $\longrightarrow$& squark (scalare) + quark \\
Supergluone \quad $\longrightarrow$& gaugino (fermione) + gluone \\
\end{tabular}
\end{table}

\begin{block}{Dualità di Seiberg con gruppo $SU(N_c)$}<2->
%\begin{aligned}
 %\hbox{Teoria Elettrica} &\qquad \hbox{Teoria Magnetica}\\
% 	 SU(N_c) \; SQCD \; \hbox{con} \; N_f \hbox{ flavours} &\longleftrightarrow SU(N_f - N_c) \; SQCD \;\hbox{con}\; N_f \hbox{ flavours}  + N_f^2 \hbox{ Mesons}\\
%\end{aligned}
 %\end{equation} 

\begin{table}
\begin{tabular}{l l}
Teoria elettrica  & $SU(N_c)$  $SQCD$ con $N_f$  flavours \\[0.2cm]
Teoria magnetica & $SU(N_f - N_c)$  $SQCD$ con $N_f$ flavours  + $N_f^2$  Mesons\\
\end{tabular}
\end{table}
\end{block}
\end{frame}



\begin{frame}
	\begin{itemize}
\item<1->	Le due teorie descrivono, a basse energie, lo stesso sistema fisico.\\
\item<2->	Sono entrambe superconformi.
\item <3-> Hanno le stesse simmetrie globali ma un diverso gruppo di gauge.
\item <4->Diverso contenuto di materia: nella teoria magnetica ci sono i mesoni elettrici.
	\end{itemize}
\vspace{0.5cm}
\begin{block}{}<5->
Per tutte le dualità di Seiberg vale la relazione fra le costanti di accoppiamento elettriche e magnetiche
 $$
  g_{el} \sim \frac{1}{\tilde{g}_{mag}} 
$$
\end{block}


\end{frame}




\begin{frame}{Dualità strong-weak coupling}
\onslide<+->{
Quando una teoria è fortemente accoppiata, la teoria duale è in regime perturbativo.\\
Si può calcolare una osservabile nella teoria fortemente accoppiata (difficile da trattare) con tecniche perturbative \alert{ben note} nella teoria duale.\\
%Teorie fortemente accoppiate sono molto difficili da trattare.
}

\vspace{0.5cm}

%\begin{block}{}<+->
Altri esempi di dualità strong-weak coupling:
\begin{itemize}
\item Dualità di Montonen-Oliven
\item AdS/CFT $\rightarrow$ gauge/gravity duality 
\item S-duality in teorie di stringa
\end{itemize}
%\end{block}

\end{frame}

\begin{frame}{Dualità di Seiberg in 3D}
Caratteristiche simili alle dualità di Seiberg in 4D, nonostante le teorie di campo in 3D presentano
\begin{itemize}
\item<1-> diverso contenuto di materia: in 3D i gluoni hanno anche una partner scalare ($\sigma_i$)
\item <2-> ulteriori simmetrie: simmetria assiale e topologica non presente in 4D
\item <3-> uno spazio dei moduli (vuoti) aggiuntivo dovuto agli scalari $\sigma_i$ (\emph{Coulomb branch}) 
\end{itemize}
\vspace{0,5cm}

\onslide<4->{
 La teoria magnetica 3D contiene, oltre ai mesoni, un insieme aggiuntivo di singoletti.\\
 \vspace{0.3cm}
	{ \centering Monopoli della teoria elettrica  $\longrightarrow $ singoletti nella teoria magnetica\\
	}
\vspace{0.3cm}
%Essi permettono il match del \emph{Coulomb branch} fra le due teorie.
}
\end{frame}

\section{Riduzione dimensionale 4D $ \rightarrow$ 3D}

\begin{frame}{4D $ \longrightarrow $ 3D: metodo n\"{a}ive}
%A causa di queste differenze, fino al 2013 non era chiaro come ridurre le teorie da 4D a 3D in modo da preservare la dualità.
\begin{center}
{\bfseries \Large Riduzione naturale: $r \rightarrow 0$}
\end{center}
\vspace{0.5cm}
Si compattificano le teorie su un cerchio di raggio $r$: $$\mathbb{R}^4 \longrightarrow \mathbb{R}^3 \times \mathbb{S}^1$$
Si ignorano tutti i modi di Kaluza-Klein dei campi sul cerchio.\\
Infine, si manda $r \rightarrow 0 $.
\\
\vspace{0.5cm}
\onslide<2->{
	\begin{center}
	\alert{\Large  \bfseries Con questo procedimento non si ottengono due teorie duali in 3D}
	\\
	\vspace{0.3cm}
	Limite a bassa energia incompatibile con la relazione di dualità.
	\end{center}
	}
\end{frame}

\begin{frame}[fragile]
\frametitle{4D $\longrightarrow $ 3D: metodo corretto}
\begin{center}
{\Large \bfseries Riduzione corretta: $r$ finito}\\
\vspace{0.5cm}
La finitezza del cerchio genera un termine di superpotenziale dovuto a un instantone di Kaluza-Klein.\\
\vspace{0.3cm}
Il superpotenziale rompe le simmetrie presenti in 3D che sono proibite in 4D.\\
\vspace{0.3cm}
Si ottengono teorie in 3D considerandone il limite a basse energie:
per energie $\ll \frac{1}{r}$ la dinamica sul cerchio si disaccoppia. 
\\
\vspace{0.3cm}
Il termine aggiuntivo nel superpotenziale rimane anche a basse energie.
\end{center}
\end{frame}


\plain{Si può arrivare a una dualità 3D senza il termine di superpotenziale causato dalla presenza del cerchio?}



\begin{frame}{RG Flow verso una teoria senza superpotenziale}
\begin{block}{Dualità di Seiberg \& KSS}
Integrando una coppia di quark con grandi masse reali è possibile giungere a una dualità 3D senza il termine di superpotenziale dovuto alla taglia finita del cerchio. 
\end{block}

\vspace{0.5cm}
A seguito di ciò il gruppo di simmetria globale si rompe e si genera la simmetria assiale

\begin{align*}
SU(N_f+1)_L \times &SU(N_f+1)_R \times U(1)_B \\
& \downarrow \\
 SU(N_f)_L \times SU& (N_f)_R \times U(1)_B \times U(1)_A
\end{align*}


\end{frame}




\begin{frame}{Riduzione dimensionale in teoria di campo}

La riduzione delle dualità è compresa in teoria di campo per diverse dualità 4D.\\
\vspace{0.3cm}
Nel caso della dualità KSS ( dualità di Seiberg con materia nell'aggiunta ),
è necessario deformare la teoria aggiungendo una perturbazione nel superpotenziale. \\
\vspace{0.3cm}
Nella teoria duale, essa rompe la teoria in vari settori senza aggiunta con gruppo di gauge 
$$
SU(k N_f - N_c)  = \prod_i^k SU(n_i) \times U(1)^k
\qquad \hbox{con} \; \sum_i n_i = k N_f - N_c
$$
\end{frame}








\section{Riduzione della dualità con la funzione di partizione}
\begin{frame}{Indice superconforme}


\end{frame}






\nocite{*}
%\bibliographystyle{plain}
\bibliography{biblio}

\end{document}
