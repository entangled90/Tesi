\documentclass[9pt,compress]{beamer}
\usepackage[italian]{babel}
\usepackage{amsmath}
\usepackage{mathtools}
\usepackage{amstext}
\usepackage{amssymb}
\usepackage{amsthm}
%\usepackage{fullpage}
%%\usepackage{mathrsfs}
\usepackage[utf8]{inputenc}
%%\usepackage{natbib}
\usepackage{listings}
\usepackage{color}
\usepackage{verbatim}
\usepackage[T1]{fontenc}
\usepackage{ color}
\usepackage{lmodern}
\usepackage[toc,page]{appendix}
\usepackage[outercaption]{sidecap}
%\usepackage{floatrow}
\usepackage{graphicx}
\usepackage{color}
\usepackage{url}
%\usepackage{makeidx}
\usepackage{hyperref}
%\usepackage{titlesec}
\usepackage{transparent}
\usepackage{eso-pic}
%\usepgfplotslibrary{dateplot}
\hypersetup{colorlinks,
            citecolor=blue,
            filecolor=black,
            linkcolor=black,
            urlcolor=blue}

\usetheme{Madrid}
\usecolortheme{}
\beamertemplatenavigationsymbolsempty	% NESSUN SIMBOLO DI NAVIGAZIONE 
\setbeamercovered{transparent}

\makeatletter
	\patchcmd{\beamer@sectionintoc}{\vskip 3em}{\vskip 2em}{}{}
	\patchcmd{\beamer@subsectionintoc}{\vskip 2em}{\vskip 2em}{}{}
\makeatother
\setbeamercovered{transparent}



\date{}
\title{\boldmath \bfseries \scshape 4D to 3D reduction of Seiberg duality for $SU(N)$ susy gauge theories with adjoint matter: a partition function approach}
\author{ \scshape{Carlo Sana} }
\institute{\scshape Università degli Studi di Milano-Bicocca\\
Scuola di Scienze \\
Dipartimento di Fisica G. Occhialini
}
\date{\scshape 29 giugno 2015}
%\logo{\includegraphics[width=1.25cm]{logounimib.jpg}}





\begin{document}


\frame{\titlepage}

\begin{frame}{Overview della tesi}
\begin{block}{Dualità KSS}<+->
	Generalizzazione della dualità di Seiberg con un ulteriore campo di materia nella aggiunta del gruppo di gauge $SU(N_c)$
\end{block}
\begin{block}{Riduzione dimensionale}<+->
Ridurre entrambe le teorie da 4D a 3D in modo da mantenere la relazione di dualità tra di loro non può essere fatto in modo banale.\\
\end{block}

\begin{block}{Riduzione sulla funzione di partizione}<+->
Effettueremo la riduzione dimensionale non usando tecniche di teoria di campo, ma calcolando la funzione di partizione e lavorando direttamente su di essa.\\
Questo metodo ha numerosi vantaggi.
\end{block}


\end{frame}


\begin{frame}{Le dualità di Seiberg in 4D}
Generalizzazione per teorie di campo supersimmetriche della dualità fra campi elettrici e magnetici di Dirac.\\

\begin{block}{Dualità di Seiberg con gruppo $SU(N_c)$}<1->
Dualità originale ('94) scoperta in teoria di campo.\\
\emph{SQCD} = estensione supersimmetrica della QCD con gruppo di gauge $SU(N_c)$
\begin{equation}
\begin{aligned}
 \hbox{Teoria Elettrica} &\qquad \hbox{Teoria Magnetica}\\
 	 SU(N_c) \; SQCD \; \hbox{con} \; N_f \hbox{ flavours} &\longleftrightarrow SU(N_f - N_c) \; SQCD \;\hbox{con}\; N_f \hbox{ flavours}  + N_f^2 \hbox{ Mesons}\\
\end{aligned}
 \end{equation} 
\end{block}

\begin{block}{}<+->
	\begin{itemize}[<+->]
\item	Le due teorie descrivono, a basse energie, lo stesso sistema fisico.\\
\item	Sono entrambe superconformi.
\item  Hanno le stesse simmetrie globali ma un diverso gruppo di gauge.
\item Diverso contenuto di materia (i mesoni)
	\end{itemize}
	
\end{block}
\end{frame}




\begin{frame}{Dualità strong-weak coupling}
\begin{block}{}<+->
Esempio di dualità strong-weak coupling:\\
quando una teoria è fortemente accoppiata, la teoria duale è in regime perturbativo.\\
La dualità mi garantisce che posso calcolare una osservabile nella teoria fortemente accoppiata (che non so trattare) con tecniche perturbative nella teoria duale.\\
%Teorie fortemente accoppiate sono molto difficili da trattare.
\end{block}

\begin{block}{}<+->
Per tutte le dualità di Seiberg vale la relazione fra le costanti di accoppiamento
 $$
  g_{el} \sim \frac{1}{\tilde{g}_{mag}} 
$$
\end{block}


\begin{block}{}<+->
Altri esempi di dualità strong-weak coupling:
\begin{itemize}
\item AdS/CFT $\rightarrow$ gauge/gravity duality 
\item Dualità di Montonen-Oliven
\item S-duality in teoria delle stringhe
\end{itemize}
\end{block}

\end{frame}

\begin{frame}{Dualità di Seiberg in 3D}
\begin{block}{}
Caratteristiche simili alle dualità di Seiberg in 4D, nonostante le teorie di campo in 3D presentano
\begin{itemize}
\item<1-> diverso contenuto di materia: in 3D i gluoni hanno anche una partner scalare
\item <2-> ulteriori simmetrie: simmetria assiale e topologica non presente in 4D
\item <3-> insieme di vuoti aggiuntivo a causa degli scalari aggiuntivi 
\end{itemize}
\end{block}

\begin{block}{Dualità in 3D}<4>
A causa della diversa struttura dei vuoti la teoria magnetica contiene, oltre ai mesoni, un insieme aggiuntivo di singoletti.
$$
\hbox{Monopoli della teoria magnetica} \; \longrightarrow \; \hbox{Singoletti aggiuntivi nella teoria magnetica}
$$
\end{block}
\end{frame}

\section{Riduzione dimensionale $4D \rightarrow 3D$}
\begin{frame}{$4D\overset{?}{ \longrightarrow} 3D $}
%A causa di queste differenze, fino al 2013 non era chiaro come ridurre le teorie da 4D a 3D in modo da preservare la dualità.

\begin{block}{Riduzione naturale delle teorie}<2->
Si compattificano le teorie su un cerchio: $\mathbb{R}^4 \longrightarrow \mathbb{R}^3 \times \mathbb{S}^1$, con  cerchio di raggio $r$.\\
Si ignorano i modi di Kaluza-Klein dei campi sul cerchio mandando $r \rightarrow 0 $
\end{block}








\begin{block}{}<3->
Il limite a $r \rightarrow 0$ comporta
$$
 g^2_{3D} = \frac{g^2_{4D}}{2 \pi r} \; \rightarrow \infty \quad \hbox{per} \; r\rightarrow 0
$$
Ma stiamo trattando dualità a strong-weak coupling dove
$$
g_{el} \sim \frac{1}{g_{mag}}
$$ 
che è incompatibile con il limite $r \rightarrow 0 $ .
\end{block}

\end{frame}

\begin{frame}{$4D \rightarrow 3D$ }
\begin{block}{}<+->
Per mantere la dualità tra le teorie è necessario mantenere il raggio del cerchio finito.\\
\end{block}
\begin{block}{}<+->
L'effetto della taglia finita del cerchio consiste nel considerare un modo istantonico sul cerchio che genera un nuovo termine superpotenziale.
\end{block}
\begin{block}{}<+->
Il superpotenziale istantonico rompe le simmetrie in 3D che non sono permesse in 4D ( simmetria assiale) . 
\end{block}

\begin{block}{}<+->
Si ottengono teorie in 3D considerandone il limite a basse energie.\\
I modi di Kaluza-Klein sul cerchio si disaccoppiano per energie $\ll \frac{1}{r}$
\end{block}

\end{frame}

\begin{frame}{Riduzione dimensionale in teoria di campo}
\begin{block}{}
La riduzione delle dualità in teoria di campo è stato effettuato per varie dualità note in 4D.\\

\end{block}

\end{frame}







\nocite{*}
%\bibliographystyle{plain}
\bibliography{biblio}

\end{document}
