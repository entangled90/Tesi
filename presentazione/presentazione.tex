\documentclass[]{beamer}
\usepackage[italian]{babel}
\usepackage{amsmath}
\usepackage{mathtools}
\usepackage{amstext}
\usepackage{amssymb}
\usepackage{amsthm}
%\usepackage{fullpage}
%%\usepackage{mathrsfs}
\usepackage[utf8]{inputenc}
%%\usepackage{natbib}
\usepackage{listings}
\usepackage{color}
\usepackage{verbatim}
\usepackage[T1]{fontenc}
\usepackage{ color}
\usepackage{lmodern}
\usepackage[toc,page]{appendix}
\usepackage[outercaption]{sidecap}
%\usepackage{floatrow}
\usepackage{graphicx}
\usepackage{color}
\usepackage{url}
%\usepackage{makeidx}
\usepackage{hyperref}
%\usepackage{titlesec}
\usepackage{transparent}
\usepackage{eso-pic}
\usetheme{Berlin}
\usecolortheme{beaver}
\date{}
\title{\boldmath \bfseries \scshape 4D to 3D reduction of Seiberg duality for $SU(N)$ susy gauge theories with adjoint matter: a partition function approach}
\author{ \scshape{Carlo Sana} }
\institute{\scshape Università degli Studi di Milano-Bicocca\\
Scuola di Scienze - Dipartimento di Fisica G. Occhialini
}
\date{\scshape 29 giugno 2015}
\logo{\includegraphics[width=1.25cm]{logounimib.jpg}}
\begin{document}

\frame{\titlepage}

\begin{frame}{Teorie quantistiche di campo e gruppo di rinormalizzazione}
	\begin{block}{Gruppo di rinormalizzazione}
		Parametri della teoria: $m_i, \, g_i$ non fissati.
		\\
		\begin{equation}
		\frac{d g_i}{d \mu} \neq 0 \; \longrightarrow \; g_i = g_i(\mu) \qquad \mu =\hbox{scala tipica del processo}
		\end{equation}
	\end{block}

\end{frame}
\begin{frame}{$QCD$ vs $QED$}
	\begin{block}{$QED$}
		costante di accoppiamento $g $= carica elettrica $e$\\
		$$
		\alpha = \frac{e^2}{4\pi} \rightarrow \alpha(\mu \rightarrow 0) = \frac{1}{137} \lle 1  \quad \hbox{basse energie} \simeq \hbox{eq. classiche}
		$$
	\end{block}
\begin{block}{$QCD$}
		
	\end{block}


\end{frame}

\begin{frame}{Dualità a strong-weak coupling}
	\begin{block}
	
	\end{block}
\end{frame}


\end{document}
