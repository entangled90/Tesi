\documentclass[]{beamer}
\usepackage[italian]{babel}
\usepackage{amsmath}
\usepackage{mathtools}
\usepackage{amstext}
\usepackage{amssymb}
\usepackage{amsthm}
%\usepackage{fullpage}
%%\usepackage{mathrsfs}
\usepackage[utf8]{inputenc}
%%\usepackage{natbib}
\usepackage{listings}
\usepackage{color}
\usepackage{verbatim}
\usepackage[T1]{fontenc}
\usepackage{ color}
\usepackage{lmodern}
\usepackage[toc,page]{appendix}
\usepackage[outercaption]{sidecap}
%\usepackage{floatrow}
\usepackage{graphicx}
\usepackage{color}
\usepackage{url}
%\usepackage{makeidx}
\usepackage{hyperref}
%\usepackage{titlesec}
\usepackage{transparent}
\usepackage{eso-pic}
\usetheme{Berlin}
\usecolortheme{beaver}
\date{}
\title{\boldmath \bfseries \scshape 4D to 3D reduction of Seiberg duality for $SU(N)$ susy gauge theories with adjoint matter: a partition function approach}
\author{ \scshape{Carlo Sana} }
\institute{\scshape Università degli Studi di Milano-Bicocca\\
Scuola di Scienze - Dipartimento di Fisica G. Occhialini
}
\date{\scshape 29 giugno 2015}
\logo{\includegraphics[width=1.25cm]{logounimib.jpg}}
\begin{document}

\frame{\titlepage}

\begin{frame}{Teorie quantistiche di campo e gruppo di rinormalizzazione}
	\begin{block}{Gruppo di rinormalizzazione}
		Parametri della teoria: $m_i, \, g_i$ non fissati.
		\\
		\begin{equation}
		\frac{d g_i}{d \mu} \neq 0 \; \longrightarrow \; g_i = g_i(\mu) \qquad \mu =\hbox{scala tipica del processo}
		\end{equation}
	\end{block}

\end{frame}

\begin{frame}{$QED$}
	
	\begin{block}{$QED$}
	costante di accoppiamento $g $ = carica elettrica $e$\\
	\begin{equation}
		\alpha = \frac{e^2}{4\pi} \qquad \alpha(\mu \rightarrow 0) \rightarrow \frac{1}{137} 
		 \ll 1
	\end{equation}
	\end{block}
	\begin{block}{}
	Basse energie $ \longrightarrow$ Teoria perturbativa\\
	La carica elettrica cresce lentamente al crescere dell'energia:
	\begin{equation}
	\alpha(0) \sim \frac{1}{137} \qquad \alpha(m_Z = 90 GeV) \sim \frac{1}{128}
	\end{equation}
	\end{block}

\end{frame}

\begin{frame}{$QCD$}
	\begin{block}{}
	Comportamento opposto: \emph{asymptotic freedom}
	\begin{equation}
	\alpha_{strong} \left( \mu  = 200 MeV\right) \gg 1 \qquad \alpha \left({\mu = \infty}\right) = 0 
	\end{equation}
	Tecniche perturbative: $\mu \geq 3$ GeV
	\end{block}

	\begin{block}{}
	Per $\mu \leq 1 GeV$
	Tecniche perturbative non applicabili\\
	
	\end{block}

\end{frame}


\end{document}
