\documentclass[10pt,compress,usenames,dvipsnames]{beamer}
\usepackage[italian]{babel}
\usepackage{amsmath}
\usepackage{mathtools}
\usepackage{amstext}
\usepackage{amssymb}
\usepackage{amsthm}
%\usepackage{fullpage}
%%\usepackage{mathrsfs}
%\usepackage[utf8]{inputenc}
%%\usepackage{natbib}
\usepackage{verbatim}
\usepackage[T1]{fontenc}
%\usepackage{lmodern}
%\usepackage[toc,page]{appendix}
%\usepackage[outercaption]{sidecap}
%\usepackage{floatrow}
%\usepackage{graphicx}
%\usepackage{color}
%\usepackage{url}
%\usepackage{makeidx}
\usepackage{hyperref}
%\usepackage[usenames,dvipsnames]{xcolor}

\hypersetup{colorlinks,
            citecolor=blue,
            filecolor=black,
            linkcolor=black,
            urlcolor=blue}

%%%% ARROWSS
\usepackage{tikz}
\usetikzlibrary{shapes.arrows}

\tikzset{
    myarrow/.style={
        draw,
        fill=BurntOrange,
        single arrow,
        minimum height=3.5ex,
        single arrow head extend=0.8ex
    }
}
\newcommand{\arrowup}{%
\tikz [baseline=-0.5ex]{\node [myarrow,rotate=90] {};}
}
\newcommand{\arrowdown}{%
\tikz [baseline=-1.5ex]{\node [myarrow,rotate=-90] {};}
}
% color degli alert


%\usetheme{Madrid}
%usecolortheme{}
\usetheme[usetitleprogressbar]{m}

\setbeamercolor{alerted text}{fg=BurntOrange}

%\usepackage{booktabs}
%\usepackage[scale=2]{ccicons}
%\usepackage{minted}


%% bibliography?
\setbeamertemplate{bibliography item}{}

%\usepgfplotslibrary{dateplot}

%\usemintedstyle{trac}

\beamertemplatenavigationsymbolsempty	% NESSUN SIMBOLO DI NAVIGAZIONE 
\setbeamercovered{transparent}

\makeatletter
	\patchcmd{\beamer@sectionintoc}{\vskip 3em}{\vskip 2em}{}{}
	\patchcmd{\beamer@subsectionintoc}{\vskip 2em}{\vskip 2em}{}{}
\makeatother
\setbeamercovered{transparent}



\date{}
\title{\boldmath \bfseries \scshape 4D to 3D reduction of Seiberg duality for {\boldmath SU(N)} susy gauge theories with adjoint matter: a partition function approach}
\author{\scshape{{\bfseries Carlo Sana}\\
Relatore: Silvia Penati\\
Correlatore esterno: Antonio Amariti\\
Correlatore: Alberto Zaffaroni} }
\institute{\scshape Università degli Studi di Milano-Bicocca\\
Scuola di Scienze \\
Dipartimento di Fisica "G. Occhialini"
}
\date{\scshape 29 giugno 2015}
%\logo{\includegraphics[width=1.25cm]{logounimib.jpg}}









\begin{document}


\begin{frame}[plain]
\maketitle
\end{frame}

\begin{frame}{Overview della presentazione}
\tableofcontents[pausesections]
\end{frame}

%\begin{frame}{Overview della tesi}
%\begin{block}{Dualità KSS}<+->
%	Generalizzazione della dualità di Seiberg con un ulteriore campo di materia nella aggiunta del gruppo di gauge $SU(N_c)$
%\end{block}
%\begin{block}{Riduzione dimensionale}<+->
%Ridurre entrambe le teorie da 4D a 3D in modo da mantenere la relazione di dualità tra di loro non può essere fatto in modo banale.\\
%\end{block}

%\begin{block}{Riduzione sulla funzione di partizione}<+->
%Effettueremo la riduzione dimensionale non usando tecniche di teoria di campo, ma calcolando la funzione di partizione e lavorando direttamente su di essa.\\
%Questo metodo ha numerosi vantaggi.
%\end{block}
%\end{frame}

\section{Dualità strong/weak coupling}
































\begin{frame}{QFT a strong coupling}
\begin{tabular}{c l l }
Relatività speciale  & \\[0,1cm]
{\Large +} & {\Large=}  & \alert{\bfseries \large Teoria quantistica dei campi (QFT) }\\
Meccanica Quantistica & \\
\end{tabular}
\\
\vspace{0.3cm}
Metodi {\bfseries perturbativi} utilizzabili a \alert{\bfseries weak coupling}: \\
sviluppi in serie nella costante di accoppiamento (\small e.g.\alert { carica elettrica}) \\[0,4cm]

\begin{block}{Gruppo di rinormalizzazione}
Le costanti di accoppiamento variano in funzione della scala di energia:	\\
la teoria a bassa energia può fluire a \alert{\bfseries strong coupling} \\
(e.g. confinamento in QCD)
\end{block}

\alert{\bfseries Nessuno strumento teorico per studiarne la dinamica.}
$\longrightarrow$ QCD su reticolo

\end{frame}

\plain{Esiste uno strumento teorico per trattare teorie a strong coupling?}


\begin{frame}{Strong coupling e dualità EM in QFT}
\begin{center}
 \alert{\bfseries \Large Dualità strong/weak coupling}
\end{center}
{\bfseries Dualità:} le due teorie sono fisicamente equivalenti 
\\[0,3cm]
\onslide<+->{
Legame fra le costanti di accoppiamento tra teorie duali:
$$
  g \sim \frac{1}{\tilde{g}}  \quad \longrightarrow \quad \hbox{\alert{strong-weak coupling}}
$$
}

\onslide<+->{
Si può calcolare una osservabile nella teoria \alert{\bfseries fortemente accoppiata} con \alert{\bfseries tecniche perturbative} ben note nella teoria duale.\\
%Teorie fortemente accoppiate sono molto difficili da trattare.
}
\end{frame}

















\begin{frame}{Esempi di dualità strong-weak coupling}
\vspace{-0,7cm}
\begin{itemize}
\item Dualità EM di Dirac
\item Dualità di Montonen-Olive
\item Dualità di Seiberg e generalizzazioni
\item AdS/CFT $\rightarrow$ gauge/gravity duality 
\item S-duality in teorie di stringa
\end{itemize}

\vspace{1cm}
\onslide<2->{
Le dualità di Seiberg sono una \alert{generalizzazione} della dualità di Dirac per teorie di gauge \alert{non-abeliane} con supersimmetria.
}
\\[0,5cm]
%\onslide<3->{
%In particolare, con la dualità AdS/CFT:
%\begin{itemize}
%\item calcolo viscosità del quark/gluon plasma
%\item modelli per superconduttori olografici (AdS$_4$/CFT$_3$)
%\end{itemize}
%}
%\end{comment}

\end{frame}

\begin{frame}{Dualità elettrica-magnetica di Dirac}

\begin{block}{Dualità di Dirac}
Aggiungendo sorgenti per il campo magnetico $J_{\, mag}^{\mu}$ si ottiene una invarianza $\mathbb{Z}_2$ delle equazioni di maxwell sotto la trasformazione
\begin{equation*}
 \left( {E}^i, {B}^i \right) \longrightarrow \left({B}^i, - {E}^i \right)
 \qquad \left( J_{\, el}^{\mu }, J_{\, mag}^{\mu} \right) \longrightarrow \left( 
 J_{\, mag}^{\mu},  - J_{\, el}^{\mu }
 \right)
 \quad  J^{\mu} = ( \rho , J^i)
\end{equation*}
Dualità EM + Meccanica Quantistica: condizione di \alert{\bfseries quantizzazione} della carica elettrica
$$
e g  = 2 \pi \hbar n
$$
\vspace{-0,3cm}
\onslide<2->{Carica elettrica e magnetica sono \alert{\bfseries inversamente} proporzionali.
\vspace{0,3cm}
\begin{center}
\alert{\bfseries \large primo esempio di dualità strong/weak coupling }
\end{center}
}\end{block}

\vspace{0.5cm}
\end{frame}



\section{Dualità di Seiberg e di Kutasov-Schwimmer-Seiberg}



\begin{frame}{Caratteristiche generali della dualità di Seiberg}
\begin{center}
 \begin{center}
 {\bfseries \Large Dualità di Seiberg e Kutasov-Schwimmer-Seiberg (KSS)} \\[0,4cm]
\end{center}
\begin{tabular}{r c l }
{ \large Teoria elettrica } & $\longleftrightarrow$ & {\large Teoria magnetica}
\end{tabular}
\\[0,2cm]

{ 
	QCD con supersimmetria minimale ($\mathcal{N}=1$) con gruppo $SU(N)$.\\
	Dualità KSS ha un ulteriore campo di materia nell'aggiunta.  
}
%
%{\bfseries Teoria elettrica} $\longleftrightarrow${\bfseries Teoria magnetica}
\end{center}
%\vspace{0,3cm}
\begin{columns}[c]
	\begin{column}{0.5 \textwidth}			
		\begin{center}
		{\bfseries Uguali} \\
		%\vspace{0,2cm}
		Funzioni di correlazione \\
		Simmetrie Globali (\alert{fisiche})
		Struttura dei vuoti (\alert{susy})
		\vspace{0,3cm}
		\end{center}
	\end{column}

	\begin{column}{0.5 \textwidth}
\vspace{-0,3cm}
		\begin{center}
		{\bfseries Diverse}\\
	%	\vspace{0,2cm}
		Particelle (\alert{mesoni})\\
		Costanti di accoppiamento \\
		Dinamica (\alert{numero di colori})\\
		\end{center}
	\end{column}
\end{columns}
\begin{center}
\onslide<2->{\large  Dualità a \alert{\bfseries basse energie} $\longrightarrow$ \alert{\bfseries punto fisso superconforme} \\
Ad alte energie descrivono sistemi diversi
}
\end{center}
\end{frame}
















\begin{frame}{Dualità di Seiberg in 3D}
Simili a teorie 4D $\mathcal{N}=1$ ma con alcune differenze
\begin{block}{Differenze delle teorie di campo 3D $\mathcal{N}=2$}
\begin{itemize}
\item ulteriori simmetrie: in 4D no simmetria \alert{\bfseries assiale} e \alert{\bfseries topologica} 
\item diverso contenuto di materia: in 3D i gluoni hanno anche una partner \alert{\bfseries scalare} 
\item uno spazio dei moduli (\alert{\bfseries vuoti supersimmetrici}) con un \emph{branch} aggiuntivo  
\end{itemize}
\end{block}
\begin{center}
\onslide<2->{\'{E} necessario far combaciare questi nuovi \emph{branch} fra le due teorie
\\[0,2cm]
\alert{\bfseries Insieme aggiuntivo di singoletti nella teoria magnetica}
}\end{center}

\end{frame}



\section{Riduzione dimensionale 4D $ \rightarrow$ 3D}






\begin{frame}{4D $ \longrightarrow $ 3D: metodo n\"{a}ive}
%A causa di queste differenze, fino al 2013 non era chiaro come ridurre le teorie da 4D a 3D in modo da preservare la dualità.
%\begin{center}
%{\bfseries \Large Riduzione naturale: $r \rightarrow 0$}
%\end{center}
\vspace{-0.5cm}
\begin{block}{Riduzione naturale: $r \rightarrow 0$}
Si compattificano le teorie su un cerchio di raggio $r$: $$\mathbb{R}^4 \longrightarrow \mathbb{R}^3 \times \mathbb{S}^1$$
Si ignora la dinamica sul cerchio e si manda $r \rightarrow 0$.\\
\alert{\bfseries  Si ottengono due teorie che non sono duali fra loro}
\end{block}

\begin{block}{Riduzione corretta: $r$ finito}
La finitezza del cerchio modifica la dinamica e impone vincoli tipici della teorie 4D (\alert{\bfseries anomalie}), generati da un termine di superpotenziale $\eta$.\\

\end{block}
Si scende a energie $\ll \frac{1}{r}$: la dinamica sul cerchio si disaccoppia.
\end{frame}


%\plain{Si può arrivare a una dualità 3D senza i vincoli causati dalla presenza del cerchio?}

\begin{frame}{Riduzione dualità KSS - Teoria deformata}

Si possono rimuovere questi vincoli con un particolare RG flow.\\
Inoltre, si riesce a generare la simmetria assiale (\alert{\bfseries assente in 4D}).
\\
\vspace{0,6cm}
Si è in grado di ridurre la \alert{\bfseries dualità KSS} come fatto in letteratura solo se si introduce una \alert{\bfseries perturbazione} al superpotenziale.\\[0,2cm]
Con essa, si riesce a generare i \alert{\bfseries singoletti} necessari per far coincidere i vuoti.
\\
\vspace{0,3cm}
\end{frame}


\begin{frame}{Riduzione dualità KSS - Teoria senza deformazione }
Per ottenere la dualità 3D standard si rimuove la deformazione.\\
L'unico problema di questo passaggio è che non si ha modo di generare i singoletti.
\\
\vspace{0,5cm}
\begin{center}
{\bfseries Devo assumere che rimuovendo la deformazione ottengo comunque i singoletti corretti.\\[0,3cm]
\alert{Non ci sono però giustificazioni teoriche a riguardo.}
}
\end{center}

\end{frame}


\plain{Ho modo di verificare se questa intuizione è corretta? }



\section{Riduzione della dualità sulla funzione di partizione}

\begin{frame}{Indice superconforme in 4D e funzioni di partizione in 3D} 
Si calcola l'indice superconforme $I_{el} \; \& \; I_{mag} $: è \alert{\bfseries uguale} per teorie duali.\\
Esso conta i multipletti BPS corti su $\mathbb{R}^3 \times \mathbb{S}^1$,
che una volta integrati i modi sul cerchio si riducono a stati sulla sfera 3D.
\\
%\vspace{0.2cm}
\vspace{0,2cm}
Infatti nel limite $r \rightarrow 0$ l'indice si riduce alla funzione di partizione della teoria in 3D \alert{\bfseries con superpotenziale $\eta$.}\\
\begin{table}
\begin{tabular}{r l }
{\bfseries Indice superconf.} & funzioni gamma ellittiche $\Gamma_e $
\\
{\bfseries Funz. di partiz.} & funzioni gamma iperboliche $\Gamma_h$
\end{tabular}
\end{table}
\begin{center}
{\bfseries Identità matematiche}: $ \Gamma_e  \overset{\small{ r \rightarrow 0}}{\longrightarrow} \Gamma_h $
\end{center}
\begin{equation*}
\begin{array}{l l l l }
\hbox{4D:}  \qquad  & I_{el} & =  &I_{mag}   \\[0,2cm]
r \rightarrow 0 &\large{\downarrow }&   & \large{\downarrow   } \\[0,2cm]
\hbox{3D:} \qquad & Z^{\eta}_{el} & =&  Z^{\eta}_{mag} \end{array}
\end{equation*}
\end{frame}

\begin{frame}{Indici e funzioni di partizione}

\begin{center}
{
\large \bfseries La dualità in 4D impone che gli indici superconformi siano uguali
}
\end{center}
\onslide<2->{\begin{center}
\arrowdown \\[0,2cm]
\alert{\bfseries \large  Le funzioni di partizione in 3D con superpotenziale $\eta$ sono uguali}

Per rimuovere i vincoli, si fa un RG flow simile a quanto fatto precedentemente, ma direttamente \alert{\bfseries sulla funzione di partizione.}\\[0,3cm]
Si genera anche la simmetria \alert{\bfseries assiale}, richiesta per la dualità.
}

\end{center}

\end{frame}


\begin{frame}{Flow verso una teoria senza superpotenziale $\eta$}

%\\
Lavorando direttamente sulla funzione di partizione \alert{\bfseries non è necessario} introdurre una deformazione, come è stato fatto precedentemente. \\
%\vspace{0.2cm}
%$$
 %SU(k (N_f + 1) - N_c ) \longrightarrow U(k N_f - N_c ) \times U(k) / U(1)
%$$
\vspace{0.5cm}
\begin{block}{Singoletti}
Utilizzando una identità \alert{\bfseries matematica} fra gamma iperboliche $\Gamma_h$  si ottengono i \alert{\bfseries singoletti} corretti per la dualità.\\
Coincidono con quelli trovati in teoria di campo con la deformazione.
\end{block}

\end{frame}

\begin{frame}{Conclusioni}
\begin{center}
{\large   
\'{E} stato \alert{\bfseries verificato} che la riduzione fatta precedentemente in letteratura, basata su un'assunzione \alert{\bfseries non giustificata}, è \alert{\bfseries corretta},
confermando così il legame fra la dualità KSS $SU(N)$ in 4D e 3D in riduzione dimensionale.
}\\
\vspace{0.7cm}
\onslide<3->{
Infine, l'identità fra le due funzioni di partizione $ Z_{el} = Z_{mag}$ porta a una \alert{\bfseries identità} integrale tra funzioni iperboliche $\Gamma_h$ 
\alert{\bfseries non ancora dimostrate} matematicamente.
}\\
\end{center}



\end{frame}

\plain{Grazie per l'attenzione}

\bibliographystyle{jhep}
\begin{frame}{Bibliografia}
\nocite{*}
{ \footnotesize
\bibliography{biblio}
}
\end{frame}

\begin{frame}{Funzione di partizione elettrica}
\begin{multline*}
Z_{el} ( \mu_i , \nu_i )\,  =  
 %\frac{1}{ (2 \ \pi i)^{N_c} }
 {1 \over N_c ! }
\Gamma_h ( \Delta_X \omega )^{N_c-1} 
\\ \int
\prod_{i=1}^{N_c} \frac{d \sigma_i}{\sqrt{- \omega_1 \omega_2}} \, \delta( \sum_i \sigma_i)  
%c \left( 4\big( \sum_{i=1}^{N_c}  \sigma_i  \big) \big( \omega( N_f (1 - \Delta)  - N_c \Delta_X)  -  m_A N_f \big) \right) 
	 \prod_{ 1 \leq i<j \leq N_c} \frac{ \Gamma_h( \Delta_X \omega \pm (\sigma_i - \sigma_j)) }{ \Gamma_h ( \pm (\sigma_i - \sigma_j) )}\\
 \prod_{a,b=1}^{N_f} \prod_{j=1}^{N_c} \Gamma_h ( m_a + m_B + m_A + \sigma_j) \Gamma_h ( -\tilde{m}_a -m_B + m_A - \sigma_j)
\end{multline*}

\end{frame}

\begin{frame}{Funzione di partizione magnetica}

\vspace{-1cm}
\begin{equation*}
\begin{aligned}
Z_{mag} % ( \mu_a , \nu_b , \tilde{\mu_a}, \tilde{\nu_b} ) 
\,= & \, \frac{1}{(k N_f - N_c)!}
 \Gamma_h ( \Delta_X \omega ; \omega_1 , \omega_2)^{ k N_f - N_c -1}  \\
	% & c \left( 4 N_c m_B) ( - m_A N_f + \omega( -N_c \Delta_X + N_f ( 1 - \Delta_Q))) \right) \\
 &  \left( \prod_{j=0}^{k-1}
\prod_a^{N_f } \prod_b^{N_f}  \Gamma_h \big( \mu_a+  \nu_b + j \omega \Delta_X) \big)  \right) \\
&\int  \prod_{i=1}^{ k N_f - N_c } \frac{d \sigma_i }{ \omega_1 \omega_2} \,\int
d \xi \, e^{ \frac{\pi i }{ \omega_1 \omega_2} 2 \xi  \left( m_B N_c + \sum \sigma_i \right)}  
\prod_{i<j}^{k N_f - N_c } \frac{ \Gamma_h( \Delta_X \omega \pm (\sigma_i - \sigma_j)) }{ \Gamma_h ( \pm (\sigma_i - \sigma_j) )}
	\\&
\begin{aligned}
	 &  \bigg( \prod_{a,b}^{N_f} \prod_{j=1}^{k N_f - N_c }
	 && \Gamma_h \big( - m_a - m_A + \omega (\Delta_X - \Delta_Q)  + \tilde{\sigma}_j \big) \\
 & && \Gamma_h \big(  + \tilde{m}_b - m_A + \omega (\Delta_X - \Delta_Q)  - \tilde{\sigma}_j \big)  \bigg)
	\end{aligned}
	\\
	& \prod_{j'=0}^{k-1} \Gamma_h \left( \pm {\xi} +\omega \left(
	N_f ( 1 - \Delta_Q ) - \Delta_X ( N_c - j')  \right) - m_A N_f
 \right)
\end{aligned}
\end{equation*}

\end{frame}



\end{document}
