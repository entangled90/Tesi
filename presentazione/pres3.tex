\documentclass[10pt,compress,usenames,dvipsnames]{beamer}
\usepackage[italian]{babel}
\usepackage{amsmath}
\usepackage{mathtools}
\usepackage{amstext}
\usepackage{amssymb}
\usepackage{amsthm}
%\usepackage{fullpage}
%%\usepackage{mathrsfs}
%\usepackage[utf8]{inputenc}
%%\usepackage{natbib}
\usepackage{verbatim}
\usepackage[T1]{fontenc}
%\usepackage{lmodern}
%\usepackage[toc,page]{appendix}
%\usepackage[outercaption]{sidecap}
%\usepackage{floatrow}
%\usepackage{graphicx}
%\usepackage{color}
%\usepackage{url}
%\usepackage{makeidx}
\usepackage{hyperref}
%\usepackage[usenames,dvipsnames]{xcolor}

\hypersetup{colorlinks,
            citecolor=blue,
            filecolor=black,
            linkcolor=black,
            urlcolor=blue}

%%%% ARROWSS
\usepackage{tikz}
\usetikzlibrary{shapes.arrows}

\tikzset{
    myarrow/.style={
        draw,
        fill=BurntOrange,
        single arrow,
        minimum height=3.5ex,
        single arrow head extend=0.8ex
    }
}
\newcommand{\arrowup}{%
\tikz [baseline=-0.5ex]{\node [myarrow,rotate=90] {};}
}
\newcommand{\arrowdown}{%
\tikz [baseline=-1.5ex]{\node [myarrow,rotate=-90] {};}
}
% color degli alert


%\usetheme{Madrid}
%usecolortheme{}
\usetheme[usetitleprogressbar]{m}

\setbeamercolor{alerted text}{fg=BurntOrange}

%\usepackage{booktabs}
%\usepackage[scale=2]{ccicons}
%\usepackage{minted}


%% bibliography?
\setbeamertemplate{bibliography item}{}

%\usepgfplotslibrary{dateplot}

%\usemintedstyle{trac}

\beamertemplatenavigationsymbolsempty	% NESSUN SIMBOLO DI NAVIGAZIONE 
\setbeamercovered{transparent}

\makeatletter
	\patchcmd{\beamer@sectionintoc}{\vskip 3em}{\vskip 2em}{}{}
	\patchcmd{\beamer@subsectionintoc}{\vskip 2em}{\vskip 2em}{}{}
\makeatother
\setbeamercovered{transparent}



\date{}
\title{\boldmath \bfseries \scshape 4D to 3D reduction of Seiberg duality for {\boldmath SU(N)} susy gauge theories with adjoint matter: a partition function approach}
\author{ \scshape{Carlo Sana} }
\institute{\scshape Università degli Studi di Milano-Bicocca\\
Scuola di Scienze \\
Dipartimento di Fisica "G. Occhialini"
}
\date{\scshape 29 giugno 2015}
%\logo{\includegraphics[width=1.25cm]{logounimib.jpg}}





\begin{document}


\frame{\titlepage}


\begin{frame}{Overview della presentazione}
\tableofcontents[pausesections]
\end{frame}

%\begin{frame}{Overview della tesi}
%\begin{block}{Dualità KSS}<+->
%	Generalizzazione della dualità di Seiberg con un ulteriore campo di materia nella aggiunta del gruppo di gauge $SU(N_c)$
%\end{block}
%\begin{block}{Riduzione dimensionale}<+->
%Ridurre entrambe le teorie da 4D a 3D in modo da mantenere la relazione di dualità tra di loro non può essere fatto in modo banale.\\
%\end{block}

%\begin{block}{Riduzione sulla funzione di partizione}<+->
%Effettueremo la riduzione dimensionale non usando tecniche di teoria di campo, ma calcolando la funzione di partizione e lavorando direttamente su di essa.\\
%Questo metodo ha numerosi vantaggi.
%\end{block}
%\end{frame}

\section{Dualità elettrica-magnetica e strong/weak coupling}



















\begin{frame}{Dualità elettrica-magnetica di Dirac}

Le dualità di Seiberg sono una \alert{generalizzazione} per teorie di campo \alert{supersimmetriche} \alert{ non-abeliane} della dualità di Dirac.
\begin{block}{Dualità di Dirac}
Aggiungendo sorgenti per il campo magnetico $J_{\, mag}^{\mu}$ ottengo una invarianza $\mathbb{Z}_2$ delle equazioni di maxwell sotto la trasformazione
\begin{equation*}
 \left( {E}^i, {B}^i \right) \longrightarrow \left({B}^i, - {E}^i \right)
 \qquad \left( J_{\, el}^{\mu }, J_{\, mag}^{\mu} \right) \longrightarrow \left( 
 J_{\, mag}^{\mu},  - J_{\, el}^{\mu }
 \right)
 \quad  J^{\mu} = ( \rho , J^i)
\end{equation*}
Unendo la dualità EM alla MQ si ottiene una condizione di \alert{\bfseries quantizzazione} della carica elettrica
$$
e g  = 2 \pi \hbar n
$$
\vspace{-0,3cm}
\onslide<2->{Carica elettrica e magnetica sono \alert{\bfseries inversamente} proporzionali.
\begin{center}
\alert{\bfseries \large dualità strong/weak coupling }
\end{center}
}\end{block}

\vspace{0.5cm}
\end{frame}














\begin{frame}{QFT a strong coupling}
\begin{tabular}{c l l }
Relatività speciale  & \\[0,1cm]
{\Large +} & {\Large=}  & \alert{ \bfseries \large Teoria quantistica dei campi (QFT) }\\
Meccanica Quantistica & \\
\end{tabular}
\\
\vspace{0.3cm}
Metodi {\bfseries perturbativi} utilizzabili a \alert{\bfseries weak coupling}: \\
sviluppi in serie nella costante di accoppiamento (\small e.g.\alert { carica elettrica}) \\

\begin{block}{RG flow}
Le costanti di accoppiamento variano in funzione della scala di energia:	\\
la teoria a bassa energia può fluire a \alert{ \bfseries strong coupling} \\
(e.g. confinamento in QCD)
\end{block}

\alert{\bfseries Nessuno strumento teorico per studiarne la dinamica.}
$\longrightarrow$ QCD su reticolo

\end{frame}

\plain{Esiste uno strumento teorico per trattare teorie a strong coupling?}


\begin{frame}{Strong coupling e dualità EM in QFT}
\begin{center}
 \alert{\bfseries \Large Dualità strong/weak coupling}
\end{center}
\onslide<+->{
Legame fra le costanti di accoppiamento delle teorie duali:
$$
  g \sim \frac{1}{\tilde{g}}  \quad \longrightarrow \quad \hbox{\alert{strong-weak coupling}}
$$
}

\onslide<+->{
Si può calcolare una osservabile nella teoria \alert{\bfseries fortemente accoppiata} con \alert{\bfseries tecniche perturbative} ben note nella teoria duale.\\
%Teorie fortemente accoppiate sono molto difficili da trattare.
}
\end{frame}

















\begin{frame}{Esempi di dualità strong-weak coupling}
\begin{itemize}
\item Dualità di Montonen-Oliven
\item Dualità di Seiberg 
\item AdS/CFT $\rightarrow$ gauge/gravity duality 
\item S-duality in teorie di stringa
\end{itemize}
\onslide<2->{
In particolare, con la dualità AdS/CFT (Anti-deSitter/Conformal field Theory):
\begin{itemize}
\item calcolo viscosità del quark/gluon plasma
\item modelli per superconduttori olografici (AdS$_4$/CFT$_3$)
\end{itemize}
}


\end{frame}



\section{Dualità di Seiberg}



\begin{frame}{Caratteristiche generali della dualità di Seiberg}
\begin{center}
\begin{tabular}{r c l }
 {\bfseries \Large Seiberg duality} & $\sim$ &{\bfseries \Large EM-duality}\\[0,3cm]
{ \large Teoria elettrica } & $\longleftrightarrow$ & {\large Teoria magnetica}
\end{tabular}
%
%{\bfseries Teoria elettrica} $\longleftrightarrow${\bfseries Teoria magnetica}
\end{center}
%\vspace{0,3cm}
\begin{columns}[c]
	\begin{column}{0.5 \textwidth}			
		\begin{center}
		{\bfseries Uguali} \\
		%\vspace{0,2cm}
		Funzioni di correlazione \\
		Matrice S \\
		Simmetrie Globali (\alert{\bfseries fisiche})
		\vspace{0,3cm}
		\end{center}
	\end{column}

	\begin{column}{0.5 \textwidth}
		\begin{center}
		{\bfseries Diverse}\\
	%	\vspace{0,2cm}
		Particelle \\
		Costanti di accoppiamento \\
		Dinamica \\
		Simmetrie locali (\alert{\bfseries non fisiche}) \\ 
		\end{center}
	\end{column}
\end{columns}
\begin{center}
{\large  Dualità a \alert{\bfseries basse energie}, ad alte energie \\ le due teorie sono fisicamente \alert{\bfseries distinguibili} }
\end{center}
\end{frame}















\begin{frame}{Dualità di Seiberg in 4D}

Primo caso di dualità scoperto nel '94 fra due teorie di \emph{Supersymmetric QCD} (SQCD)
\begin{table}
\begin{tabular}{l l}
Superquark \quad & $\longrightarrow$ \quad squark $\left( \hbox{spin} \;  0 \right)$ + quark$\left (\hbox{spin} \;  \frac{1}{2} \right)$ \\
Supergluone \quad &  $\longrightarrow$ \quad gaugino $\left (\hbox{spin} \;  \frac{1}{2} \right)$ + gluone  $\left( \hbox{spin} \;  1 \right)$ \\
\end{tabular}
\end{table}

\begin{block}{Perchè la supersimmetria?}<2->
Ci sono più strumenti teorici a disposizione per teorie di campo supersimmetriche,
soprattutto in regime \alert{\bfseries non perturbativo.} \\
%Le teorie di campo supersimmetriche sono più facilmente trattabili, soprattutto in regime \alert{\bfseries non perturbativo}:\\
%olomorfia, rinormalizzazione, vuoti 
\vspace{0.2cm}
%Studiare un sistema fisico più vincolato può dare \alert{\bfseries informazioni} che possono essere valide anche per sistemi \alert{\bfseries con meno simmetria}.  
Sono il terreno ideale per fare nuove scoperte, che possono essere \alert{\bfseries estese} anche a sistemi fisici non supersimmetrici.

\end{block}


\end{frame}




\begin{comment}
\begin{frame}{In cosa consiste la dualità?}
\begin{block}{}
Entrambe le teorie nello stesso intervallo di sapori e colori fluiscono a basse energie a un \alert{\bfseries punto fisso superconforme non banale:} $g \neq 0$
\\
\vspace{0.3cm}
Gli operatori \alert{\bfseries gauge invarianti} vengono mappati fra l'una e l'altra teoria
\begin{table}
\begin{tabular}{r l }
{\bfseries Seiberg duality} &  \\
$M_{el}=Q \tilde{Q} $ & $\longleftrightarrow \quad M_{mag} \neq q \tilde{q}$ \\
$B_{el} = Q_{[i} \dotsc Q_{N_c]}$ & $\longleftrightarrow \quad B_{mag} = q_{[i} \dotsc q_{N_f - N_c]}$\\
$\tilde{B}_{el} = \tilde{Q}_{[i} \dotsc \tilde{Q}_{N_c]}$ & $\longleftrightarrow \quad \tilde{B}_{mag} = \tilde{q}_{[i} \dotsc \tilde{q}_{N_f - N_c]}$\\
\end{tabular}
\end{table}
\end{block}
\end{frame}

\begin{frame}{In cosa consiste la dualità?}

{Tutte le \alert{\bfseries osservabili} della teoria elettrica possono essere calcolate nella teoria duale. \\
}\vspace{0.3cm}

\onslide<2->{\begin{center}
\alert{\bfseries \large Non ho modo di sapere se si osservano i gradi di libertà dell'una o dell'altra teoria.}
\end{center}
}

\end{frame}
\end{comment}


\begin{frame}{Dualità di Seiberg in 4D}

\begin{block}{Dualità di Seiberg con gruppo $SU(N_c)$ - [Seiberg '94] }<+->
\begin{tabular}{r l}
{\bfseries  Teoria elettrica} &   $SQCD$ con $N_c$ colori e $N_f$ sapori \\[0.1cm]
{\bfseries   Teoria magnetica} &   $SQCD$ con $N_c$ colori e $N_f$ sapori e 
 $N_f^2$  mesoni,  \\ 
 & costruiti con i \alert{\bfseries quark elettrici}\\
\end{tabular}
\end{block}

\vspace{0.5cm}


\begin{block}{Dualità KSS con gruppo $SU(N_c)$ - [Kutasov-Schwimmer-Seiberg '95]}<+->
%Scoperta nel '95 da Kutasov, Schwimmer e Seiberg \\
\begin{tabular}{r l}
{\bfseries  Teoria elettrica}  &  $SQCD$ con $N_c$ colori e $N_f$ sapori e  materia \\ &  nell'aggiunta del gruppo di gauge  \\[0.1cm]
{\bfseries  Teoria magnetica}  &  $SQCD$ con $k N_f - N_c$ colori e $N_f$ sapori, materia \\  & nell'aggiunta del gruppo di gauge  e  $k N_f^2$ mesoni, \\
&  costruiti con i campi della \alert{\bfseries teoria elettrica}.
\end{tabular}
\end{block}
\vfill 
\end{frame}



\begin{frame}{Dualità di Seiberg in 3D}
\begin{block}{Differenze delle teorie di campo 3D}
\begin{itemize}
\item diverso contenuto di materia: in 3D i gluoni hanno anche una partner \alert{\bfseries scalare} 
\item ulteriori simmetrie: in 4D no simmetria \alert{\bfseries assiale} e \alert{\bfseries topologica} 
\item uno spazio dei moduli (\alert{\bfseries vuoti supersimmetrici}) con un \emph{branch} aggiuntivo  
\end{itemize}
\end{block}

\onslide<2->{
 La teoria magnetica 3D contiene, oltre ai mesoni, un insieme aggiuntivo di singoletti.\\
 \vspace{0.3cm}
	{ \centering \alert{\bfseries Monopoli} della teoria elettrica  $\longrightarrow $ \alert{\bfseries singoletti} nella teoria magnetica\\
	}
\vspace{0.3cm}
%Essi permettono il match del \emph{Coulomb branch} fra le due teorie.
}
\end{frame}

\section{Riduzione dimensionale 4D $ \rightarrow$ 3D}

\begin{frame}{4D $ \longrightarrow $ 3D: metodo n\"{a}ive}
%A causa di queste differenze, fino al 2013 non era chiaro come ridurre le teorie da 4D a 3D in modo da preservare la dualità.
\begin{center}
{\bfseries \Large Riduzione naturale: $r \rightarrow 0$}
\end{center}
\vspace{0.5cm}
Si compattificano le teorie su un cerchio di raggio $r$: $$\mathbb{R}^4 \longrightarrow \mathbb{R}^3 \times \mathbb{S}^1$$
Si ignorano tutti i modi di Kaluza-Klein dei campi sul cerchio.\\
Infine, si manda $r \rightarrow 0 $.
\\
\vspace{0.5cm}
\onslide<2->{
	\begin{center}
	\alert{\bfseries \Large  \bfseries Con questo procedimento non si ottengono due teorie duali in 3D}
	\\
	\vspace{0.3cm}
	Limite a bassa energia incompatibile con la relazione di dualità.
	\end{center}
	}
\end{frame}

\begin{frame}[fragile]
\frametitle{4D $\longrightarrow $ 3D: metodo corretto}
\begin{center}
{\Large \bfseries Riduzione corretta: $r$ finito}\\
\vspace{0.5cm}
La finitezza del cerchio genera un termine di superpotenziale ($\eta$) dovuto a un modo istantonico di Kaluza-Klein.\\
\vspace{0.3cm}
Il superpotenziale $\eta$ impone vincoli tipici delle 4D (\alert{\bfseries anomalie}). \\
In 3D questi vincoli non ci sono e dovranno essere rimossi.
\\
%Il superpotenziale $\eta$ rompe le simmetrie 3D, anomale in 4D.\\
%Impone anche un vincolo aggiuntivo sulle cariche dei campi.

\vspace{0.3cm}
\onslide<2->{
Si ottengono teorie in 3D considerandone il limite a basse energie:
per energie $\ll \frac{1}{r}$ la dinamica sul cerchio si disaccoppia. 
\\
\vspace{0.3cm}
Il superpotenziale $\eta$ rimane anche scendendo  a basse energie.
\end{center}
}
\end{frame}


\plain{Si può arrivare a una dualità 3D senza il superpotenziale $\eta $ causato dalla presenza del cerchio?}



\begin{frame}{RG Flow verso una teoria senza superpotenziale}
\begin{block}{Dualità di Seiberg \& KSS}
Facendo un RG flow con masse reali si ottengono teorie senza il vincolo imposto dal superpotenziale $\eta$.\\
\vspace{1cm}
Si genera anche la simmetria assiale che è \alert{ \bfseries anomala in 4D}, \\ma che è permessa in 3D.
Anche gli altri vincoli imposti dal superpotenziale $\eta $ sui campi vengono rimossi.
\end{block}

\end{frame}




\begin{frame}{4D $\longrightarrow$ 3D in teoria di campo: dualità KSS}

La riduzione delle dualità è compresa in teoria di campo per diverse dualità 4D.\\
\vspace{0.3cm}
Nel caso della dualità KSS
% ( dualità di Seiberg con materia nell'aggiunta ),
è necessario deformare la teoria aggiungendo una \alert{\bfseries perturbazione} nel superpotenziale. \\
\vspace{0.3cm}
Nella teoria duale, essa rompe la teoria in $k$ settori \alert{\bfseries senza aggiunta} con gruppo di gauge 
$$
SU(k N_f - N_c)  = \prod_i^k U(n_i) \times U(1)^k / U(1)
\qquad \hbox{con} \; \sum_i n_i = k N_f - N_c
$$

\end{frame}


\begin{frame}{Riduzione dimensionale in teoria di campo}
Si può dualizzare il settore $U(1)^k$  in modo da ottenere i singoletti non presenti nella dualità 4D.
\\
\vspace{0.5cm}

\alert{\bfseries \bfseries Rimuovendo la deformazione $U(1)^k \longrightarrow U(k)$ e non si sa come si possono ottenere i singoletti della teoria magnetica.}
\onslide<2->{
\begin{center}
{\arrowdown}
\end{center}
{\large 
	Devo assumere che rimuovendo la deformazione ottengo gli stessi  singoletti.\\
}
\vspace{0.3cm}
\alert{\bfseries Non è chiaro come si può giustificare questa affermazione.}
}
\end{frame}

\plain{Ho modo di verificare se questa intuizione è corretta? }



\section{Riduzione della dualità sulla funzione di partizione}

\begin{frame}{Indice superconforme in 4D e funzioni di partizione in 3D} 
Si calcola l'indice superconforme $I_{el} \; \& \; I_{mag} $:
conta i multipletti BPS corti della teoria su $\mathbb{R}^3 \times \mathbb{S}^1$.\\
\vspace{0.2cm}
Nel limite $r \rightarrow 0$ l'indice si riduce alla funzione di partizione della teoria in 3D \alert{\bfseries con superpotenziale $\eta$.}\\
\vspace{0.2cm}
{\bfseries Indice superconf.} integrale sul gruppo di gauge di $\Gamma_e$ ellittiche
\\
{\bfseries Funz. di partiz.} integrale sul gruppo di gauge di $\Gamma_h$ iperboliche
\\[0,2cm]
\begin{equation}
\begin{array}{l l l l l }
\hbox{4D:}  \qquad  & I_{el} & =  &I_{mag}   & \Gamma_e  \\
r \rightarrow 0 &\downarrow &   &\downarrow     & \downarrow\\ 
\hbox{3D:} \qquad & Z^{\eta}_{el} & =&  Z^{\eta}_{mag} & \Gamma_h
\end{array}
\end{equation}
\end{frame}

\begin{frame}{Indici e funzioni di partizione}
La dualità in 4D (\alert{\bfseries fisicamente}) e identità integrali (\alert{\bfseries matematicamente}) dimostrano l'identità fra gli indici in 4D.\\
\begin{center}
\arrowdown \\
\end{center}
\alert{\bfseries \large \bfseries Le funzioni di partizione in 3D con superpotenziale $\eta$ sono  uguali grazie all'identità degli indici in 4D}
\end{frame}


\begin{frame}{Flow verso una teoria senza superpotenziale $\eta$}

Si fa un RG flow con masse reali direttamente sulla funzione di partizione.
\\
Con questo metodo \alert{ \bfseries non è necessario} introdurre una deformazione , a differenza che in teoria di campo. \\
%\vspace{0.2cm}
%$$
 %SU(k (N_f + 1) - N_c ) \longrightarrow U(k N_f - N_c ) \times U(k) / U(1)
%$$
\vspace{0.3cm}
Utilizzando una identità \alert{\bfseries matematica} fra gamma iperboliche $\Gamma_h$ sul settore $U(k)$ si ottengono gli \alert{\bfseries stessi singoletti} trovati in teoria di campo dualizzando $U(1)^k$.
\end{frame}


\begin{frame}{Verifica del procedimento in teoria di campo}
\begin{center}
{\large   
Il nostro lavoro è una \alert{\bfseries verifica indipendente} dei risultati ottenuti in teoria di campo, senza fare \alert{\bfseries assunzioni} che non si è in grado di giustificare.
\\[0.3cm]
\onslide<2->
{\alert{\bfseries La riduzione attraverso la funzione di partizione non era presente in letteratura per il caso SU(N).}
}
} 
\\
\vspace{0.5cm}
\onslide<3->{L'identità fra le due funzioni di partizione $ Z_{el} = Z_{mag}$ porta a una \alert{\bfseries nuova identità} integrale tra funzioni iperboliche $\Gamma_h$ \\
\alert{\bfseries non ancora dimostrate} matematicamente.
}\\
\end{center}



\end{frame}

\plain{Grazie per l'attenzione}

\bibliographystyle{apalike}
\begin{frame}{Bibliografia}
\nocite{*}
{ \footnotesize
\bibliography{biblio}
}
\end{frame}
\end{document}
