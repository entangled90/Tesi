%!TEX root = tesi.tex
%\fancyhf{}
\chapter*{\bfseries Conclusions}
%\fancypagestyle{conclusion}

In this thesis we analysed the relations between Seiberg-like dualities in four and three dimensions. 
Even though Seiberg duality was discovered over twenty years ago, it is still an active area of research.\\
In particular, in 2013 was found a method to reduce duality from four to three dimensions, unveiling the relation between dualities in different spacetime dimensions more clearly.
At the same time, it gave the possibility to find new dualities in three dimensions by reducing dualities in four dimensions.

For some dualities, the reduction process is theoretically well-understood in field theory, but for other dualities a solid proof of the reduction is still missing.\\
This is due mainly to the different matter content in the magnetic theory in three and four dimensions.
In fact, in three dimensions the magnetic theory features an additional set of singlet fields, which are mapped to the magnetic monopoles of the electric theory and interact with the magnetic monopoles of the magnetic theory through the superpotential.
These singlet fields are not present in the four dimensional magnetic theory.
\\
It is necessary to overcome this difficulty in order to find a three dimensional duality.
For example, in KSS dualities one has to introduce a deformation in the superpotential for the adjoint matter field in order to use mirror symmetry on the broken gauge group sector, obtaining the singlets fields. 
However, once the deformation is removed, mirror symmetry cannot be applied anymore since the gauge group is not broken anymore.
As a result, one has to made the assumption that enhancement of the gauge group does change the result.
This argument requires a justification which is not completely clear at the moment.

On the other hand, the reduction process can be performed by working on the superconforma index in four dimensions and then on the partition function in three dimensions.
The advantage of this argument in the case of KSS duality is that it does not require the introduction of a deformation in the superpotential.
The dualization of the gauge sector can be made by using a mathematical identity which generates terms in the partition function that have the correct charges to be interpreted as singlets.\\



