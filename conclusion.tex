%!TEX root = tesi.tex
%\fancyhf{}
\chapter*{Conclusions}
\pagestyle{conclusion}

In this thesis we analysed the relations between Seiberg-like dualities in four and three dimensions. 
Even though Seiberg duality was discovered over twenty years ago, it is still an active area of research.\\
In particular, in 2013 was found a method to reduce duality from four to three dimensions.
At the same time, it gave the possibility to find new dualities in three dimensions by the reduction of four ones.

For some dualities, the reduction process is theoretically well-understood in field theory, but for others a solid proof of the reduction is still missing.\\
This is due mainly to the different matter content in the magnetic theory between three and four dimensions.
In fact, in three dimensions the magnetic theory features an additional set of singlet fields that are mapped to the magnetic monopoles of the electric theory and that interact with the magnetic monopoles of the magnetic theory through the superpotential.
These singlet fields are not present in the original four dimensional magnetic theory.
\\
It is necessary to overcome this difficulty in order to find a three dimensional duality.
For example, in KSS dualities one has to introduce a deformation in the superpotential for the adjoint matter field in order to use mirror symmetry on the broken gauge group sector, obtaining the singlets fields. 
However, once the deformation is removed, mirror symmetry cannot be used anymore since it is not known for the initial unbroken gauge group.
As a result, one has to made the assumption that the enhancement of the 
This statement requires a more detailed explanation which isn
ot known yet.

\pagestyle{conclusion}

On the other hand, the reduction process can be performed with the partition function approach we used in previous chapter. 
The advantage of this procedure in the case of KSS duality iss that it does not require the introduction of a deformation in the superpotential.
The dualization of the gauge sector can be made by using a mathematical identity which generates terms in the partition function that have the correct charges to be interpreted as the singlet fields of Kim-Park duality.\\
In this way we overcame the difficulties encountered in field theory and obtained Kim-Park duality for $SU(N_c)$ gauge group providing an indpendent check of the results obtained with field theory arguments.


Moreover, the reduction of the duality on the partition function led us to the discovery of new integral identities between hyperbolic gamma function that are not yet demonstrated in mathematics.
As a result, this could be one of the situations where physics could provide insights into mathematics.












