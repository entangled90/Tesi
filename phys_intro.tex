%!TEX root = tesi.tex


\section{Introduction}
Supersymmetric quantum field theories enjoy an enlarged group of  symmetries compared to other field theories. 
Since the symmetry group is a non-trivial combination of internal and spacetime symmetries, they have many unexpected features and new techniques were found to study them.
Almost all of the new tools found are available only for supersymmetric field theories, making them the theatre for many advances in physics. 

A more technical introduction on supersymmetry and its representation on fields can be found in appendix \ref{appendice_susy}.

In this section we will analyse more advanced features of supersymmetric field theories that have been used intensively in the discovery and in the analysis of electric magnetic duality and its generalisations.














\subsection{General renormalization properties}

A remarkable feature of supersymmetry is the constraint that the additional symmetry imposes on the renormalization properties of the theories.
\\
One of the first aspects that brought attention to supersymmetry was that divergences of loop diagrams were milder because of the cancellation between diagrams with bosons and fermions running in the loops. 
\\
Nowadays we know powerful theorems that restrict the behaviour of supersymmetric field theories under renormalization.
In order to preserve supersymmetry, the renormalization process has to preserve the Hilbert space structure. For example the wave function renormalization of different \emph{particles} inside a multiplet must be the same, otherwise the renormalized lagrangian is not supersymmetric invariant anymore. 
\\
Moreover, in the supersymmetry algebra $P^2$ is still a Casimir operator i.e. it commutes with every operator in the algebra: particles in the same multiplet must have the same mass.
Renormalization cannot break this condition, otherwise it would break supersymmetry.
\\
For a \emph{Super Yang Mills} theory with $\mathcal{N} = 1$ symmetry considerations lead to the additional requirement that $g V$, where $g$ is the coupling and $V$ is the vector superfield, cannot be renormalized. 
%This is equivalent to require that  $ Z_g = Z_V^{-1}$.
\\
Adding more supersymmetry the wave function renormalization of the various fields are even more constrained by symmetry.
For example, for $\mathcal{N}=4$ \emph{SYM} the coupling constant is not renormalized at all.

















\subsubsection{Beta function for SYM and SQCD}
Another nice feature of supersymmetric field theories is that some quantities can be calculated exactly.
The first object of this kind that we encounter is the $\beta$-function of four dimensional $\mathcal{N} =  1 $ \emph{Super Yang Mills} theories with matter fields in representations $R_i$.
\\
It is given by the exact \emph{NSVZ $\beta$-function} 
\begin{equation}
  \beta (g) = \mu \, \frac{\mathrm{d} \, g}{\mathrm{d}\, \mu} = - \frac{g^3}{16 \pi^2} \left[ 3 \; T(Adj) - \sum_i T( R_i) ( 1 - \gamma_i ) \right]  \left( 1 - \frac{ g^2 \; T(Adj)  }{8 \pi^2} \right)^{-1}
\label{beta-exact}
\end{equation} 
where $\gamma_i$ are the anomalous dimensions of the matter fields and $T(R_i)$ are the Dynkin indices\footnote{The Dynkin index $T(R) $ of a representation $R$ is defined as $\Tr( T^a T^b) = T(R) \delta^{ab} 
$ where $T^a \, , \, T^b$ are the generators of the algebra in the representation $R$. } of their representation.\\
The anomalous dimensions are defined as
\begin{equation}
 \gamma_i = - \mu \, \frac{ \mathrm{d} \, \log(Z_i) }{\mathrm{d} \, \mu}
\end{equation}
where $Z_i$ is the wave function renormalization coefficient. The Dynkin indices of the gauge group $SU(N)$ for the fundamental and adjoint representation are 
\begin{equation}
 T(\mathrm{N}) = \frac{1}{2} \qquad T( \mathrm{Adj}) = N 
\label{eqn:suN_Dynkin_indices}
\end{equation}
The \emph{NSVZ $\beta$-function} was first calculated using instanton methods in \cite{Novikov:1985rd}.



\begin{comment}
Over the years it has been calculated in other ways using the fact that the action is holomorphic in the complexified coupling
\begin{equation}
	%\frac{1}{g_h^2}
	 \tau = \frac{4 \pi i }{g_c^2 } + \frac{\theta_{YM}}{2 \pi} 
\end{equation}
Using the holomorphic coupling the action for the vector field is written as
\begin{equation}
 \mathcal{L}_h ( V_h) = \frac{1}{16 \pi i } \int d^2 \theta \; \tau \; W^a ( V_h) W^a(V_h)  + h.c.
\end{equation}
whereas with the canonical normalization for the vector field is
\begin{equation}
 \mathcal{L}_c ( V_c) = \frac{1}{16 \pi i } \int d^2 \theta \left( \frac{4 \pi i }{g_c^2 } + \frac{\theta_{YM}}{2 \pi} \right) W^a ( g_c V_c) W^a(g_c V_c) + h. c.  
\end{equation}
Using the canonical normalization $g_c V_c$ is a real superfield, imposing that $g_c$ is real.
For this reason with the canonical normalization the lagrangian is not holomorphic in $\tau$.
Thanks to holomorphicity, the holomorphic coupling is only renormalized at one-loop and the $\beta$-function can be computed exactly at one loop but its expression is different from \emph{NSVZ $\beta$-function}.
The cause of this mismatch is that the \emph{NSVZ $\beta$-function} is defined using the canonical (or physical) coupling constant and receives contribution from all orders in perturbation theory.

At first sight, one should expect that the expressions should match since the first two orders in $\alpha$ of the $\beta$-function are scheme independent. 
The reason why the two expressions differ is that the Jacobian of the transformation between canonical and holomorphic normalization is anomalous. 
Once the anomaly is taken into account the two expressions for the $\beta$-function agree. \\
An explicit calculation relating the comparison between the two different approach can be found in \cite{ArkaniHamed:1997mj}.
\end{comment}


















\subsection{Superpotential: holomorphy and non-renormalization}
\label{sec:superpotential_hol_renorm}

Other than renormalization constraints, supersymmetry provides non-renormalization theorems for certain objects, such as the superpotential. 
\\
In \cite{Grisaru:1979wc} it has been demonstrated that the superpotential is tree-level exact, i.e. it does not receive correction in perturbation theory. 
However it usually receive contributions from nonperturbative dynamics.
The superpotential features this property in theories with at least four supercharges and can be demonstrated independently using perturbative calculations or its holomorphic properties. 
\\
It was first demonstrated perturbatively, using the fact that for general supersymmetric field theories, supergraph loops diagrams with $n$ external leg yield a term that can be written in the form
\begin{equation}
\int  d^4 x_1 \dots d^4 x_n d^2 \theta d^2 \bar{\theta} \; G (x_1 , \dots , x_n) \,F_1 ( x_1, \theta, \bar{\theta}) \dots  F_n ( x_n, \theta, \bar{\theta}) 
\end{equation} 
where $F_i$ are given by products of superfields and their covariant derivatives and $G (x_1 , \dots , x_n) $ is translationally invariant function.\\
The importance of this result is that all contribution from Feynman diagrams are given by a single integral over full superspace ($d^2 \theta d^2 \bar{\theta} $) whereas the superpotential must be written as an integral in half-superspace($d^2 \theta $ only) of chiral fields.
Exploiting the fact that a product of chiral fields is a chiral field, the most general form of a superpotential is 
\begin{equation}
 W (\lambda, \Phi) = \sum_{n=1}^{\infty} \left( \int d^2 \theta \, \lambda_n \Phi^n  \; +\;  \int d^2 \bar{\theta} \, \lambda_n^{\dagger} \bar{\Phi}^n \right)
 %\int d^2 \bar{\theta} \bar{\lambda}_n {\bar{\Phi}}^n
 \end{equation} 
 The second term of the superpotential is added in order to give a real lagrangian after the integration in superspace.
From the definition, we can see that the superpotential is holomorphic in the fields and in the coupling constants.
\\
An alternative proof of this theorem was provided by Seiberg \cite{Seiberg:1993vc} using a different approach.
He noted that the coupling constants $\lambda_n$ can be treated as background chiral superfields with no dynamics.
\\
Using this observation we can assign transformation laws to the coupling constants, making the lagrangian invariant under a larger symmetry.
Fields and coupling constants are charged under this symmetry and only certain combinations of them can appear in the superpotential.
%The superpotential in the weak coupling limit is subject to the constraints imposed by holomorphy and symmetry.
In addition, in a suitable weak coupling limit the effective superpotential must be identical with the tree-level one.
These conditions, taken together, fix the expansion of the superpotential to the expression of the tree-level potential.
A more detailed discussion can be found in \cite{Seiberg:1994bp} and \cite{Intriligator:1995au}.






\subsection{Moduli space}
\label{sec:subsection_moduli_space}
The \emph{classical moduli space} is the set of vacuum expectation values of the scalar fields that correspond to a zero of the scalar potential.
The zeroes of the scalar potential can be found by solving to two different sets of equations, called $D-term$ and $F-term $ equations
\begin{equation}
\bar{F}^i ( \phi) = 0 \qquad D^a (\phi , \bar{\phi}) = 0
\end{equation}
\emph{F-term} equations are present only if there is a superpotential while the \emph{D-term} equations are associated to the kinetic part of the lagrangian.\\
If the minimum of the scalar potential is different from zero the vacuum is not supersymmetric. 
In this case supersymmetry is spontaneously broken.
Another possible situation is that the scalar potential has no minimum at all: the theory does not have any stable vacua.\\
If it is not possible to find vacuum expectation values that correspond to a zero in the scalar potential, then the theory does not posses a moduli space.\\
Gauge transformations should be taken into account in order to avoid redundance in the description.
The moduli space describes physically inequivalent vacua, since the mass spectrum of the theory depends on the \emph{VEVs} of the scalar fields, which differ in every point of the moduli space. \\
Because of supersymmetry, radiative corrections do not lift the energy of the ground state and the vacuum remains supersymmetric.
As a result, only superpotentials generated from nonperturbative dynamics can lift the moduli space. 
We will see examples of this phenomenon in the analysis of SQCD.
\\
An alternative description of the space of classical \emph{D-flat} directions is given by the space of all holomorphic gauge invariant polynomials of scalar fields modulo classical relations between them \cite{Luty:1995sd}.
As a result, gauge invariant polynomials of operators parametrize the classical moduli space of the theory. 
Using this description it's easier to find the moduli space of the theory in consideration.
If a superpotential is present, \emph{F-term} equations should be imposed on the gauge invariant polynomial used to describe \emph{D-flat} direction.
We will use this convenient description in the next chapters.

\subsubsection{Quantum moduli space}
The quantum-corrected moduli space is in general different from the classical one, altough there are situations in which they coincide, as we will see in the next sections.\\
The quantum moduli space can be found by searching for the minima of  the Wilsonian low-energy effective action.
In general the effects of renormalization on the Kahl\"{e}r potential do not modify  the geometry of the moduli space.\\
On the other hand, non-perturbative dynamics generates in general non-trivial terms in the superpotential that modify the moduli space.
The non-perturbative term in the superpotential can be found by considering the the symmetries of the theory and the holomorphy of the superpotential.
In most situations, they constrain its form in such a way that it can be found by using solely symmetry considerations.
However, in some cases they are not enough to fix its expression and we don't have a description of the quantum moduli space.
\\
If a non-trivial effective superpotential is generated, it will usually lift part of the moduli space or modify its geometry, removing the singular point present in the classical moduli space.
The singularities in the classical moduli space are located in point of enhanced gauge symmetries or where are present massless degrees of freedom such as in the origin of the moduli space.
The quantum moduli space does not have such singularities and it is in fact a smooth manifold.



\subsection{Phases of gauge theories}
The dynamics of gauge theories can be classified according to the low-energy effective potential $V(R)$ between two test charges separated by a large distance $R$.
The possible forms of the potential, up to addictive constant, are
\begin{align}
\text{Coulomb} \qquad V(R) & \sim \frac{1}{R} \\
\text{free electric} \qquad V(R) & \sim \frac{1}{R \, \log(R \Lambda) } \\
\text{free magnetic} \qquad V(R) & \sim \frac{\log(R \Lambda)}{R \,  }\\
\text{Higgs} \qquad V(R) & \sim \text{constant} \\
\text{confining} \qquad V(R) & \sim  \sigma \, R  
\end{align}
The first three phases feature massless gauge fields and their potential is $V(R) \sim g^2(R) / R$ and they differ because of the renormalization of the charge in the infrared.
In the Coulomb phase, $g^2_{IR} = \text{constant}$, while in the free abelian/non-Abelian phase the coupling constant goes to zero as $g^2(R) \sim 1/ \log(R \, \Lambda)$.
The free electric phases is possible for abelian or non-Abelian theories.
In the latter case for asymptotically free theories it is necessary that the renormalization group has a non-trivial infrared fixed point.
The free magnetic phases is generated by magnetic monopoles acting as source of the field. Since magnetic and electric charges are related by Dirac quantization condition, the running of the coupling constant for magnetic monopoles is the inverse of electric charges.\\
The situation is completely different in the last two cases.
In the Higgs phase gauge fields are massive and the potential is given by a Yukawa potential, exponentially suppressed at long distances that results in a constant value.
The confining phase can be described by tube of confined gauge flux between the charges which, at large distances, acts as a string with costant tension, yielding a linear potential.


\subsection{'t Hooft anomaly matching conditions}

The 't Hooft anomaly matching conditions are a great tool to investigate the global symmetries of the low-energy degrees of freedom of the theory.\\ 
Let's consider an asymptotically free gauge theory with global symmetry group $G$. 
While gauge symmetries cannot be anomalous because that would spoil the unitarity of the theory, this condition does not apply to anomalies involving only global symmetries.\\
We can compute the triangle anomaly for the global symmetry group in the ultraviolet and we will call it $A_{UV}$. 
After weakly gauging $G$ we introduce additional fermions that are charged only under $G$ in order to cancel the anomaly, since now it is a gauge symmetry.\\
Flowing towards the infrared, the anomaly is still zero if the global symmetry group is not broken.
After constructing the low-energy effective field theory, 
we can calculate the triangle anomalies for the group $G$ involving the composite low energy fields, which results in the term $A_{IR}$.
The contribution of the additional fermions remains identical since they are massless.
\begin{equation}
 0 = A_{IR} + A_{F} = A_{UV} + A_{F} \quad \rightarrow \quad A_{IR} = A_{UV}
\end{equation}
The anomaly coefficient can be easily computed since it is proportional to the group theoretical factor
\begin{equation}
 A = \Tr \; ( T^a \{ T^b, T^c \})
\end{equation}
Summarizing the result, we found that if the global symmetry group is not broken by the strong dynamics, triangle anomalies involving only the global symmetry group should be equal in the ultraviolet and in the infrared.\\
We will use these anomaly matching conditions to find if two dual theories are invariant under the same global symmetries in the IR as an additional check of electric-magnetic duality.


