%!TEX root = tesi.tex
\begin{lstlisting}
---- INTRODUCTION OUTLINE-----
~ More symmetry = more tools for studying t\heories
~ State structure: multiplet & superspace
~ Milder divergences 
~ Renormalization constraints
~ Non renormalization theorems (perturbative)
~ Holomorphicity, couplings as background fields 
  (important smoothness of weak coupling limits, e.g. classic imit g in  well defined.
~ Exact results (superpotential, witten index , exact beta function )
~Moduli space
\end{lstlisting}

\section{Introduction}
Supersymmetric quantum field theories enjoy an enlarged group of  symmetries compared to other field theories. 
Since the symmetry group is a non trivial combination of internal and spacetime symmetries, they have many unexpected features and new techniques were found to study them.
%%On top of that \emph{superstring theory} provided many insights and explanations that were not clear from a field theory perspective only. 
Almost all of the new tools found are available only for supersymmetric field theories, making them the theater for many advances in physics. 

In this section we will analyse the features of supersymmetric field theories that are crucial to the discovery of electric magnetic duality and its generalisations.

\subsection{States and their representation}

Being a symmetry between bosons and fermions, supersymmetry imposes that states are organized in multiplets containing different representations of the \emph{Lorentz group} i.e different type of  particles.
Various multiplets exist and their properties depend on the number of supercharges of the theory and on what they represent e.g matter, glue or gravity. 

We will introduce the multiplets that can be defined for $4d \; \mathcal{N} = 1$ theories and only later we will explain the differences with $3d \; \mathcal{N} = 2$ theories.
For four dimensional theories, we can define two different multiplet that are invariant under supersymmetry transformation.
The matter or chiral multiplet contains a complex scalar (\emph{squark}) and a Weyl fermion. It identifies the matter content of the theory.
The vector or gauge multiplet contains a Weyl fermion (\emph{gaugino}) and a vector.
Particles in the same multiplet transform in the same representation of global or gauge symmetries. For this reason the gaugino cannot represent matter.

%%The theories we will consider will not contain particle with spin greater than one.
%%This requirement imposes that no other multiplets can be present in these kind of theories.

A representation of these multiplets on fields can be easily found using the \emph{superspace} formalism.
In this formalism it is possible to represent fields  that are \emph{off-shell}, in contrast with multiplets that we introduced previously that are \emph{on-shell} since they represent states in Hilbert space.\\
Matter and gauge multiplets are represented by (anti)chiral and real superfields respectively.



\subsection{General renormalization properties}

A remarkable feature of supersymmetry is the constraint that the additional symmetry imposes on the renormalization properties of the theories.

One of the first aspects that brought attention to supersymmetry was that divergences of loop diagrams were milder because of the cancellation between diagrams with bosons and fermions running in the loops. 
%%This feature was thought to provide a nice way of solving the \emph{hierarchy problem} which is related to the fine tuning of UV-scale that is needed in order to keep under control radiative correction for the Higgs mass.
%The mass correction in the \emph{Standard Model} is quadratic in the cut-off scale and with the addition of supersymmetry it becomes a protected, non renormalized quantity. 
%On the other hand, the Higgs mass is still a parameter in the lagrangian whose value is not fixed by anything else.
%For this reason this is not believed to be the actual solution to the \emph{hierarchy problem.} 

Nowadays we know powerful theorems that restrict the behaviour of supersymmetric field theories during renormalization.
In order to preserve supersymmetry, the renormalization process has to preserve the Hilbert space structure. For example the wave function renormalization of different \emph{particles} inside a multiplet must be the same, otherwise the renormalized lagrangian is not supersymmetric invariant anymore. 

Moreover, in the supersymmetry algebra $P^2$ is still a Casimir operator i.e. it commutes with every operator in the algebra: particles in the same multiplet must have the same mass.
Renormalization cannot break this condition, otherwise it would break supersymmetry.

For a \emph{Super Yang Mills} theory with $\mathcal{N} = 1$ we have the additional requirement that $g V$, where $g$ is the coupling and $V$ is the vector superfield, cannot be renormalized by symmetry considerations. 
%This is equivalent to require that  $ Z_g = Z_V^{-1}$.

Adding more supersymmetry the wave function renormalization of the various field are even more constrained by symmetry.
For example, for $\mathcal{N}=4$ \emph{SYM} the fields and the coupling are not renormalized at all.


\subsubsection{Beta function for SYM and SQCD}
Another nice feature of supersymmetric field theories is that some quantities can be calculated exactly.
The first object of this kind that we encounter is the $\beta$ function of four dimensional $\mathcal{N} =  1 $ \emph{Super Yang Mills} and \emph{Super QCD} theories.
With more supersymmetry the formula can be used, with due modifications.

It is given by the \emph{NSVZ $\beta$ function} 
\begin{equation}
  \beta (\alpha) = - \frac{\alpha^2}{2 \pi} \left[ 3 \; T(Adj) - \sum_i T( R_i) ( 1 - \gamma_i ) \right]  \left( 1 - \frac{ \alpha \; T(Adj)  }{2 \pi} \right)^{-1} \qquad \alpha = \frac{g^2}{4 \pi}
\label{beta-exact}
\end{equation} 
where $\gamma_i$ are the anomalous dimensions of the matter fields and $T(R_i)$ are the dynkin indices of their representation.\\
The anomalous dimensions are defined as
\begin{equation}
 \gamma_i = - \frac{d \, \log(Z_i) }{d \, \log( \mu)}
\end{equation}
where $Z_i$ is the wave function renormalization coefficient.
For example for gauge group $SU(N)$ we have
\begin{equation*}
 T(N) = \frac{1}{2} \qquad T(Adj) = N 
\end{equation*}
The \emph{NSVZ $\beta$ function} was first calculated using instanton methods in \cite{Novikov:1985rd}. 
Over the years it has been calculated in other ways using the fact that the action is holomorphic in the complexified coupling
\begin{equation}
	\frac{1}{g_h^2} = \frac{1}{g^2 } + i \frac{\theta}{8 \pi^2} 
\end{equation}
Using the holomorphic coupling the action for the vector field is written as
\begin{equation}
 \mathcal{L}_h ( V_h) = \frac{1}{16} \int d^2 \theta \frac{1}{g_h^2} W^a ( V_h) W^a(V_h) 
\end{equation}
whereas with the canonical normalization for the vector field is
\begin{equation}
 \mathcal{L}_c ( V_c) = \frac{1}{16} \int d^2 \theta \left( \frac{1}{g_c^2}  +i \frac{\theta}{8 \pi^2} \right) W^a ( g_c V_c) W^a(g_c V_c) 
\end{equation}
Using the canonical normalization $g_c V_c$ is a real superfield, imposing that $g_c$ is real. For this reason with the canonical normalization the lagrangian is not holomorphic in the combination $ \frac{1}{g^2 } + i \frac{\theta}{8 \pi^2} $.
Thanks to holomorphy, the holomorphic coupling is only renormalized at one-loop and the $\beta$ function can be computed but its expression is different from \emph{NSVZ $\beta$ function}.
In fact, the \emph{NSVZ $\beta$ function} is defined using the canonical (or physical) coupling constant and receives contribution from all orders in perturbation theory.

At first sight, one should expect that the expressions should match since the first two orders in $\alpha$ of the $\beta$ function are scheme independent. 
The reason why the two expressions differ is that the Jacobian of the transformation between canonical and holomorphic normalization is anomalous. 
Once the anomaly is taken into account the two expressions for the $\beta$ function agree \cite{ArkaniHamed:1997mj}.


\subsection{Superpotential: holomorphy and non-renormalization}
%Even though supersymmetry imposes strong constraints on the renormalization properties of a theory we will be more interest in theorems that prohibit the renormalization of some objects in the theory.
Other than renormalization constraints, supersymmetry provides non-renormalization theorems for certain objects, such as the superpotential.
In \cite{Grisaru:1979wc} it has been demonstrated that the superpotential is tree-level exact, i.e. it does not receive correction in perturbation theory. 
However it usually receive contributions from non perturbative dynamics.

Perturbative calculations can be done using supergraphs, i.e. Feynman diagrams with superfields. 
The advantage of this approach is that supersymmetry is explicit and many semplification occur naturally. 
The demonstration is based on the fact that for general supersymmetric field theories, supergraph loops diagrams yield a term that can be written in the form
\begin{equation}
\int  d^4 x_1 \dots d^4 x_n d^2 \theta d^2 \bar{\theta} \; G (x_1 , \dots , x_n) \,F_1 ( x_1, \theta, \bar{\theta}) \dots  F_n ( x_n, \theta, \bar{\theta}) 
\end{equation} 
where $G (x_1 , \dots , x_n) $ is translationally invariant function.\\
The importance of this result is that all contribution from Feynman diagrams are given by a single integral over full superspace ($d^2 \theta d^2 \bar{\theta} $) whereas the superpotential must be written as an integral in half-superspace($d^2 \theta $ only) of chiral fields.
Exploiting the fact that a product of chiral fields is a chiral field, the most general form of a superpotential is the following
\begin{equation}
 W (\lambda, \Phi) = \sum_{n=1}^{\infty} \left( \int d^2 \theta \, \lambda_n \Phi^n  \; +\;  \int d^2 \bar{\theta} \, \lambda_n^{\dagger} \bar{\Phi}^n \right)
 %\int d^2 \bar{\theta} \bar{\lambda}_n {\bar{\Phi}}^n
 \end{equation} 
 The second term of the superpotential is added in order to give a real lagrangian after the integration in superspace.
From the definition, we can see that the superpotential is holomorphic in the fields and in the coupling constants.

Fifteen years later, Seiberg \cite{Seiberg:1993vc} provided a proof of this theorem using a different approach.
He noted that the coupling constants $\lambda_n$ can be treated as background fields, i.e. chiral superfields with no dynamics.
%This situation is similar to string theory where there are no parameters except the string tension and every other coupling or mass is given by an expectation value of a scalar field. 
%The most immediate example is that of the dilaton, whose \emph{vev} governs the perturbative expansion of the theory.



%% DAVEDERE

Using this observation we can assign transformation laws to the coupling constants, making the lagrangian invariant under a bigger symmetry.
Fields and coupling constants are charged under this symmetry and only certain combinations of them can appear in the superpotential.
%The superpotential in the weak coupling limit is subject to the constraints imposed by holomorphy and symmetry.
In addition, in a suitable weak coupling limit the effective superpotential must be identical with the tree-level one.
These conditions, taken together, fix the expansion of the superpotential to the expression of the tree-level potential.
A more detailed discussion can be found in \cite{Seiberg:1994bp} and \cite{Intriligator:1995au}



\subsection{Moduli space}



\subsection{Other exact results}







