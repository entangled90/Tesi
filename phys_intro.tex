%!TEX root = tesi.tex
\begin{lstlisting}
---- INTRODUCTION OUTLINE-----
~ More symmetry = more tools for studying theories
~ State structure: multiplet & superspace
~ Milder divergences 
~ Renormalization constraints
~ Non renormalization theorems (perturbative)
~ Exact results (superpotential, witten index , exact beta function )
~ Holomorphicity, couplings as background fields 
  (important smoothness of weak coupling limits, e.g. classic imit g in  well defined.
~Use of wilsonian action: no IR divergences)
~Superconformal group more relations, r charge, dimensions ecc
~Superconformal index
~Localization
~Moduli space
\end{lstlisting}

\section{Introduction}
Supersymmetric quantum field theories enjoy an enlarged group of  symmetries compared to other field theories. 
Since the symmetry group is a non trivial combination of internal and spacetime symmetries, they have many unexpected features and new techniques were found to study them.
On top of that \emph{superstring theory} provided many insights and explanations that were not clear from a field theory perspective only. 
Almost all of the new tools found are available only for supersymmetric field theories, making them the theater for exciting discoveries in physics.

In this chapter we will analyse the new features of supersymmetric field theories that made possible the discovery and the analysis of electric magnetic duality and its generalisations.

\subsection{ States and their representation}

Being a symmetry between bosons and fermions, supersymmetry imposes that states are organized in multiplets containing different representations of the \emph{Lorentz group} i.e different type of  particles.
Various multiplets exist and their properties depend on the number of supercharges of the theory and on what they represent e.g matter, glue or gravity. 

We will introduce the multiplets that can be defined for $4d \; \mathcal{N} = 1$ theories and only later we will explain the small differences with $3d \; \mathcal{N} = 2$ theories.
For four dimensional theories, we can define two different multiplet that are invariant under supersymmetry transformation.

The matter or chiral multiplet contains a complex scalar and a weyl fermion. Its name indicates that it is used for the matter content of the theory (quark and squark for example).

The vector or gauge multiplet contains a Weyl fermion and a vector. 
Notice that in order to preserve supersymmetry the Weyl fermion has to transform in the same representation of the vector, i.e. the adjoint representation.
For this reason the fermion in this multiplet does not represent matter and in fact is called \emph{gaugino} .

The theories we will consider will not contain particle with spin greater than one.
This requirement imposes that no other multiplets can be present in these kind of theories.

A representation of these multiplets on fields can be easily found using the \emph{superspace} formalism.
In this formalism it is possible to represent fields  that are \emph{off-shell}, in contrast with multiplets that we introduced previously that are \emph{on-shell} since they represent states in Hilbert space.\\
Matter and gauge multiplets are represented by (anti)chiral and real superfields respectively.



\subsection{ General renormalization properties}

A remarkable feature of supersymmetry is the constraints that the additional symmetry imposes on the renormalization properties of the theories.

One of the first aspects that brought attention to supersymmetry was that divergences coming from loop diagrams were milder because of the cancellation between diagrams with bosons and fermions running in the loops. 
Nowadays we know more powerful theorems that restrict the behaviours of supersymmetric field theories under the process of renormalization.

In order to preserve supersymmetry, the renormalization process has to preserve the Hilbert space structure. For example the wave function renormalization of different \emph{ particles} inside a multiplet must be the same, otherwise the renormalized lagrangian is not supersymmetric invariant anymore. 

Moreover, in the supersymmetry algebra $P^2$ is still a Casimir operator i.e. it commutes with every operator in the algebra: particles in the same multiplet must have the same mass.
Renormalization cannot break this condition, otherwise it would break supersymmetry.

For a \emph{Super Yang Mills} theory with $\mathcal{N} = 1$ we have the additional requirement that $g V$, where $g$ is the coupling and $V$ is the vector superfield, cannot be renormalized by symmetry considerations. This is equivalent to require that 
$ Z_g = Z_V^{-1}$.

Adding more supercharges the wave function renormalization of the various field are even more constrained by symmetry.
For $\mathcal{N}=4$ \emph{SYM}, which is the  theory with maximum amount of supersymmetry in four dimensions without gravity, the fields and the coupling are not renormalized at all.
This means that the $\beta$ function of the theory is zero and the theory is invariant under dilatations also at the quantum level.
Indeed the theory has a even larger symmetry: it is superconformal invariant.

The superconformal group is the supersymmetric extension of the classical conformal group and because of this larger symmetry it constrains even more the dynamics of the theory. 
We will encounter other superconformal theories and we will use some of their properties while studying electric magnetic duality. 


\subsection{Non renormalization}



\newpage