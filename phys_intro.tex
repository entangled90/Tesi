%!TEX root = tesi.tex
\begin{comment}
	---- OUTLINE-----
	***Supersimmetria:
		~ + simmetria = + strumenti per studiare teorie
		~Non renormalization theorems (perturbative)
		~ milder divergence 
		~exact results (e.g. superpotential, witten index , exact beta function )
		~Holomorphicity, couplings as background fields (important is smoothness of weak coupling limits, e.g. classic limit g in 0 well defined.
		Use of wilsonian action: no IR divergences)
		~moduli spaces
		--Superconformal group-- more relations, r charge, dimensions ecc
		~brane constructions?
		~Superconformal index
		~Localizations

\end{comment}
\section{Introduction}
Supersymmetric quantum field theories enjoy an enlarged group of  symmetries compared to other field theories. 
Since the symmetry group is a non trivial combination of internal and spacetime symmetries, they have many unexpected features and new techniques were found to study them.
On top of that \emph{superstring theory} provided many insights and explanations that were not clear from a field theory perspective only. 
Almost all of the new tools found are available only for supersymmetric field theories, making them the theater for exciting discoveries in physics.


\subsection{ Renormalization (and non-renormalization)}

A remarkable feature of supersymmetry is the constraints that imposes on the renormalization properties of the theories.

One of the first aspects that brought attention to supersymmetry was that divergences coming from loop diagrams were milder because of the cancellation between diagrams with bosons and fermions running in the loops. 
Even if this looks like a very promising feature, now it is a property that is exploited by even more powerful theorems about the renormalization properties of the theory.


\subsection{Renormalization}
Being a symmetry between bosons and fermions, supersymmetry imposes that states need be organized in multiplets containing different representations of the \emph{Lorentz group} such as scalars and spinors for example. Different multiplets exist and their properties depend on their explicit construction and on the number of supercharges of the theory.


In order to preserve supersymmetry, the renormalization process has to preserve the Hilbert Space structure i.e. for example the wave function renormalization of different \emph{ particles} inside a multiplet must be the same. 
With the same reasoning, the masses of different particles of the same multiplet must be the same (supersymmetry algebra requires this explicitly). 