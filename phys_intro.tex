%!TEX root = tesi.tex



\begin{lstlisting}
----  INTRODUCTION OUTLINE  -----
~ More symmetry = more tools for studying t\heories
~ Milder divergences 
~ Renormalization constraints
~ Non renormalization theorems (perturbative)
~ Holomorphicity, couplings as background fields 
~ Exact results (superpotential, exact beta function )
~ Moduli space
\end{lstlisting}





\section{Introduction}
Supersymmetric quantum field theories enjoy an enlarged group of  symmetries compared to other field theories. 
Since the symmetry group is a non trivial combination of internal and spacetime symmetries, they have many unexpected features and new techniques were found to study them.
Almost all of the new tools found are available only for supersymmetric field theories, making them the theater for many advances in physics. 

A more technical introduction on supersymmetry and its representation on fields can be found in appendix \ref{appendice_susy}.

In this section we will analyse more advanced features of supersymmetric field theories that has been used intensively in the discovery and in the analysis of electric magnetic duality and its generalisations.














\subsection{General renormalization properties}

A remarkable feature of supersymmetry is the constraint that the additional symmetry imposes on the renormalization properties of the theories.

One of the first aspects that brought attention to supersymmetry was that divergences of loop diagrams were milder because of the cancellation between diagrams with bosons and fermions running in the loops. 

Nowadays we know powerful theorems that restrict the behaviour of supersymmetric field theories during renormalization.
In order to preserve supersymmetry, the renormalization process has to preserve the Hilbert space structure. For example the wave function renormalization of different \emph{particles} inside a multiplet must be the same, otherwise the renormalized lagrangian is not supersymmetric invariant anymore. 

Moreover, in the supersymmetry algebra $P^2$ is still a Casimir operator i.e. it commutes with every operator in the algebra: particles in the same multiplet must have the same mass.
Renormalization cannot break this condition, otherwise it would break supersymmetry.

For a \emph{Super Yang Mills} theory with $\mathcal{N} = 1$ we have the additional requirement that $g V$, where $g$ is the coupling and $V$ is the vector superfield, cannot be renormalized by symmetry considerations. 
%This is equivalent to require that  $ Z_g = Z_V^{-1}$.

Adding more supersymmetry the wave function renormalization of the various field are even more constrained by symmetry.
For example, for $\mathcal{N}=4$ \emph{SYM} the fields and the coupling are not renormalized at all.

















\subsubsection{Beta function for SYM and SQCD}
Another nice feature of supersymmetric field theories is that some quantities can be calculated exactly.
The first object of this kind that we encounter is the $\beta$ function of four dimensional $\mathcal{N} =  1 $ \emph{Super Yang Mills} and \emph{Super QCD} theories.

It is given by the \emph{NSVZ $\beta$ function} 
\begin{equation}
  \beta (g) = \mu \, \frac{\mathrm{d} \, g}{\mathrm{d}\, \mu} = - \frac{g^3}{16 \pi^2} \left[ 3 \; T(Adj) - \sum_i T( R_i) ( 1 - \gamma_i ) \right]  \left( 1 - \frac{ g^2 \; T(Adj)  }{8 \pi^2} \right)^{-1} \qquad \alpha = \frac{g^2}{4 \pi}
\label{beta-exact}
\end{equation} 
where $\gamma_i$ are the anomalous dimensions of the matter fields and $T(R_i)$ are the Dynkin indices of their representation.\\
The anomalous dimensions are defined as
\begin{equation}
 \gamma_i = - \mu \, \frac{ \mathrm{d} \, \log(Z_i) }{\mathrm{d} \, \mu}
\end{equation}
where $Z_i$ is the wave function renormalization coefficient. In general, anomalous dimensions are not know exactly.\\
The Dynkin index
\footnote{The Dynkin index $T(R) $ of a representation $R$ is defined as $\Tr( T^a T^b) = T(R) \delta^{ab} 
$ where $T^a \, , \, T^b$ are the generators of the algebra in the representation $R$. }
 of the gauge group $SU(N)$ for the fundamental and adjoint representation are 
\begin{equation}
 T(\mathrm{N}) = \frac{1}{2} \qquad T( \mathrm{Adj}) = N 
\label{eqn:suN_Dynkin_indices}
\end{equation}
The \emph{NSVZ $\beta$ function} was first calculated using instanton methods in \cite{Novikov:1985rd}. 
Over the years it has been calculated in other ways using the fact that the action is holomorphic in the complexified coupling
\begin{equation}
	%\frac{1}{g_h^2}
	 \tau = \frac{4 \pi i }{g_c^2 } + \frac{\theta_{YM}}{2 \pi} 
\end{equation}
Using the holomorphic coupling the action for the vector field is written as
\begin{equation}
 \mathcal{L}_h ( V_h) = \frac{1}{16 \pi i } \int d^2 \theta \; \tau \; W^a ( V_h) W^a(V_h)  + h.c.
\end{equation}
whereas with the canonical normalization for the vector field is
\begin{equation}
 \mathcal{L}_c ( V_c) = \frac{1}{16 \pi i } \int d^2 \theta \left( \frac{4 \pi i }{g_c^2 } + \frac{\theta_{YM}}{2 \pi} \right) W^a ( g_c V_c) W^a(g_c V_c) + h. c.  
\end{equation}
Using the canonical normalization $g_c V_c$ is a real superfield, imposing that $g_c$ is real.
For this reason with the canonical normalization the lagrangian is not holomorphic in $\tau$.
Thanks to holomorphicity, the holomorphic coupling is only renormalized at one-loop and the $\beta$ function can be computed exactly at one loop but its expression is different from \emph{NSVZ $\beta$ function}.
The cause of this mismatch is that the \emph{NSVZ $\beta$ function} is defined using the canonical (or physical) coupling constant and receives contribution from all orders in perturbation theory.

At first sight, one should expect that the expressions should match since the first two orders in $\alpha$ of the $\beta$ function are scheme independent. 
The reason why the two expressions differ is that the Jacobian of the transformation between canonical and holomorphic normalization is anomalous. 
Once the anomaly is taken into account the two expressions for the $\beta$ function agree. \\
An explicit calculation relating the comparison between the two different approach can be found in \cite{ArkaniHamed:1997mj}.



















\subsection{Superpotential: holomorphy and non-renormalization}
\label{sec:superpotential_hol_renorm}
Other than renormalization constraints, supersymmetry provides non-renormalization theorems for certain objects, such as the superpotential.
In \cite{Grisaru:1979wc} it has been demonstrated that the superpotential is tree-level exact, i.e. it does not receive correction in perturbation theory. 
However it usually receive contributions from non perturbative dynamics.

Perturbative calculations can be done using supergraphs, i.e. Feynman diagrams with superfields. 
The advantage of this approach is that supersymmetry is explicit and many semplification occur naturally. 
The demonstration is based on the fact that for general supersymmetric field theories, supergraph loops diagrams with $n$ external leg yield a term that can be written in the form
\begin{equation}
\int  d^4 x_1 \dots d^4 x_n d^2 \theta d^2 \bar{\theta} \; G (x_1 , \dots , x_n) \,F_1 ( x_1, \theta, \bar{\theta}) \dots  F_n ( x_n, \theta, \bar{\theta}) 
\end{equation} 
where $G (x_1 , \dots , x_n) $ is translationally invariant function.\\
The importance of this result is that all contribution from Feynman diagrams are given by a single integral over full superspace ($d^2 \theta d^2 \bar{\theta} $) whereas the superpotential must be written as an integral in half-superspace($d^2 \theta $ only) of chiral fields.
Exploiting the fact that a product of chiral fields is a chiral field, the most general form of a superpotential is 
\begin{equation}
 W (\lambda, \Phi) = \sum_{n=1}^{\infty} \left( \int d^2 \theta \, \lambda_n \Phi^n  \; +\;  \int d^2 \bar{\theta} \, \lambda_n^{\dagger} \bar{\Phi}^n \right)
 %\int d^2 \bar{\theta} \bar{\lambda}_n {\bar{\Phi}}^n
 \end{equation} 
 The second term of the superpotential is added in order to give a real lagrangian after the integration in superspace.
From the definition, we can see that the superpotential is holomorphic in the fields and in the coupling constants.

Fifteen years later, Seiberg \cite{Seiberg:1993vc} provided a proof of this theorem using a different approach.
He noted that the coupling constants $\lambda_n$ can be treated as background fields, i.e. chiral superfields with no dynamics.

Using this observation we can assign transformation laws to the coupling constants, making the lagrangian invariant under a larger symmetry.
Fields and coupling constants are charged under this symmetry and only certain combinations of them can appear in the superpotential.
%The superpotential in the weak coupling limit is subject to the constraints imposed by holomorphy and symmetry.
In addition, in a suitable weak coupling limit the effective superpotential must be identical with the tree-level one.
These conditions, taken together, fix the expansion of the superpotential to the expression of the tree-level potential.
A more detailed discussion can be found in \cite{Seiberg:1994bp} and \cite{Intriligator:1995au}.





\






\subsection{Moduli space}
Supersymmetric field theories have a larger set of vacua compared to ordinary field theories because of the presence of many scalar fields in the supermultiplets.

Lorentz invariance of the vacuum forbids fields with spin different from zero to acquire a vacuum expectation value.
With the same reasoning, derivatives of scalar fields must be set to zero because of translational invariance of the vacuum.
The scalar potential is the only term in the Lagrangian and in the Hamiltonian that can differ from zero.
As a result, the minimums of the scalar potential are in one-to-one correspondence with the states of minimal energy of the theory.

For $4D \; \mN=1$ gauge theories with matter, the scalar potential for the squarks reads
\begin{equation}
 V ( \phi_i, \bar{\phi}_j) = F \bar{F} + \frac{1}{2} D^2  \quad \overset{\mathclap{on-shell}}=   \quad \frac{\partial W}{\partial \phi_i} F^i \frac{\partial \bar{W}} {\partial \bar{\phi}_i} \bar{F}^i + \frac{g^2}{2} \sum_a | \bar{\phi}_j (T^a)^j_i \phi^j + \xi^a | ^2  \geq 0
\end{equation} 
$\xi^a$ is the Fayet-Iliopulos coefficient and differs from zero only for abelian factors of the gauge group.
The last equality is valid since $D$ and $F$ are auxiliary fields with no dynamics.
Their value is set by their equations of motion
\begin{equation}
 \bar{F}_i  = \frac{\partial \bar{W}} {\partial \bar{\phi}_i} \qquad D^a = - g \bar{\phi} T^a \phi - g \xi^a 
\end{equation}
Supersymmetric vacua are described by the sets of values of the scalar \emph{VEVs} that give a zero scalar potential. 
This requirement is equivalent to two different sets of equations, called $D-term$ and $F-term $ equations
\begin{equation}
\bar{F}^i ( \phi) = 0 \qquad D^a (\phi , \bar{\phi}) = 0
\end{equation}
\emph{F-term} equations are present only if there is a superpotential while the \emph{D-term} equations are always present.\\
If the minimum of the scalar potential is different from zero the vacuum is not supersymmetric. 
In this case supersymmetry is spontaneously broken.
Another possible situation is that the scalar potential has no minimum at all: the theory does not have any stable vacua.

The \emph{classical moduli space} is the set of solution of these equations for scalar \emph{VEVs} and represents the classical supersymmetric vacua of the theory. 
Gauge transformation should be taken into account in order to avoid redundance in the description.
The moduli space describe physically inequivalent vacua, since the mass spectrum of the theory depends on the \emph{VEVs} of the scalar fields, that differ in every point of the moduli space. 

Because of supersymmetry radiative corrections do not lift the energy of the ground state and the vacuum remains supersymmetric.
As a result, only superpotentials generated from non perturbative dynamics can lift the moduli space. 
We will see examples of this phenomenon in the analysis of SQCD.

An alternative description of the space of classical \emph{D-flat} directions is given by the space of all holomorphic gauge invariant polynomials of scalar fields modulo classical relations between them \cite{Luty:1995sd}.
As a result, gauge invariant polynomials of operators parametrize the classical moduli space of the theory. 
Using this description it's easier to find the moduli space of the theory in consideration.
If a superpotential is present, \emph{F-term} equations should be imposed on the gauge invariant polynomial used to describe \emph{D-flat} direction.
We will use this convenient description in the next chapters.










\subsection{Phases of gauge theories}
The dynamics of gauge theories can be classified according to the low-energy effective potential $V(R)$ between two test charges separated by a large distance $R$.
The possible forms of the potential, up to addictive constant, are
\begin{align}
\text{Coulomb} \qquad V(R) & \sim \frac{1}{R} \\
\text{free electric} \qquad V(R) & \sim \frac{1}{R \, \log(R \Lambda) } \\
\text{free magnetic} \qquad V(R) & \sim \frac{\log(R \Lambda)}{R \,  }\\
\text{Higgs} \qquad V(R) & \sim \text{constant} \\
\text{confining} \qquad V(R) & \sim  \sigma \, R  
\end{align}
The first three phases feature massless gauge fields and their potential is $V(R) \sim g^2(R) / R$ and they differ because of the renormalization of the charge in the IR.
In the Coulomb phase, $g^2_{IR} = \text{constant}$, while in the free abelian/non-Abelian phase the coupling constant goes to zero as $g^2(R) \sim 1/ \log(R \, \Lambda)$.
The free electric phases is possible for abelian or non-Abelian theories.In the latter case for asymptotically free theories it's necessary that the renormalization group has a non trivial infrared fixed point.
The free magnetic phases is generated by magnetic monopoles acting as source of the field. Since magnetic and electric charges are related by Dirac quantization condition, the running of the coupling constant for magnetic monopoles is the inverse of electric charges.\\
The situation is completely different in the last two cases.
In the Higgs phase gauge fields are massive and the potential is given by a Yukawa potential, exponential suppressed at long distances that results in a constant value.
The confining phase can be described by tube of confined gauge flux between the charges which, at large distances, acts as a string with costant tension, yielding a linear potential.


\subsection{'t Hooft anomaly matching conditions}

The description of strongly coupled field theories is usually made using gauge invariant composite operators in the infrared.
The 't Hooft anomaly matching conditions are a great tool to investigate the low-energy degrees of freedom of the theory.\\
They provide a strong check whether global UV symmetries are broken by the strong dynamics in the infrared or they are still a symmetry for the composite fields.

Let's consider an asymptotically free gauge theory with global symmetry group $G$. 
Gauge symmetries can't be anomalous because that would spoil the unitarity of the theory but there's nothing wrong with the global symmetries in being anomalous.\\
We can compute the triangle anomaly for the global symmetry group in the ultraviolet and we will call it $A_{UV}$. 
After weakly gauging $G$ we introduce additional fermions that are charged only under $G$ in order to cancel the anomaly, since now it is a gauge symmetry.\\
Flowing towards the infrared, the anomaly is still zero if the global symmetry group is not broken.
After constructing the low-energy effective field theory, 
we can calculate the triangle anomalies for the group $G$ involving the composite low energy fields which results in the term $A_{IR}$.
Since the fermions we added contribute to the anomaly with the same term we have that 
\begin{equation}
 0 = A_{IR} + A_{F} = A_{UV} + A_{F} \quad \rightarrow \quad A_{IR} = A_{UV}
\end{equation}
The anomaly coefficient can be easily computed since it is proportional to the group theoretic factor
\begin{equation}
 A = \Tr \; ( T^a \{ T^b, T^c \})
\end{equation}
Summarizing the result, we found that if the global symmetry group is not broken by the strong dynamics, triangle anomalies involving only the global symmetry group should be equal in the ultraviolet and in the infrared.\\
Moreover, we will use these anomaly matching conditions to find if two different theories are invariant under the same global symmetries in the IR as an additional check of electric magnetic duality.


